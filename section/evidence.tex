\section*{General Reflection}

\subsection*{Key Principle}
Probably the most important thing in evidence law is that evidence needs to be 

Evidence law focuses on the issue of relevance. Relevance, in this context, determines whether the evidence directly responds to or supports the arguments presented by counsel or the prosecutor. The key question is whether the evidence is probative in relation to the legal arguments being made.

Probably one of the most important things in evidence law is that evidence needs to be about the fact. Therefore opinions are usually not admissible. The evidence needs to help jury to determine the fact. 

The materials used in arguments are either prejudicial or probative. To ensure a fair trial, the court excludes prejudicial materials while retaining probative materials as evidence. Therefore, most rules that govern evidence admission are designed to protect the fairness of the trial.  

The probative evidence refers to evidence that has the ability to prove or disprove a fact in issue in a legal proceeding. The probative value of evidence is assessed based on its tendency to make a fact more or less probable than it would be without the evidence.

Therefore, the key principle is the evidence needs to be relevant and probative to be admitted. 

\subsection*{Type of Evidence}
There are two types of evidence: real evidence and documentary evidence. Real evidence consists of physical objects or events presented directly in the courtroom. Documentary evidence, on the other hand, includes written or recorded materials and is generally classified as hearsay, making it inadmissible under common law unless an exception applies. To admit documentary evidence, specific exceptions must be met, such as an affidavit in a civil case.
\newpage
\section*{Source of Law}
\subsection*{Statue}
\begin{itemize}
    \item Common Law
    \item \textit{Evidence Act 1995}(Cth) (EAC)
    \item \textit{Evidence Act 1977}(Qld) (EAQ)
        \begin{itemize}
            \item If a commonwealth matter is being litigated in a Qld court, the Qld's evidence law applies\footnote{\textit{Judiciary Act 1903}(Cth),s79}
                \begin{description}
                    \item[Example:]  If a CTH criminal offence is being tried in a QLD court, the EAQ applies.
                    \item[Example:] If a civil matter is being litigated in a QLD court under, say, the \textit{Competition and Consumer Act 2010}(Cth), the EAQ applied
                \end{description}
        \end{itemize}
\end{itemize}

\subsection*{Source of important concept}
\begin{itemize}
    \item Burden of Prove \& Beyond Reasonable doubt (\textbf{Criminal Case})
        \begin{description}
            \item[Case:]\textit{Woolmington v DPP}[1935] AC 463
            \item[Note:]the Crown bears the ultimate burden of proving the guilt of an accused person beyond reasonable doubt 
        \end{description}
    \item Legal evidential burden \& on the balance of probabilities(\textbf{Civil Case})
        \begin{description}
            \item[Case:] \textit{Briginshaw v Briginshaw}(1938) 60 CLR 336,363
            \item[Note:]Balance of probabilities means that the nature of the issue necessarily affects the process by which reasonable satisfaction is attained
        \end{description}
    \item Hearsay
        \begin{description}
            \item[Principle:] inadmissible\footnote{J D Heydon, \textit{Cross on Evidence} 7th Aust ed, Lexis Nexis Butterworths, Sydney, 2004, [31030]}
            \item[Exceptions:] admissible
                \begin{itemize}
                    \item Res gestae
                        \begin{description}
                            \item[Note:]The very matters into which the court is enquiring
                            \item[Contemporaneity:] \textit{O'Leary v R} (1946) 73 CLR 566
                            \item[Spontaneity:]\textit{Vocisano v Vocisano}(1974) 130 CLR 267,273
                        \end{description}
                    \item Declarations by persons now deceased
                        \begin{description}
                            \item[Note:] Must be deceased, missing does not count\footnote{\textit{Evidence Act 1977}(Qld) s93(1)(b)}
                            \item[Case:]\textit{Sussex Peerage Case}(1844) 8 ER 1034 
                        \end{description}
                    \item Declaration against pecuniary or proprietary interest
                        \begin{description}
                            \item[Case:] \textit{Higham v Ridgway}(1808) 103 ER 717 
                        \end{description}
                    \item Declaration made in the course of duty
                        \begin{description}
                            \item[Case:] \textit{Henry Coxon} (1878) LR 3PD 156
                        \end{description}

                    \item Dying declaration
                        \begin{description}
                            \item[Case:] \textit{R v Bernadotti} (1869) 11 Cox CC 316 
                        \end{description}
                \end{itemize}
        \end{description}
    \item Privilege
        \begin{description}
            \item[Note:] Privilege means that the accused or person can refuse answering questions. Privilege applies to both oral and documentary evidence. 
            \item[Privilege against self-discrimination:]EAQ s10 
            \item[Legal Profession Privilege:]Protects communications between lawyers and clients for the dominant purpose of seeking legal advice or preparing for actual or contemplated litigation
                \begin{itemize}
                    \item \textit{Baker v Campbell}[1983] HCA 39 -- LPP protects the disclosure of communications.
                    \item valid LPP can be abrogated by statue by the use of express words\footnote{\textit{Daniels Corporation International Pty Ltd v Australian Competition and Consumer Commission}[2002]HCA 49.}For example, in \textit{Crime and Corruption Act 2001 (Qld)} section 190 explicitly states that a witness cannot refuse to answer questions on the grounds of self-incrimination when appearing before the Crime and Corruption Commission (CCC)
                    \item valid LPP can be abrogated by statue by necessary implication(by public interests)
                \end{itemize}
            \item[Without Prejudice Privilege:]\textit{Field v Commissioner for Railways (NSW)}
            \item[Wavier:] In principle, only the privilege holder can waive the privilege. The waiver can be expressed or implied, deliberate or inadvertent, or imputed by law. However there is a public interest immunity.\footnote{\textit{Sankey v Whitlam} [1978] HCA 43}
                \begin{itemize}
                    \item The privilege can be wavied due to inconsistent with maintaining the confidentiality of the documents.\footnote{\textit{Mann v Carnell}(1999) 168 ALR 86}
                
                \end{itemize}
            \item[Direction:]especially when accused refuse to answer questions, the judge should warn the jury that no inference can be drawn as to his/her guilt from that silence.\footnote{\textit{Sanchez v R}; Direction 29}
            \item[Weissensteiner:] remember Weissensteiner v R: Exception applies when accused uniquely has knowledge of facts but chooses to remain silent – Cannot be directly used to draw adverse inference, but only in weighing prosecution evidence. 
        \end{description}
    \item Competent
        \begin{description}
            \item[Note:] Everyone is competent.\footnote{EAQ, s9}
        \end{description}

    \item Compellable
        \begin{description}
            \item[Note:]The general rule is that a competent witness is also compellable to testify\footnote{\textit{ACC v Stoddart}; EAQ s7} as competence implies compellability.
            \item[Direction:]It is prohibited for the judge to warn or suggest that the law regards any class of persons as unreliable witnesses.\footnote{\textit{QCC},s632(3).}
            \item[Character:] may be of good or bad character. It may be about the character of the accused, witness and/or the victim.
                \begin{itemize}
                    \item Character of accused.
                        \begin{itemize}
                            \item Accused Leading Evidence of own Good Character.\footnote{\textit{R v Rowton}(1865) 169 R 1497}
                            \item Prosecution rebutting Accused’s Good Character, by leading Evidence of Bad Character.\footnote{\textit{R v Rowton}(1865) 169 R 1497}
                            \item Co-accused Leading Evidence of the Accused’s Bad Character.\footnote{\textit{Lowery v R}[1974] AC 85}
                        \end{itemize}
                    \item Character of a witness
                        \begin{itemize}
                            \item Prior inconsistent statements\footnote{EAQ, ss 18-19}
                            \item Prior convictions of the witness may be proved\footnote{EAQ, s 16}
                            \item Credibility
                                \begin{description}
                                    \item[Admission:] Previous convictions of a witness are admissible on the issue of their credit Subject to EAQ, s 15A
                                    \item[Limitation:] EAQ s20,s21 
                                \end{description}
                        \end{itemize}
                \end{itemize}
        \end{description}

    \item Opinion
        \begin{description}
            \item[Note:] Generally opinions are inadmissible. Except expert opinions and lay opinions.
            \item[Expert opinion:]
                \begin{itemize}
                    \item Is an expert opinion necessary?\footnote{\textit{Farrell v R}(1998) 194 CLR 286}
                    \item Is it a recognised field of expertise?\footnote{\textit{Clark v Ryan}(1960) 103 CLR 486}
                    \item Is the witness truly an expert within that field?\footnote{\textit{Clark v Ryan}(1960) 103 CLR 486}
                    \item Does the witness’s evidence stay within their range of expertise?\footnote{R v Mackenzie [2016] QCA 277}
                    \item Is the witness’ testimony based on admissible evidence?\footnote{Ramsay v Watson (1961) 108 CLR 462;inadmissible}
                \end{itemize}
                
            \item[Lay opinion:]Opinion evidence by laypersons (‘non-experts’) may be admissible\footnote{\textit{Sherrard v Jacob }[1965] NI 151}.
                \begin{itemize}
                    \item can be used for certain issues, such as handwriting identification, etc. 
                \end{itemize}
        \end{description}
        
        \item Confession
        		\begin{description}
			\item[Note:] Confession, by its nature, is an out of court statement (Hearsay). However, confession may be admissible. Confession can still be excluded under the judge's discretions (common law practice) or by statue.
			\item[Mandatory exclusion:] Involuntary confession.\footnote{ \textit{ Criminal Law Amendment Act 1894}(Qld) s10}
				\begin{itemize}
					\item Involuntary includes duress, intimidation, presistent importunity, sustained, or even undue insistence or pressure\footnote{ \textit{McDermott v R}(1948) 76 CLR 501}
					\item Standards of voluntariness includes free from threats, freedom from compulsion, coercion.\footnote{\textit{Tofilau v R}(2007)231 CLR 396}
				\end{itemize}
			\item[Discretion Exclusion:] Unfairness\footnote{R v Swaffield; Pavic v R (1998) 192 CLR 159}, public policy\footnote{R v Ireland (1970) 126 CLR 32}
		\end{description}
	\item Label-Documentary
		\begin{description}
			\item[Note:] Label needs to have witness to testify the label itself. Otherwise, it will be inadmissiable due to lack of evidence chain. 
			\item[Case:] \textit{Commissioner for Railways (NSW) v Young}(1962) 106 CLR 535
			\item[Case:] \textit{Godfrey v Woolworths (WA) Pty L:td}(1998) 103 A Crim R336.
		\end{description}
	
	\item Corroboration
		\begin{description}
			\item[QLD practice:] No need for corroboration.\footnote{\textit{QCC}, s632}
			\item[Jury Warning:] A judge may give a corroboration warning, but any warning that is given is subject to s 632(3) of QCC.\footnote{\textit{Robinson v R }(1999) 197 CLR 162} The need for a warning is contextual.\footnote{\textit{Longman v R}(1989) 64 ALJR 73}
			\item[Give Warning:] There are exceptional cases requires warning direction. 
				\begin{itemize}
				\item verballed confession was made by the accused while in police custody.\footnote{\textit{McKinney v R }(1991) 171 CLR 468}
				\item corroborative evidence is needed before relying solely on an eye-witness identification by a stranger.\footnote{\textit{Domican v R }(1992) 173 CLR 555}
				\item If Prisoner A has their own reasons for lying about Prisoner B confessing to a crime while they were in prison together, this is known as a "jail-yard confession."\footnote{\textit{Pollitt v R} (1992) 174 CLR 558} 
				\end{itemize}
		\end{description}
\end{itemize}