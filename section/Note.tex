\subsection{Sources of Evidence Law}\label{sources-of-evidence-law}

\subsubsection{Common Law and Evidence
Acts}\label{common-law-and-evidence-acts}

\begin{itemize}
\tightlist
\item
  \textbf{Evidence Act 1995 (Cth)} (EAC): Applies in Federal Courts.
\item
  \textbf{Evidence Act 1997 (QLD)} (EAQ): Applies in Queensland Courts.
\end{itemize}

\paragraph{Key Points:}\label{key-points}

\begin{itemize}
\tightlist
\item
  If a Commonwealth matter is litigated in a State court, the state's
  own evidence law applies.
\item
  \textbf{Judiciary Act 1903 (Cth), s 79}: States the laws of each state
  or territory, including those relating to procedure, evidence, and
  witness competency, are binding on all courts exercising federal
  jurisdiction in that state or territory.
\end{itemize}

\subsubsection{Application of Evidence
Acts}\label{application-of-evidence-acts}

\begin{longtable}[]{@{}
  >{\raggedright\arraybackslash}p{(\linewidth - 2\tabcolsep) * \real{0.5000}}
  >{\raggedright\arraybackslash}p{(\linewidth - 2\tabcolsep) * \real{0.5000}}@{}}
\toprule\noalign{}
\begin{minipage}[b]{\linewidth}\raggedright
Scenario
\end{minipage} & \begin{minipage}[b]{\linewidth}\raggedright
Applicable Law
\end{minipage} \\
\midrule\noalign{}
\endhead
\bottomrule\noalign{}
\endlastfoot
CTH criminal offence in QLD court & EAQ \\
Civil matter under Commonwealth law in QLD & EAQ \\
\end{longtable}

\subsubsection{Benchbooks}\label{benchbooks}

\begin{itemize}
\tightlist
\item
  Benchbooks provide judges with standard directions to juries to
  minimize risks of inappropriate directions.
\item
  \textbf{R v BCC {[}2006{]} QCA 435}: Mackenzie J noted that while the
  Benchbook is a useful resource, modifications may be required for
  specific cases.
\end{itemize}

\subsection{Burden and Standard of
Proof}\label{burden-and-standard-of-proof}

\subsubsection{Definitions}\label{definitions}

\begin{quote}
\textbf{Burden/onus of proof}: The obligation on a party seeking to have
a particular issue decided in their favor to provide sufficient evidence
to support their contention.
\end{quote}

\subsubsection{Types of Burdens}\label{types-of-burdens}

\begin{longtable}[]{@{}
  >{\raggedright\arraybackslash}p{(\linewidth - 2\tabcolsep) * \real{0.6364}}
  >{\raggedright\arraybackslash}p{(\linewidth - 2\tabcolsep) * \real{0.3636}}@{}}
\toprule\noalign{}
\begin{minipage}[b]{\linewidth}\raggedright
Burden Type
\end{minipage} & \begin{minipage}[b]{\linewidth}\raggedright
Definition
\end{minipage} \\
\midrule\noalign{}
\endhead
\bottomrule\noalign{}
\endlastfoot
\textbf{Legal / Persuasive Burden} & The burden of satisfying the court
on a particular issue to the appropriate standard (s 13 CC (Cth)). \\
\textbf{Evidential Burden} & The burden of providing enough evidence
suggesting a reasonable possibility that a matter exists (s 13 CC
(Cth)). \\
\end{longtable}

\subsubsection{Standard of Proof}\label{standard-of-proof}

\begin{itemize}
\tightlist
\item
  \textbf{Standard of proof} (or \textbf{Quantum of proof}): The amount
  and quality of evidence needed to discharge the ultimate burden of
  proof.
\end{itemize}

\paragraph{Key Standards:}\label{key-standards}

\begin{itemize}
\tightlist
\item
  \textbf{Balance of Probabilities (BOP)}: Standard applied in civil
  cases.
\item
  \textbf{Beyond Reasonable Doubt (BRD)}: Standard applied in criminal
  cases.
\end{itemize}

\subsubsection{Jury Instructions on BRD}\label{jury-instructions-on-brd}

\begin{itemize}
\tightlist
\item
  \textbf{R v Chatzidimitriou (2000)}: Attempts to elaborate on BRD have
  often led to appeals.
\item
  \textbf{R v Irlam; Ex parte A-G of Qld {[}2002{]}}: Approved direction
  that a reasonable doubt is a doubt that jury members consider
  reasonable based on evidence.
\end{itemize}

\subsection{Criminal Cases}\label{criminal-cases}

\subsubsection{General Rule}\label{general-rule}

\begin{itemize}
\tightlist
\item
  Under QLD law, the Crown bears the ultimate burden of proving the
  guilt of an accused beyond reasonable doubt, a principle established
  by \textbf{Woolmington v DPP {[}1935{]} AC 462}.
\end{itemize}

\subsubsection{Presumption of Innocence}\label{presumption-of-innocence}

\begin{quote}
``No matter what the charge or where the trial, the principle that the
prosecution must prove the guilt of the prisoner is part of the common
law of England.''
\end{quote}

\subsubsection{Specific Cases}\label{specific-cases}

\begin{itemize}
\tightlist
\item
  In cases where intent is part of a crime, the defendant does not need
  to prove the act was accidental.
\item
  \textbf{Woolmington Case}: Lord Sankey articulated that if there's a
  reasonable doubt from the evidence, the prosecution has not made its
  case, leading to acquittal.
\end{itemize}

\subsubsection{Legal Burden in Criminal
Cases}\label{legal-burden-in-criminal-cases}

\begin{longtable}[]{@{}
  >{\raggedright\arraybackslash}p{(\linewidth - 2\tabcolsep) * \real{0.5676}}
  >{\raggedright\arraybackslash}p{(\linewidth - 2\tabcolsep) * \real{0.4324}}@{}}
\toprule\noalign{}
\begin{minipage}[b]{\linewidth}\raggedright
Role
\end{minipage} & \begin{minipage}[b]{\linewidth}\raggedright
Burden of Proof
\end{minipage} \\
\midrule\noalign{}
\endhead
\bottomrule\noalign{}
\endlastfoot
\textbf{Crown} & Prove each element of the offence beyond reasonable
doubt. \\
\textbf{Defence} & Legal burden in recognized situations (e.g.,
insanity). \\
\end{longtable}

\subsection{Insanity and Legal
Burden}\label{insanity-and-legal-burden}

\paragraph{Insanity Defense}\label{insanity-defense}

\begin{itemize}
\tightlist
\item
  Presumed sound mind until proven otherwise (s 26 QC (Cth)).
\item
  Defence must establish that the accused lacked sufficient soundness of
  mind at the time of the crime.
\end{itemize}

\paragraph{Statutory Exceptions}\label{statutory-exceptions}

\begin{itemize}
\tightlist
\item
  \textbf{Diminished Responsibility} (s 304A CC (Qld)): Defence that
  reduces murder to manslaughter if the accused suffered from an
  abnormality of mind akin to insanity.
\item
  \textbf{Lawful Possession of Dangerous Drugs} (s 124 Drugs Misuse Act
  1986 (Qld)): Provides a defence for possession if acquired lawfully.
\end{itemize}

\subsection{ Legal Burdens in Criminal
Cases}\label{legal-burdens-in-criminal-cases}

\subsubsection{General Principles}\label{general-principles}

\begin{itemize}
\tightlist
\item
  The \textbf{legal burden} generally lies with the Crown to prove the
  guilt of the accused beyond a reasonable doubt (BRD).
\item
  The \textbf{accused} has the burden to establish certain defenses,
  which may shift the legal burden in specific circumstances.
\end{itemize}

\subsubsection{Implied Statutory
Exceptions}\label{implied-statutory-exceptions}

\begin{itemize}
\tightlist
\item
  The case of \textbf{DPP v United Telecasters (Sydney) Ltd (1990)}
  highlights that statutory exceptions can imply a burden of proof on
  the accused.
\item
  \textbf{Key Point:} The intention to impose a burden on the accused
  may be evident in how the statutory offense is expressed.
\end{itemize}

\subsubsection{Evidential Burden}\label{evidential-burden}

\begin{itemize}
\tightlist
\item
  The \textbf{evidential burden} is placed on the accused when raising
  facts related to an excuse or justification.
\item
  The prosecution must prove that the excuse or justification does not
  exist, BRD.
\end{itemize}

\begin{longtable}[]{@{}
  >{\raggedright\arraybackslash}p{(\linewidth - 4\tabcolsep) * \real{0.2927}}
  >{\raggedright\arraybackslash}p{(\linewidth - 4\tabcolsep) * \real{0.2845}}
  >{\raggedright\arraybackslash}p{(\linewidth - 4\tabcolsep) * \real{0.4228}}@{}}
\toprule\noalign{}
\begin{minipage}[b]{\linewidth}\raggedright
Burden Type
\end{minipage} & \begin{minipage}[b]{\linewidth}\raggedright
Who Bears It
\end{minipage} & \begin{minipage}[b]{\linewidth}\raggedright
Condition
\end{minipage} \\
\midrule\noalign{}
\endhead
\bottomrule\noalign{}
\endlastfoot
Legal Burden & Crown & To prove each element of the offense BRD \\
Evidential Burden & Accused & When raising facts for
justification/excuse \\
\end{longtable}

\subsubsection{Example Scenario}\label{example-scenario}

\begin{itemize}
\tightlist
\item
  If the Crown fails to show evidence of lack of consent in a rape case,
  it fails to meet the legal burden, and the case may be withdrawn from
  the jury.
\end{itemize}

\subsection{ Defenses and Their
Burdens}\label{defenses-and-their-burdens}

\subsubsection{Common Law Principles}\label{common-law-principles}

\begin{itemize}
\tightlist
\item
  The general rule is that the Crown bears the evidential burdens for
  elements of the offense.
\item
  The \textbf{accused} bears the evidential burden for defenses that
  involve mental impairment or lawful excuses.
\end{itemize}

\begin{longtable}[]{@{}
  >{\raggedright\arraybackslash}p{(\linewidth - 4\tabcolsep) * \real{0.3252}}
  >{\raggedright\arraybackslash}p{(\linewidth - 4\tabcolsep) * \real{0.3659}}
  >{\raggedright\arraybackslash}p{(\linewidth - 4\tabcolsep) * \real{0.3089}}@{}}
\toprule\noalign{}
\begin{minipage}[b]{\linewidth}\raggedright
Defense Type
\end{minipage} & \begin{minipage}[b]{\linewidth}\raggedright
Elements
\end{minipage} & \begin{minipage}[b]{\linewidth}\raggedright
Burden Type
\end{minipage} \\
\midrule\noalign{}
\endhead
\bottomrule\noalign{}
\endlastfoot
Mental Impairment (e.g., insanity) & Suggests mental impairment &
Accused bears evidential and legal burden \\
Intoxication & Claims intoxication as a defense & Accused bears
evidential burden \\
Automatism & Suggests lack of control & Accused bears evidential
burden \\
Self-Defense & Justifies actions taken in self-defense & Accused bears
evidential burden \\
Duress & Claims actions were forced & Accused bears evidential burden \\
Honest and Reasonable Mistake of Fact & Justifies actions based on
mistaken belief & Accused bears evidential burden \\
\end{longtable}

\subsection{ No Case to Answer}\label{no-case-to-answer}

\begin{itemize}
\tightlist
\item
  In a criminal trial, a judge may rule that there is \textbf{no case to
  answer} if insufficient evidence exists for a jury to find guilt
  beyond a reasonable doubt.
\item
  This ruling is primarily factual, although it involves a legal
  assessment.
\end{itemize}

\subsection{Legal Burden in Civil
Cases}\label{legal-burden-in-civil-cases}

\subsubsection{General Principles}\label{general-principles-1}

\begin{itemize}
\tightlist
\item
  \textbf{Legal Burden:} The party asserting a claim must prove it.
\item
  \textbf{Evidential Burden:} The party raising a defense must prove its
  elements on the balance of probabilities.
\end{itemize}

\begin{longtable}[]{@{}
  >{\raggedright\arraybackslash}p{(\linewidth - 4\tabcolsep) * \real{0.2258}}
  >{\raggedright\arraybackslash}p{(\linewidth - 4\tabcolsep) * \real{0.3764}}
  >{\raggedright\arraybackslash}p{(\linewidth - 4\tabcolsep) * \real{0.3978}}@{}}
\toprule\noalign{}
\begin{minipage}[b]{\linewidth}\raggedright
Case Type
\end{minipage} & \begin{minipage}[b]{\linewidth}\raggedright
Burden Type
\end{minipage} & \begin{minipage}[b]{\linewidth}\raggedright
Who Bears It
\end{minipage} \\
\midrule\noalign{}
\endhead
\bottomrule\noalign{}
\endlastfoot
Civil Cases & Ultimate Burden & Crown must prove cause of action \\
& Shifts to Defendant after prima facie case & Defendant must prove
defense elements \\
\end{longtable}

\subsubsection{Balance of Probabilities}\label{balance-of-probabilities}

\begin{itemize}
\tightlist
\item
  The standard of proof in civil cases is based on the \textbf{balance
  of probabilities}.
\item
  To discharge the ultimate burden, a party must provide evidence
  allowing the court to draw a positive inference.
\end{itemize}

\subsubsection{Briginshaw Principle}\label{briginshaw-principle}

\begin{itemize}
\tightlist
\item
  The \textbf{Briginshaw principle} states that more serious allegations
  require more persuasive proof.
\item
  This principle applies even in civil cases where allegations may
  involve criminal conduct.
\end{itemize}

\subsection{Roles and Functions of the Judge and
Jury}\label{roles-and-functions-of-the-judge-and-jury}

\begin{itemize}
\tightlist
\item
  In cases without a jury, the judge (or magistrate) has the sole
  responsibility for evaluating evidence and making findings of fact.
\end{itemize}

\subsection{Role of Judges and Juries in
Trials}\label{role-of-judges-and-juries-in-trials}

\subsubsection{Judge's Responsibilities}\label{judges-responsibilities}

\begin{itemize}
\tightlist
\item
  \textbf{Deciding Legal Questions}: The judge is responsible for
  determining what evidence can be considered during a trial. This
  involves distinguishing between \textbf{questions of law} and
  \textbf{questions of fact}.
\item
  \textbf{Trier of Fact}: In summary criminal trials and judge-only
  civil matters, the judge acts as the trier of fact. In such cases,
  they evaluate the evidence presented and make factual determinations.
\end{itemize}

\subsubsection{Jury's Role}\label{jurys-role}

\begin{itemize}
\tightlist
\item
  \textbf{Ultimate Tribunal}: When a jury is present, they serve as the
  ultimate tribunal for deciding issues of fact, particularly in serious
  crimes. As established in \emph{R v Baden-Clay} {[}2016{]} HCA 35, the
  jury is constitutionally designated to decide factual issues.
\end{itemize}

\subsubsection{Judge's Function in Jury
Trials}\label{judges-function-in-jury-trials}

\begin{itemize}
\tightlist
\item
  Judges direct juries on points of law and ensure they are not misled
  by inadmissible evidence as outlined in \textbf{s 620 Criminal Code
  (Qld)}.
\item
  The judge assists the jury in understanding how certain evidence may
  impact the accused's culpability, summarizing both the relevant law
  and evidence.
\end{itemize}

\subsubsection{Voir Dire Procedure}\label{voir-dire-procedure}

\begin{itemize}
\tightlist
\item
  A \textbf{voir dire} is conducted when a judge must make factual
  determinations before resolving a legal question. This is often
  necessary to decide the admissibility of evidence.
\end{itemize}

\begin{longtable}[]{@{}
  >{\raggedright\arraybackslash}p{(\linewidth - 2\tabcolsep) * \real{0.3186}}
  >{\raggedright\arraybackslash}p{(\linewidth - 2\tabcolsep) * \real{0.6814}}@{}}
\toprule\noalign{}
\begin{minipage}[b]{\linewidth}\raggedright
Aspect
\end{minipage} & \begin{minipage}[b]{\linewidth}\raggedright
Description
\end{minipage} \\
\midrule\noalign{}
\endhead
\bottomrule\noalign{}
\endlastfoot
\textbf{Definition} & A mini hearing on the admissibility of
evidence. \\
\textbf{Purpose} & To determine factual conclusions before legal
rulings. \\
\textbf{Process} & Conducted outside the presence of the jury. \\
\end{longtable}

\subsubsection{Trial Judge's Duties}\label{trial-judges-duties}

\begin{itemize}
\tightlist
\item
  A trial judge must ensure a fair trial for the accused and provide
  adequate jury direction based on the evidence. They must summarize the
  evidence accurately at the end of the trial.
\end{itemize}

\subsubsection{Admissibility of
Evidence}\label{admissibility-of-evidence}

\begin{itemize}
\tightlist
\item
  Judges decide on the admissibility of contested evidence, which may
  involve evaluating conflicting factual allegations. This is typically
  performed in a voir dire process.
\end{itemize}

\subsubsection{Evidence Types}\label{evidence-types}

\paragraph{Direct Evidence}\label{direct-evidence}

\begin{itemize}
\tightlist
\item
  \textbf{Definition}: Evidence that directly proves a fact without
  needing inference.
\item
  \textbf{Examples}: Observations or perceptions related to the fact in
  issue.
\item
  \textbf{Limitations}: Some facts, such as states of mind, cannot be
  directly observed.
\end{itemize}

\paragraph{Circumstantial Evidence}\label{circumstantial-evidence}

\begin{itemize}
\tightlist
\item
  \textbf{Definition}: Indirect evidence that requires inference to draw
  conclusions about a fact.
\item
  \textbf{Nature}: It serves as proof of a principal fact through
  secondary facts (e.g., motive, opportunity).
\end{itemize}

\begin{longtable}[]{@{}
  >{\raggedright\arraybackslash}p{(\linewidth - 2\tabcolsep) * \real{0.3103}}
  >{\raggedright\arraybackslash}p{(\linewidth - 2\tabcolsep) * \real{0.6897}}@{}}
\toprule\noalign{}
\begin{minipage}[b]{\linewidth}\raggedright
Type
\end{minipage} & \begin{minipage}[b]{\linewidth}\raggedright
Characteristics
\end{minipage} \\
\midrule\noalign{}
\endhead
\bottomrule\noalign{}
\endlastfoot
\textbf{Direct Evidence} & Directly related to a fact in issue,
observable. \\
\textbf{Circumstantial Evidence} & Requires inference, not directly
observable. \\
\end{longtable}

\subsubsection{Important Cases}\label{important-cases}

\begin{itemize}
\tightlist
\item
  \emph{R v Mawson} {[}1967{]} VR 205: Judges may express non-binding
  opinions on evidence but must clarify to the jury that they are the
  triers of fact.
\item
  \emph{Plomp v R} (1963) 110 CLR 234: In wholly circumstantial cases, a
  guilty inference can only be rationally made if it is the only
  reasonable inference available on all evidence.
\end{itemize}

\subsubsection{Pre-Trial Procedures}\label{pre-trial-procedures}

\begin{itemize}
\item
  Since 2008, some trials in higher courts may occur without a jury. In
  such cases, the judge determines both legal and factual issues.
\item
  \textbf{QCC, s 590AA}: This section allows parties to seek pre-trial
  directions regarding the admissibility of evidence among other issues.
\end{itemize}

\begin{longtable}[]{@{}
  >{\raggedright\arraybackslash}p{(\linewidth - 2\tabcolsep) * \real{0.2955}}
  >{\raggedright\arraybackslash}p{(\linewidth - 2\tabcolsep) * \real{0.7045}}@{}}
\toprule\noalign{}
\begin{minipage}[b]{\linewidth}\raggedright
Provision
\end{minipage} & \begin{minipage}[b]{\linewidth}\raggedright
Description
\end{minipage} \\
\midrule\noalign{}
\endhead
\bottomrule\noalign{}
\endlastfoot
\textbf{Pre-Trial Directions} & Judge may direct parties on trial
conduct and issues. \\
\textbf{Binding Rulings} & Directions are binding unless reopened by the
presiding judge. \\
\end{longtable}

\subsubsection{Summary of Key Points}\label{summary-of-key-points}

\begin{itemize}
\tightlist
\item
  The judge has a dual role in trials, determining both legal and
  factual aspects, particularly in the absence of a jury.
\item
  Jurors are crucial for determining facts in serious criminal cases,
  guided by the judge's instructions.
\item
  Evidence can be categorized into direct and circumstantial, each
  serving different roles in proving facts in issue.
\end{itemize}

\subsection{Legal Standards of Proof}\label{legal-standards-of-proof}

\subsubsection{Burden of Proof}\label{burden-of-proof}

\begin{quote}
``The prosecution bears the burden of proving all the elements of the
crime beyond reasonable doubt.''
\end{quote}

\begin{itemize}
\tightlist
\item
  This means that \textbf{all essential ingredients} of each element
  must be proven.
\item
  It does not imply that \textbf{every fact or piece of evidence} relied
  upon must be individually proven beyond reasonable doubt.
\end{itemize}

\subsubsection{Distinction of Evidence
Types}\label{distinction-of-evidence-types}

\begin{itemize}
\tightlist
\item
  \textbf{Strands in a Rope vs.~Links in a Chain}:

  \begin{itemize}
  \tightlist
  \item
    Indispensable links in a chain must be proved beyond reasonable
    doubt.
  \item
    A judge is only required to direct the jury on this if deemed
    necessary.
  \end{itemize}
\end{itemize}

\subsection{Case Study: R v
Baden-Clay}\label{case-study-r-v-baden-clay}

\subsubsection{Evidence for the
Prosecution}\label{evidence-for-the-prosecution}

\begin{itemize}
\tightlist
\item
  \textbf{Gerard Baden-Clay's Affair}: Suggests motive due to his double
  life.
\item
  \textbf{Financial Stress}: Business debts exceeding \$500,000 may have
  contributed to the stress leading to the alleged crime.
\item
  \textbf{Physical Evidence}:

  \begin{itemize}
  \tightlist
  \item
    Scratches on his face were considered more consistent with
    fingernail scratches than shaving cuts.
  \item
    DNA evidence from under Mrs.~Baden-Clay's fingernails.
  \item
    Leaves from six species found on Mrs.~Baden-Clay's body, all present
    around the Baden-Clay residence.
  \item
    Blood DNA match found in the boot of the Baden-Clay car, which was
    only owned for a brief period.
  \end{itemize}
\item
  \textbf{Timeline}: Gerard's mobile phone was charged at 1:48 am on the
  night of his wife's death, contradicting his claim of being in bed by
  10 pm.
\item
  \textbf{Legal Representation}: Hired a criminal defense lawyer before
  reporting his wife missing.
\end{itemize}

\subsubsection{Evidence for the Defence}\label{evidence-for-the-defence}

\begin{itemize}
\tightlist
\item
  \textbf{History of Depression}: Argued that Mrs.~Baden-Clay may have
  taken her own life or overdosed.
\item
  \textbf{Lack of Direct Evidence}: Claimed no evidence linked Gerard to
  the crime scene.
\item
  \textbf{Character Evidence}: Presented as a good businessman and
  community leader.
\item
  \textbf{Infidelity}: Admitted to affairs but contended that moral
  failings do not equate to murder.
\end{itemize}

\subsubsection{Verdict and Appeal}\label{verdict-and-appeal}

\begin{itemize}
\tightlist
\item
  \textbf{Initial Verdict}: Jury found Gerard Baden-Clay guilty.
\item
  \textbf{Appeal Arguments}:

  \begin{itemize}
  \tightlist
  \item
    The jury should have been directed on the need for satisfaction
    beyond reasonable doubt for certain critical facts.
  \item
    The court found that neither the finding of blood in the Captiva nor
    that he placed his wife's body in the creek were indispensable to
    concluding that he killed her.
  \end{itemize}
\end{itemize}

\begin{longtable}[]{@{}
  >{\raggedright\arraybackslash}p{(\linewidth - 2\tabcolsep) * \real{0.2500}}
  >{\raggedright\arraybackslash}p{(\linewidth - 2\tabcolsep) * \real{0.7500}}@{}}
\toprule\noalign{}
\begin{minipage}[b]{\linewidth}\raggedright
Aspect
\end{minipage} & \begin{minipage}[b]{\linewidth}\raggedright
Court Finding
\end{minipage} \\
\midrule\noalign{}
\endhead
\bottomrule\noalign{}
\endlastfoot
Blood Evidence & Not indispensable to prove guilt; could support the
finding but not essential. \\
Juror Instruction & No error in trial judge's directions regarding
evidence. \\
Elements of Murder & Must prove unlawful death with intent to kill or
cause grievous bodily harm. \\
Circumstantial Evidence & Could support murder or unintentional death;
jury could reject any claims of accidental death. \\
\end{longtable}

\subsubsection{Key Legal Principles}\label{key-legal-principles}

\begin{itemize}
\tightlist
\item
  \textbf{Testimony}:

  \begin{itemize}
  \tightlist
  \item
    Witness testimony is cogent evidence if it helps establish the
    truth.
  \item
    Hearsay evidence may be rejected if it relies on what others
    perceived.
  \end{itemize}
\item
  \textbf{Competence, Compellability, and Privilege}:

  \begin{itemize}
  \tightlist
  \item
    \textbf{Competent Witness}: Can lawfully be called to give evidence.
  \item
    \textbf{Compellable Witness}: Can be lawfully obliged to provide
    evidence.
  \item
    All competent witnesses are generally compellable, but there are
    exceptions.
  \end{itemize}
\end{itemize}

\begin{longtable}[]{@{}
  >{\raggedright\arraybackslash}p{(\linewidth - 2\tabcolsep) * \real{0.1512}}
  >{\raggedright\arraybackslash}p{(\linewidth - 2\tabcolsep) * \real{0.8488}}@{}}
\toprule\noalign{}
\begin{minipage}[b]{\linewidth}\raggedright
Term
\end{minipage} & \begin{minipage}[b]{\linewidth}\raggedright
Definition
\end{minipage} \\
\midrule\noalign{}
\endhead
\bottomrule\noalign{}
\endlastfoot
Competent & A person may lawfully be called to give evidence. \\
Compellable & A person can lawfully be obliged to give evidence. \\
Privilege & Legal right to refuse to testify or provide evidence. \\
\end{longtable}

\subsubsection{Elements of Murder}\label{elements-of-murder}

\begin{itemize}
\item
  \textbf{Unlawful Death}: The act must result in the death of another
  person.
\item
  \textbf{Intent}: Must demonstrate intent to kill or to cause grievous
  bodily harm.
\item
  The Court of Appeal noted that circumstantial evidence can support a
  finding of guilt, and the absence of a weapon complicates the
  narrative of the killing, indicating intent was necessary for
  conviction.
\end{itemize}

\subsection{  Witness Competence and
Compellability}\label{witness-competence-and-compellability}

\subsubsection{Competence of a Witness}\label{competence-of-a-witness}

A witness must be \textbf{competent} to testify, meaning they can
lawfully provide evidence.

\begin{quote}
``A person is competent if that person MAY lawfully be called to give
evidence: (ACC v Stoddart per French CJ \& Gummow J).''
\end{quote}

\paragraph{Legislative Framework}\label{legislative-framework}

\begin{itemize}
\tightlist
\item
  The \textbf{Evidence Act} (EAQ) has replaced old common law categories
  of incompetence:

  \begin{itemize}
  \tightlist
  \item
    \textbf{Sections 6, 7, and 8} outline these changes.
  \end{itemize}
\item
  \textbf{General Rule}:

  \begin{itemize}
  \tightlist
  \item
    According to EAQ, \(s9(1)\): ``Every person, including a child, is
    presumed to be competent to give evidence in a proceeding.''
  \end{itemize}
\end{itemize}

\paragraph{Exceptions and Conditions for
Competence}\label{exceptions-and-conditions-for-competence}

\begin{itemize}
\tightlist
\item
  \textbf{Sections 9A-D} address when competency can be questioned:

  \begin{itemize}
  \tightlist
  \item
    \(s9A\): A witness is competent if they can provide an intelligible
    account of events.
  \item
    \(s9B\): A witness can give sworn evidence if they understand the
    seriousness of the matter and have an obligation to tell the truth.
  \item
    \(s9C\): Expert evidence may be introduced when evaluating
    competency under sections 9A and 9B or when a child under 12 gives
    evidence.
  \item
    \(s9D\): There is no legal distinction between sworn and unsworn
    evidence.
  \end{itemize}
\end{itemize}

\subsubsection{Affirmations and Oaths}\label{affirmations-and-oaths}

\begin{quote}
``An affirmation is the equivalent to an oath.''
\end{quote}

\begin{itemize}
\tightlist
\item
  \textbf{Oaths Act,} \(s17\):

  \begin{itemize}
  \tightlist
  \item
    If a witness objects to being sworn, they may instead make a solemn
    affirmation, which holds the same legal weight as an oath.
  \end{itemize}
\end{itemize}

\subsubsection{Case Study: R v D {[}2003{]} QCA
151}\label{case-study-r-v-d-2003-qca-151}

\begin{itemize}
\tightlist
\item
  \textbf{Facts}: Involved indecent dealings with a child under 12.
\item
  \textbf{Significance}:

  \begin{itemize}
  \tightlist
  \item
    The trial judge admitted expert evidence regarding the child's
    memory.
  \item
    The appeal highlighted the distinction between a witness's capacity
    to retain memories and the accuracy of those memories.
  \end{itemize}
\end{itemize}

\paragraph{Roles of Judge and Jury}\label{roles-of-judge-and-jury}

\begin{enumerate}
\def\labelenumi{\arabic{enumi}.}
\tightlist
\item
  \textbf{Step 1}:

  \begin{itemize}
  \tightlist
  \item
    The judge determines if a child is competent to give evidence
    (capable of remembering).
  \end{itemize}
\item
  \textbf{Step 2}:

  \begin{itemize}
  \tightlist
  \item
    The jury assesses whether the child's memory is accurate and whether
    the events occurred as stated.
  \end{itemize}
\end{enumerate}

\subsubsection{Compellability of a
Witness}\label{compellability-of-a-witness}

\begin{itemize}
\tightlist
\item
  Generally, a competent witness is also \textbf{compellable} to
  testify.
\end{itemize}

\begin{quote}
``Competence implies compellability: EAQ, s 7.''
\end{quote}

\paragraph{Exceptions}\label{exceptions}

\begin{itemize}
\tightlist
\item
  \textbf{Criminal Accused}:

  \begin{itemize}
  \tightlist
  \item
    As per EAQ, \(s8(1)\), an accused cannot be compelled to testify.
  \item
    The right to silence is preserved unless the accused has unique
    knowledge that they choose to remain silent about.
  \end{itemize}
\item
  \textbf{Spouses}:

  \begin{itemize}
  \tightlist
  \item
    Typically compellable even in criminal matters without the accused's
    consent (EAQ, \(s7(2)\)).
  \end{itemize}
\end{itemize}

\subsubsection{Privilege}\label{privilege}

Even if a witness is competent and compellable, they may still be
\textbf{privileged} against answering certain questions.

\paragraph{Types of Privilege}\label{types-of-privilege}

\begin{enumerate}
\def\labelenumi{\arabic{enumi}.}
\tightlist
\item
  \textbf{Privilege Against Self-Incrimination}:

  \begin{itemize}
  \tightlist
  \item
    A witness cannot be forced to answer questions that may expose them
    to criminal liability.
  \end{itemize}
\item
  \textbf{Legal Professional Privilege}:

  \begin{itemize}
  \tightlist
  \item
    Confidential communications with a lawyer regarding legal advice are
    protected.
  \end{itemize}
\item
  \textbf{Without Prejudice Privilege}:

  \begin{itemize}
  \tightlist
  \item
    Negotiation processes cannot be used as an admission of guilt.
  \end{itemize}
\item
  \textbf{Public Interest Immunity}:

  \begin{itemize}
  \tightlist
  \item
    The Crown may withhold evidence for national interest reasons.
  \end{itemize}
\end{enumerate}

\subsubsection{Legal Professional Privilege
(LPP)}\label{legal-professional-privilege-lpp}

\begin{itemize}
\tightlist
\item
  \textbf{Privilege is held by the client}, not the lawyer. Thus,
  lawyers cannot prevent clients from answering questions.
\item
  \textbf{Waiver of Privilege}:

  \begin{itemize}
  \tightlist
  \item
    The privilege can be waived by the privilege holder.
  \end{itemize}
\end{itemize}

\subsubsection{Ethical Duties of
Lawyers}\label{ethical-duties-of-lawyers}

\begin{itemize}
\tightlist
\item
  Lawyers are bound by fiduciary duties to keep client information
  confidential.
\item
  Confidentiality extends beyond privileged communications.
\end{itemize}

\subsubsection{Key Takeaway on
Privilege}\label{key-takeaway-on-privilege}

\begin{itemize}
\tightlist
\item
  \textbf{Claiming Privilege}:

  \begin{itemize}
  \tightlist
  \item
    A privilege must be specifically claimed by the privilege holder.
  \end{itemize}
\item
  \textbf{Bona Fide Claim}:

  \begin{itemize}
  \tightlist
  \item
    The claim must be made in good faith (Brebner v Perry).
  \end{itemize}
\end{itemize}

\paragraph{Legislative Abrogation of
Privilege}\label{legislative-abrogation-of-privilege}

\begin{itemize}
\tightlist
\item
  Privilege can be revoked by statute (e.g., Pyneboard v TPCo).
\item
  Courts cannot abrogate privilege; only legislation can.
\end{itemize}

\subsubsection{Indirect Incrimination}\label{indirect-incrimination}

\begin{itemize}
\tightlist
\item
  The privilege also protects against indirect incrimination (Sorby v
  Cwth).
\item
  There is no privilege for corporations against self-incrimination (EPA
  v Caltex).
\end{itemize}

\subsubsection{Summary of Key Sections}\label{summary-of-key-sections}

\begin{longtable}[]{@{}
  >{\raggedright\arraybackslash}p{(\linewidth - 2\tabcolsep) * \real{0.1744}}
  >{\raggedright\arraybackslash}p{(\linewidth - 2\tabcolsep) * \real{0.8256}}@{}}
\toprule\noalign{}
\begin{minipage}[b]{\linewidth}\raggedright
Section
\end{minipage} & \begin{minipage}[b]{\linewidth}\raggedright
Description
\end{minipage} \\
\midrule\noalign{}
\endhead
\bottomrule\noalign{}
\endlastfoot
EAQ, s 9(1) & Presumption of competency for all individuals \\
EAQ, s 9A & Competency based on the ability to give an intelligible
account \\
EAQ, s 8(1) & Accused cannot be compelled to testify \\
EAQ, s 7 & Competent witnesses are generally compellable \\
LPP & Confidential communications with lawyers are privileged \\
Self-Incrimination & Witness cannot be forced to produce incriminating
evidence \\
\end{longtable}

\subsection{  Legal Privileges}\label{legal-privileges}

\subsubsection{\texorpdfstring{1. \textbf{Privilege Against
Self-Incrimination}}{1. Privilege Against Self-Incrimination}}\label{privilege-against-self-incrimination}

\begin{quote}
``The related privilege against exposure to penalty can apply to civil
penalties under the Corporations Law.''
\end{quote}

\begin{itemize}
\tightlist
\item
  \textbf{Key Sections:}

  \begin{itemize}
  \tightlist
  \item
    \textbf{EAQ, s 10:} Preserves the privilege against
    self-incrimination. \textgreater{} ``Nothing in this Act shall
    render any person compellable to answer any question tending to
    criminate the person.''
  \item
    \textbf{EAQ, s 15(1):} In a criminal proceeding, the privilege does
    not apply if the accused elects to testify.
  \end{itemize}
\end{itemize}

\subsubsection{\texorpdfstring{2. \textbf{Legal Professional Privilege
(LPP)}}{2. Legal Professional Privilege (LPP)}}\label{legal-professional-privilege-lpp-1}

\begin{quote}
``A person has the right for their confidential communications with
their lawyer relating to legal advice/preparation for trial to be
privileged.''
\end{quote}

\begin{itemize}
\tightlist
\item
  \textbf{Dominant Purpose Test:}

  \begin{itemize}
  \tightlist
  \item
    Originates from \textbf{Baker v Campbell}, which establishes that
    privilege protects communications between lawyers and clients for
    the dominant purpose of seeking legal advice or preparing for
    litigation.
  \end{itemize}
\end{itemize}

\begin{longtable}[]{@{}
  >{\raggedright\arraybackslash}p{(\linewidth - 2\tabcolsep) * \real{0.2596}}
  >{\raggedright\arraybackslash}p{(\linewidth - 2\tabcolsep) * \real{0.7404}}@{}}
\toprule\noalign{}
\begin{minipage}[b]{\linewidth}\raggedright
\textbf{Aspect}
\end{minipage} & \begin{minipage}[b]{\linewidth}\raggedright
\textbf{Details}
\end{minipage} \\
\midrule\noalign{}
\endhead
\bottomrule\noalign{}
\endlastfoot
\textbf{Applies to:} & Civil and criminal matters \\
\textbf{Test Established by:} & Baker v Campbell (Dominant Purpose
Test) \\
\textbf{Function:} & Ensures uninhibited communication between lawyer
and client \\
\textbf{Scope:} & Applies to judicial proceedings and executive
functions \\
\end{longtable}

\subsubsection{\texorpdfstring{3. \textbf{Abrogation of
LPP}}{3. Abrogation of LPP}}\label{abrogation-of-lpp}

\begin{itemize}
\tightlist
\item
  \textbf{Daniels Corp v ACCC:} LPP can be abrogated by statute using
  express words or necessary implication.
\item
  \textbf{Comm AFP v Propend:} Copies of documents can be privileged
  even if the original is not.
\end{itemize}

\subsubsection{\texorpdfstring{4. \textbf{Waiver of Legal Professional
Privilege}}{4. Waiver of Legal Professional Privilege}}\label{waiver-of-legal-professional-privilege}

\begin{quote}
``Only a privilege holder (i.e., client) can waive privilege.''
\end{quote}

\begin{itemize}
\tightlist
\item
  \textbf{Types of Waiver:}

  \begin{itemize}
  \tightlist
  \item
    Express or implied
  \item
    Deliberate or inadvertent
  \end{itemize}
\end{itemize}

\begin{longtable}[]{@{}
  >{\raggedright\arraybackslash}p{(\linewidth - 2\tabcolsep) * \real{0.1880}}
  >{\raggedright\arraybackslash}p{(\linewidth - 2\tabcolsep) * \real{0.8120}}@{}}
\toprule\noalign{}
\begin{minipage}[b]{\linewidth}\raggedright
\textbf{Case}
\end{minipage} & \begin{minipage}[b]{\linewidth}\raggedright
\textbf{Details}
\end{minipage} \\
\midrule\noalign{}
\endhead
\bottomrule\noalign{}
\endlastfoot
\textbf{Ng v Goldberg} & Waiver is found when it would be unfair to
assert continued existence of privilege. \\
\textbf{Mann v Carnell} & Introduced a correctness test asking if there
is inconsistency between client conduct and privilege. \\
\end{longtable}

\subsubsection{\texorpdfstring{5. \textbf{Without Prejudice
Privilege}}{5. Without Prejudice Privilege}}\label{without-prejudice-privilege}

\begin{quote}
``People cannot use their process of negotiating as some form of implied
admission.''
\end{quote}

\begin{itemize}
\tightlist
\item
  \textbf{Key Points:}

  \begin{itemize}
  \tightlist
  \item
    Cannot use genuine attempts to settle litigation as evidence of
    guilt or liability.
  \item
    \textbf{Field v Commissioner for Railways (NSW):} Words ``without
    prejudice'' often added to letters indicate negotiations.
  \item
    Privilege applies only if the communication is an actual attempt to
    settle.
  \end{itemize}
\end{itemize}

\begin{longtable}[]{@{}
  >{\raggedright\arraybackslash}p{(\linewidth - 2\tabcolsep) * \real{0.2100}}
  >{\raggedright\arraybackslash}p{(\linewidth - 2\tabcolsep) * \real{0.7900}}@{}}
\toprule\noalign{}
\begin{minipage}[b]{\linewidth}\raggedright
\textbf{Case}
\end{minipage} & \begin{minipage}[b]{\linewidth}\raggedright
\textbf{Details}
\end{minipage} \\
\midrule\noalign{}
\endhead
\bottomrule\noalign{}
\endlastfoot
\textbf{Rodgers v Rodgers} & Communication can be privileged even
without ``without prejudice'' if it's a genuine attempt to settle. \\
\end{longtable}

\subsubsection{\texorpdfstring{6. \textbf{Public Interest
Immunity}}{6. Public Interest Immunity}}\label{public-interest-immunity}

\begin{quote}
``The Crown claims they should not have to hand over evidence due to
State or National interests.''
\end{quote}

\begin{itemize}
\tightlist
\item
  \textbf{General Rule:}

  \begin{itemize}
  \tightlist
  \item
    Courts will not order production of documents if it would injure
    public interest.
  \end{itemize}
\item
  \textbf{Balance of Public Interest:}

  \begin{itemize}
  \tightlist
  \item
    Courts must balance between the public interest in withholding
    documents and the administration of justice.
  \end{itemize}
\end{itemize}

\begin{longtable}[]{@{}
  >{\raggedright\arraybackslash}p{(\linewidth - 2\tabcolsep) * \real{0.2105}}
  >{\raggedright\arraybackslash}p{(\linewidth - 2\tabcolsep) * \real{0.7895}}@{}}
\toprule\noalign{}
\begin{minipage}[b]{\linewidth}\raggedright
\textbf{Case}
\end{minipage} & \begin{minipage}[b]{\linewidth}\raggedright
\textbf{Details}
\end{minipage} \\
\midrule\noalign{}
\endhead
\bottomrule\noalign{}
\endlastfoot
\textbf{Sankey v Whitlam} & Established the principle regarding public
interest and document production. \\
\textbf{Conway v Rimmer} & Discussed the court's duty to weigh public
interest aspects and the potential injury to the state. \\
\end{longtable}

\subsubsection{\texorpdfstring{7. \textbf{Exceptions to
Privilege}}{7. Exceptions to Privilege}}\label{exceptions-to-privilege}

\begin{quote}
``If an exception applies, it means that privilege never existed in the
first place.''
\end{quote}

\begin{itemize}
\tightlist
\item
  \textbf{Communications in Furtherance of Unlawful Purpose:}

  \begin{itemize}
  \tightlist
  \item
    Example: \textbf{R v Bell; ex parte Leeso}
  \end{itemize}
\end{itemize}

\begin{longtable}[]{@{}
  >{\raggedright\arraybackslash}p{(\linewidth - 2\tabcolsep) * \real{0.2105}}
  >{\raggedright\arraybackslash}p{(\linewidth - 2\tabcolsep) * \real{0.7895}}@{}}
\toprule\noalign{}
\begin{minipage}[b]{\linewidth}\raggedright
\textbf{Case}
\end{minipage} & \begin{minipage}[b]{\linewidth}\raggedright
\textbf{Details}
\end{minipage} \\
\midrule\noalign{}
\endhead
\bottomrule\noalign{}
\endlastfoot
\textbf{A-G (NT) v Kearney} & Discusses abuse of statutory power. \\
\textbf{Carter v Managing Partner NHDL} & Not applicable if it would
establish innocence. \\
\end{longtable}

\subsubsection{\texorpdfstring{8. \textbf{Vulnerable
Witnesses}}{8. Vulnerable Witnesses}}\label{vulnerable-witnesses}

\begin{itemize}
\tightlist
\item
  \textbf{Competence, Credit, and Disability:}

  \begin{itemize}
  \tightlist
  \item
    \textbf{R v D {[}2003{]} QCA 151:} Discusses reliability of memories
    in child sexual abuse cases.
  \end{itemize}
\end{itemize}

\begin{longtable}[]{@{}
  >{\raggedright\arraybackslash}p{(\linewidth - 2\tabcolsep) * \real{0.3108}}
  >{\raggedright\arraybackslash}p{(\linewidth - 2\tabcolsep) * \real{0.6892}}@{}}
\toprule\noalign{}
\begin{minipage}[b]{\linewidth}\raggedright
\textbf{Key Legislation}
\end{minipage} & \begin{minipage}[b]{\linewidth}\raggedright
\textbf{Details}
\end{minipage} \\
\midrule\noalign{}
\endhead
\bottomrule\noalign{}
\endlastfoot
\textbf{Evidence (Protection of Children) Amendment Act 2003 (Qld)} &
Provides extra protections for children under 16 who are witnesses. \\
\end{longtable}

\begin{itemize}
\tightlist
\item
  \textbf{Competency and Credibility:}

  \begin{itemize}
  \tightlist
  \item
    \textbf{Coombe v Bessell {[}1994{]} 4 Tas R 149:} Discusses the
    impact of speech impediments on witness credibility.
  \end{itemize}
\end{itemize}

\subsubsection{\texorpdfstring{9. \textbf{Witnesses' Physical or Mental
Unreliability}}{9. Witnesses' Physical or Mental Unreliability}}\label{witnesses-physical-or-mental-unreliability}

\begin{itemize}
\tightlist
\item
  \textbf{Toohey v Metropolitan Police Commissioner:}

  \begin{itemize}
  \tightlist
  \item
    Facts regarding the reliability of witness testimony in assault
    cases.
  \end{itemize}
\end{itemize}

\subsection{  Witness Credibility and Medical
Evidence}\label{witness-credibility-and-medical-evidence}

\subsubsection{The Case of Madden}\label{the-case-of-madden}

\begin{itemize}
\tightlist
\item
  \textbf{Incident Description}: Madden alleged he was assaulted in a
  dark alley, where he was hit in the stomach and searched for money.
\item
  \textbf{Defense Evidence}: A police surgeon testified that no signs of
  injury were found on Madden, who appeared intoxicated and emotionally
  unstable during the examination.
\end{itemize}

\subsubsection{Key Issue}\label{key-issue}

\begin{quote}
\textbf{Issue}: Whether it was permissible to impeach Madden's
credibility as a witness through medical evidence regarding his
emotional state.
\end{quote}

\subsubsection{Legal Steps (per Lord
Pearce)}\label{legal-steps-per-lord-pearce}

\begin{enumerate}
\def\labelenumi{\arabic{enumi}.}
\tightlist
\item
  \textbf{Competence of the Witness}: A voir dire may be called to
  determine if the witness is competent.

  \begin{itemize}
  \tightlist
  \item
    A witness might be cross-examined before being sworn in if evidence
    suggests they have a diseased mind that could affect their
    testimony.
  \end{itemize}
\item
  \textbf{Medical Evidence}: The court can allow for medical evidence to
  reveal if a witness is incapable of providing a reliable account due
  to physical or mental conditions.

  \begin{itemize}
  \tightlist
  \item
    Example: If a witness claims to have seen something from 50 yards
    away, an oculist could testify that the witness could only see up to
    20 yards.
  \end{itemize}
\end{enumerate}

\subsubsection{Expert Evidence on Witness's
Condition}\label{expert-evidence-on-witnesss-condition}

\begin{itemize}
\tightlist
\item
  \textbf{Farrell v The Queen}: Expert evidence on the psychological and
  physical conditions of a witness can be admissible if:

  \begin{enumerate}
  \def\labelenumi{\arabic{enumi}.}
  \tightlist
  \item
    Provided by an expert in a relevant field.
  \item
    Goes beyond ordinary experience of the trier of fact.
  \item
    The jury is warned that the expert cannot determine the witness's
    credibility.
  \end{enumerate}
\end{itemize}

\subsection{  Vulnerable Witnesses}\label{vulnerable-witnesses-1}

\subsubsection{Definition and
Categories}\label{definition-and-categories}

\begin{itemize}
\tightlist
\item
  \textbf{Vulnerable Witnesses}: An umbrella term for witnesses in civil
  and criminal proceedings who may face disadvantages, including:

  \begin{itemize}
  \tightlist
  \item
    Children and young people.
  \item
    Victims of violence, especially sexual assault.
  \item
    Individuals with impaired intellectual or mental abilities.
  \item
    Those suffering severe emotional trauma or intimidation.
  \end{itemize}
\end{itemize}

\subsubsection{Legislative Provisions}\label{legislative-provisions}

\begin{itemize}
\tightlist
\item
  \textbf{Special Measures}: Various legislative options exist to assist
  vulnerable witnesses:

  \begin{itemize}
  \tightlist
  \item
    \textbf{Criminal Law (Sexual Offences) Act 1978 (Qld)} provides
    specific provisions for victims of sexual assaults.
  \item
    \textbf{Evidence Act (Qld)} classifies vulnerable witnesses into:

    \begin{enumerate}
    \def\labelenumi{\arabic{enumi}.}
    \tightlist
    \item
      Special witnesses (s 21A)
    \item
      Protected witnesses (s 21L)
    \item
      Affected children (s 9E and Part 2, Division 4A)
    \end{enumerate}
  \end{itemize}
\end{itemize}

\subsubsection{Special Witness Definition (Section
21A)}\label{special-witness-definition-section-21a}

\begin{quote}
\textbf{Special Witness}: Defined as: - A child under 16 years. - A
person likely to be disadvantaged due to mental, intellectual, or
physical impairments. - A person likely to suffer severe emotional
trauma or intimidation.
\end{quote}

\subsubsection{Court Procedures for Special
Witnesses}\label{court-procedures-for-special-witnesses}

\begin{itemize}
\tightlist
\item
  Special measures may include:

  \begin{itemize}
  \tightlist
  \item
    Exclusion of individuals from the courtroom.
  \item
    Examination in a separate room.
  \item
    Allowing a support person to be present.
  \item
    Use of videotaped evidence or closed-circuit TV.
  \end{itemize}
\end{itemize}

\subsubsection{Additional Provisions (Section
21A(2))}\label{additional-provisions-section-21a2}

\begin{itemize}
\tightlist
\item
  The court can modify how evidence is received from special witnesses,
  including:

  \begin{itemize}
  \tightlist
  \item
    Exclusion of the accused from the courtroom.
  \item
    Use of screens to obscure the witness's view of the accused.
  \item
    General provisions for simplifying questioning and allowing for
    breaks.
  \end{itemize}
\end{itemize}

\subsubsection{Important Note on Court
Authority}\label{important-note-on-court-authority}

\begin{itemize}
\tightlist
\item
  The court has the power to adjust the process for special witnesses,
  even if neither party requests it, to minimize trauma and ensure fair
  testimony.
\end{itemize}

\subsection{  Special Witnesses and
Evidence}\label{special-witnesses-and-evidence}

\subsubsection{Admissibility of Video-Taped
Evidence}\label{admissibility-of-video-taped-evidence}

\begin{itemize}
\tightlist
\item
  \textbf{Section 21A(6) EA (Qld)} states that a video-taped recording
  of a special witness's testimony is:

  \begin{itemize}
  \tightlist
  \item
    As admissible as if the evidence were given orally in court.
  \item
    Admissible in:

    \begin{itemize}
    \tightlist
    \item
      Any rehearing or re-trial of the proceeding.
    \item
      In criminal proceedings for another charge arising from the same
      circumstances or for civil proceedings related to the offence.
    \end{itemize}
  \end{itemize}
\end{itemize}

\subsubsection{Jury Directions}\label{jury-directions}

\begin{itemize}
\tightlist
\item
  \textbf{Section 21A(8) EA (Qld)} requires that if evidence is given in
  a proceeding on indictment:

  \begin{itemize}
  \tightlist
  \item
    The judge must instruct the jury that:

    \begin{itemize}
    \tightlist
    \item
      They should not draw any inference regarding the defendant's guilt
      from the order or direction.
    \item
      The probative value of the evidence is unaffected by the order or
      direction.
    \item
      The evidence should not be given greater or lesser weight because
      of the order or direction.
    \end{itemize}
  \end{itemize}
\end{itemize}

\subsubsection{Case Study: R v Michael {[}2008{]} QCA
33}\label{case-study-r-v-michael-2008-qca-33}

\begin{itemize}
\tightlist
\item
  In this rape case:

  \begin{itemize}
  \tightlist
  \item
    The prosecution sought to classify the victim as a special witness
    due to her aboriginality.
  \item
    The court allowed the victim to have a support person during her
    testimony.
  \item
    The accused was convicted but appealed on the basis that the
    mandatory jury direction under \textbf{s 21A(8)} was not given.
  \end{itemize}
\end{itemize}

\paragraph{Key Findings}\label{key-findings}

\begin{itemize}
\tightlist
\item
  A failure to comply with \textbf{s 21A(8)} is an error of law
  rendering the trial irregular.
\end{itemize}

\subsubsection{Protective Measures for Victim
Witnesses}\label{protective-measures-for-victim-witnesses}

\begin{itemize}
\tightlist
\item
  Queensland has implemented protective measures to assist witnesses and
  victims in sexual offence cases through the \textbf{Criminal Law
  (Sexual Offences) Act 1978 (Qld)} (CL(SO)A):

  \begin{itemize}
  \tightlist
  \item
    \textbf{Purpose:}

    \begin{itemize}
    \tightlist
    \item
      To create a more accessible courtroom for complainants.
    \item
      To limit the accused's ability to discredit the complainant's past
      sexual behaviour.
    \item
      To restrict types of questions during cross-examination.
    \end{itemize}
  \end{itemize}
\end{itemize}

\begin{longtable}[]{@{}
  >{\raggedright\arraybackslash}p{(\linewidth - 2\tabcolsep) * \real{0.1836}}
  >{\raggedright\arraybackslash}p{(\linewidth - 2\tabcolsep) * \real{0.8164}}@{}}
\toprule\noalign{}
\begin{minipage}[b]{\linewidth}\raggedright
\textbf{Provision}
\end{minipage} & \begin{minipage}[b]{\linewidth}\raggedright
\textbf{Description}
\end{minipage} \\
\midrule\noalign{}
\endhead
\bottomrule\noalign{}
\endlastfoot
\textbf{Section 4, Rule 1} & Prevents admission of evidence relating to
the complainant's general reputation for chastity, with court discretion
to allow if relevant to guilt or innocence. \\
\textbf{Section 4, Rule 2} & Evidence of complainant's sexual activities
may be permissible only with court leave, showing substantial relevance
to the facts in issue or as a proper matter for cross-examination as to
credit. \\
\end{longtable}

\subsubsection{Exclusion of the Public}\label{exclusion-of-the-public}

\begin{itemize}
\tightlist
\item
  Under \textbf{CL(SO)A}, the court has the power to provide an
  environment conducive to giving evidence, including:

  \begin{itemize}
  \tightlist
  \item
    Excluding the general public but allowing the complainant's support
    person or parent/guardian if beneficial.
  \end{itemize}
\end{itemize}

\subsubsection{Relationship Between
Sections}\label{relationship-between-sections}

\begin{itemize}
\tightlist
\item
  \textbf{R v Samson {[}2011{]} QCA 112} outlines interactions between
  \textbf{s 5 CL(SO)A} and \textbf{s 21A EA (Qld)}, stating:

  \begin{itemize}
  \tightlist
  \item
    Section 5 applies to complainants giving evidence for specified
    sexual offences, irrespective of whether they are special witnesses
    under \textbf{s 21A}.
  \end{itemize}
\end{itemize}

\subsubsection{Protected Witnesses}\label{protected-witnesses}

\begin{itemize}
\tightlist
\item
  \textbf{Sections 21L-21R EAQ} establish a regime for dealing with
  cross-examination of protected witnesses:

  \begin{itemize}
  \tightlist
  \item
    Applies only to certain criminal matters.
  \end{itemize}
\end{itemize}

\paragraph{Definition of Protected
Witnesses}\label{definition-of-protected-witnesses}

\begin{itemize}
\tightlist
\item
  Includes:

  \begin{itemize}
  \tightlist
  \item
    Witnesses under 16 years.
  \item
    Intellectually impaired witnesses.
  \item
    Alleged victims of prescribed special offences.
  \end{itemize}
\end{itemize}

\subsubsection{Cross-Examination
Regulations}\label{cross-examination-regulations}

\begin{itemize}
\tightlist
\item
  \textbf{Section 21N, EA (Qld)} prohibits the accused from personally
  cross-examining a protected witness.
\item
  \textbf{Sections 21O-21R} establish mechanisms for cross-examination
  by legally aided lawyers if the accused is unrepresented.
\end{itemize}

\subsubsection{Treatment of Children as
Witnesses}\label{treatment-of-children-as-witnesses}

\begin{itemize}
\tightlist
\item
  \textbf{Section 9E EAQ} provides guidelines for dealing with children
  as witnesses, emphasizing:

  \begin{itemize}
  \tightlist
  \item
    Treatment with dignity, respect, and compassion.
  \end{itemize}
\end{itemize}

\subsection{  Child Witnesses and Affected
Children}\label{child-witnesses-and-affected-children}

\subsubsection{Guidelines for Child
Witnesses}\label{guidelines-for-child-witnesses}

\begin{itemize}
\tightlist
\item
  \textbf{Distress Minimization:} Measures should be taken to limit, to
  the greatest practical extent, the distress or trauma suffered by the
  child when giving evidence.
\item
  \textbf{Intimidation Prevention:} The child should not be intimidated
  during cross-examination.
\item
  \textbf{Quick Resolution:} The proceedings should be resolved as
  quickly as possible.
\end{itemize}

\begin{quote}
\textbf{Definition:} According to s 9E(3) EAQ, a child is defined as an
individual under the age of 16.
\end{quote}

\subsubsection{Affected Children}\label{affected-children}

\paragraph{Definition of Affected
Children}\label{definition-of-affected-children}

\begin{itemize}
\tightlist
\item
  An \textbf{affected child} is defined as a child under 16 years of age
  who is a witness to a sexual or violent offence.
\item
  Relevant provisions apply regardless of the applicability of special,
  protected, and affected children provisions.
\end{itemize}

\paragraph{Legal References}\label{legal-references}

\begin{itemize}
\tightlist
\item
  \textbf{Part 2, Division 4A EAQ}: Addresses the rights and protections
  for affected children.
\item
  \textbf{Section 21AA-21AX}: Pertains to the evidence of affected
  children.
\item
  \textbf{Section 21AC EA (Qld)}: Defines an affected child as a witness
  in a relevant proceeding but excludes defendants.
\end{itemize}

\subsubsection{Relevant Proceedings}\label{relevant-proceedings}

\paragraph{Definition and Criteria}\label{definition-and-criteria}

\begin{itemize}
\tightlist
\item
  A child is considered in relevant proceedings under these
  circumstances:

  \begin{enumerate}
  \def\labelenumi{\arabic{enumi}.}
  \tightlist
  \item
    The defendant in the proceeding is arrested.
  \item
    A complaint is made under the Justices Act 1886, s 428.
  \item
    A notice to appear is served on the defendant under the Police
    Powers and Responsibilities Act 2000, s 382.
  \item
    An individual who is 16 or 17 years at the time of the events and is
    classified as a special witness.
  \end{enumerate}
\end{itemize}

\begin{quote}
\textbf{Important Note:} The definition of a child can be complex,
particularly in relation to the age at which the complaint, charge, or
notice was made.
\end{quote}

\subsubsection{Special Arrangements for Committal and Trial
Evidence}\label{special-arrangements-for-committal-and-trial-evidence}

\paragraph{Introduction of Special
Protections}\label{introduction-of-special-protections}

\begin{itemize}
\tightlist
\item
  \textbf{Divisions 4A and 4B} were added to the EAQ in 2003 to provide
  additional protections for child witnesses.
\item
  \textbf{Division 4A} specifically does not apply to children who are
  defendants.
\end{itemize}

\paragraph{Key Provisions under Division
4A}\label{key-provisions-under-division-4a}

\begin{longtable}[]{@{}
  >{\raggedright\arraybackslash}p{(\linewidth - 4\tabcolsep) * \real{0.2115}}
  >{\raggedright\arraybackslash}p{(\linewidth - 4\tabcolsep) * \real{0.4231}}
  >{\raggedright\arraybackslash}p{(\linewidth - 4\tabcolsep) * \real{0.3654}}@{}}
\toprule\noalign{}
\begin{minipage}[b]{\linewidth}\raggedright
Provision
\end{minipage} & \begin{minipage}[b]{\linewidth}\raggedright
Criminal Proceedings
\end{minipage} & \begin{minipage}[b]{\linewidth}\raggedright
Civil Proceedings
\end{minipage} \\
\midrule\noalign{}
\endhead
\bottomrule\noalign{}
\endlastfoot
Evidence Recording & Pre-recorded evidence required; if not possible,
use audio-visual links or screens. & Evidence given via audio-visual
link or screen. \\
Cross-Examination & Generally, the child is not called for
cross-examination unless the magistrate requires it. & Not specified in
this context. \\
Courtroom Arrangements & Public exclusion and support persons allowed. &
Same as criminal proceedings. \\
\end{longtable}

\subsubsection{Jury Directions}\label{jury-directions-1}

\paragraph{Requirements under s 21AW EA
(Qld)}\label{requirements-under-s-21aw-ea-qld}

\begin{itemize}
\tightlist
\item
  Jury must receive directions that comply with s 21AW when evidence is
  recorded under specific subdivisions.
\item
  Instructions include:

  \begin{enumerate}
  \def\labelenumi{\arabic{enumi}.}
  \tightlist
  \item
    The measure is a routine practice of the court, and no inferences
    should be drawn about the defendant's guilt from the measure.
  \item
    The probative value of the evidence is not affected by the measure.
  \item
    The weight of the evidence remains unchanged regardless of how it
    was given.
  \end{enumerate}
\end{itemize}

\subsubsection{Important Definitions for Division
4A}\label{important-definitions-for-division-4a}

\begin{longtable}[]{@{}
  >{\raggedright\arraybackslash}p{(\linewidth - 2\tabcolsep) * \real{0.3333}}
  >{\raggedright\arraybackslash}p{(\linewidth - 2\tabcolsep) * \real{0.6667}}@{}}
\toprule\noalign{}
\begin{minipage}[b]{\linewidth}\raggedright
Term
\end{minipage} & \begin{minipage}[b]{\linewidth}\raggedright
Definition
\end{minipage} \\
\midrule\noalign{}
\endhead
\bottomrule\noalign{}
\endlastfoot
Relevant Offence & An offence of a sexual nature or involving violence
with a prescribed relationship to the child. \\
Relevant Proceeding & A criminal proceeding for a relevant offence or a
civil proceeding arising from the commission of a relevant offence. \\
Prescribed Relationship & Includes relationships such as parent,
sibling, grandparent, or others who had regular authority over the
child. \\
\end{longtable}

\subsubsection{Additional
Considerations}\label{additional-considerations}

\begin{itemize}
\tightlist
\item
  Affected children are entitled to a support person during proceedings
  as per s 21AV.
\item
  Special provisions for the management and editing of video recordings
  of evidence are outlined in Division 4B, which applies to all special
  witnesses, not just children.
\end{itemize}

\subsection{  Witness Examination and
Evidence}\label{witness-examination-and-evidence}

\subsubsection{Importance of Witness
Testimony}\label{importance-of-witness-testimony}

\begin{quote}
``Witness testimony is evidence capable of proving facts.''
\end{quote}

\begin{itemize}
\tightlist
\item
  There is a \textbf{general preference} for oral testimony given from
  memory in court.
\item
  \textbf{Jogging of memory} may be allowed under certain circumstances.
\end{itemize}

\subsubsection{Flow of Testimony}\label{flow-of-testimony}

\begin{itemize}
\tightlist
\item
  The \textbf{general flow} of testimony is controlled by a
  \textbf{narrow question-answer format}.
\item
  Counsel has control over:

  \begin{itemize}
  \tightlist
  \item
    Which witnesses are called
  \item
    The order in which they are called
  \end{itemize}
\end{itemize}

\subsubsection{Procedural Issues in
Testimony}\label{procedural-issues-in-testimony}

\paragraph{How Pre-Recorded Video Evidence is
Given}\label{how-pre-recorded-video-evidence-is-given}

\begin{longtable}[]{@{}
  >{\raggedright\arraybackslash}p{(\linewidth - 2\tabcolsep) * \real{0.4348}}
  >{\raggedright\arraybackslash}p{(\linewidth - 2\tabcolsep) * \real{0.5652}}@{}}
\toprule\noalign{}
\begin{minipage}[b]{\linewidth}\raggedright
Scenario
\end{minipage} & \begin{minipage}[b]{\linewidth}\raggedright
Description
\end{minipage} \\
\midrule\noalign{}
\endhead
\bottomrule\noalign{}
\endlastfoot
\textbf{Scenario 1} & Child gives evidence via CCTV at a pre-trial
hearing with no jury present. The witness is cross-examined, and the
evidence is recorded on videotape. The jury later views the
videotape. \\
\textbf{Scenario 2} & Child's evidence-in-chief is recorded during an
interview with police. The child attends a pre-trial hearing where the
videotape is played. Supplementary questions may be asked, followed by
cross-examination. Both videotapes are shown to the jury later. \\
\end{longtable}

\subsubsection{Examination-in-Chief}\label{examination-in-chief}

\begin{itemize}
\tightlist
\item
  The purpose of examination-in-chief is to obtain testimony supporting
  your party's version of the facts.
\item
  Start by identifying the witness and use \textbf{non-leading
  questions} to prompt testimony.
\end{itemize}

\subsubsection{Key Rules for
Examination-in-Chief}\label{key-rules-for-examination-in-chief}

\begin{itemize}
\tightlist
\item
  \textbf{No leading questions}: These suggest a desired answer or
  assume facts not yet in evidence.
\item
  Refreshing memory is allowed, but a witness may not corroborate
  themselves.
\end{itemize}

\subsubsection{Leading Questions}\label{leading-questions}

\begin{itemize}
\tightlist
\item
  A leading question suggests a desired answer or assumes the existence
  of a disputed fact.
\item
  Leading questions are \textbf{objectionable} due to the risk of
  collusion or suggesting evidence not yet presented.
\end{itemize}

\subsubsection{Exceptions to the Rule Against Leading
Questions}\label{exceptions-to-the-rule-against-leading-questions}

\begin{longtable}[]{@{}
  >{\raggedright\arraybackslash}p{(\linewidth - 2\tabcolsep) * \real{0.4583}}
  >{\raggedright\arraybackslash}p{(\linewidth - 2\tabcolsep) * \real{0.5417}}@{}}
\toprule\noalign{}
\begin{minipage}[b]{\linewidth}\raggedright
Situation
\end{minipage} & \begin{minipage}[b]{\linewidth}\raggedright
Description
\end{minipage} \\
\midrule\noalign{}
\endhead
\bottomrule\noalign{}
\endlastfoot
\textbf{Preliminary/Formal Matters} & Questions about undisputed facts
such as name or occupation are permissible. \\
\textbf{Matters Not in Dispute} & Questions about facts everyone agrees
on can be asked as leading questions. \\
\end{longtable}

\subsubsection{Important Cases}\label{important-cases-1}

\begin{itemize}
\tightlist
\item
  \textbf{Briscoe v Briscoe}: Counsel controls the flow of testimony.
\item
  \textbf{R v Apositilidies}: The prosecutor's discretion on calling a
  witness should not be interfered with.
\item
  \textbf{Rule in Jones v Dunkel}: Unexplained failure to call a witness
  may lead to unfavorable inferences in civil matters.
\item
  \textbf{R v Lister}: Accused may give evidence at any point, though it
  is desirable they go first.
\item
  \textbf{R v Shaw}: Leading questions may be acceptable if they
  summarize the essence of a witness's evidence.
\end{itemize}

\subsubsection{Conclusion on Leading
Questions}\label{conclusion-on-leading-questions}

\begin{itemize}
\tightlist
\item
  Answers to leading questions are not inadmissible, but the method of
  obtaining them can affect their significance.
\item
  Avoid leading questions with yes/no answers, as they can undermine the
  integrity of a witness's testimony.
\end{itemize}

\subsubsection{General Tips for Examining
Witnesses}\label{general-tips-for-examining-witnesses}

\begin{itemize}
\tightlist
\item
  Ensure that the witness's testimony comes from their own recollection,
  not from a script.
\item
  Break up questions that assume facts to maintain clarity and avoid
  leading implications.
\item
  Always consider what is in dispute based on the context of the case
  (pleadings in civil cases, charges in criminal cases).
\end{itemize}

\subsection{  Examining Witnesses}\label{examining-witnesses}

\subsubsection{Hostile/Adverse Witness
Definition}\label{hostileadverse-witness-definition}

\begin{quote}
``A \textbf{hostile witness} is not simply someone who fails to give the
evidence which it was hoped that they would give, but in fact
deliberately withholds from giving that evidence, for whatever reason.''
\end{quote}

\paragraph{Key Statutes}\label{key-statutes}

\begin{itemize}
\tightlist
\item
  \textbf{EA (Qld), s 21}: Provides assistance in discrediting one's own
  witness if deemed hostile.
\item
  \textbf{EA (Qld), ss 17, 19, 101 (1)(a) \& 102}: Outlines definitions
  and rules for declaring a witness as adverse.
\end{itemize}

\subsubsection{Distinction Between Hostile and Unfavorable
Witnesses}\label{distinction-between-hostile-and-unfavorable-witnesses}

\begin{longtable}[]{@{}
  >{\raggedright\arraybackslash}p{(\linewidth - 2\tabcolsep) * \real{0.2268}}
  >{\raggedright\arraybackslash}p{(\linewidth - 2\tabcolsep) * \real{0.7732}}@{}}
\toprule\noalign{}
\begin{minipage}[b]{\linewidth}\raggedright
Type of Witness
\end{minipage} & \begin{minipage}[b]{\linewidth}\raggedright
Definition
\end{minipage} \\
\midrule\noalign{}
\endhead
\bottomrule\noalign{}
\endlastfoot
Hostile Witness & Deliberately withholds evidence. \\
Unfavorable Witness & Fails to meet expectations in their testimony. \\
\end{longtable}

\subsubsection{Judge's Discretion}\label{judges-discretion}

\begin{itemize}
\tightlist
\item
  The decision to declare a witness as \textbf{hostile} is left to the
  discretion of the trial judge.
\item
  The judge considers factors such as:

  \begin{itemize}
  \tightlist
  \item
    Prior inconsistent statements
  \item
    Witness demeanor
  \item
    Conduct in the witness box
  \item
    Responses to non-leading questions
  \end{itemize}
\end{itemize}

\subsubsection{Legal Precedents}\label{legal-precedents}

\begin{enumerate}
\def\labelenumi{\arabic{enumi}.}
\tightlist
\item
  \textbf{R v Mullins {[}2001{]} QCA 440}: Provides guidelines for
  judges when ruling on witness hostility.
\item
  \textbf{Mclellan v Bowyer (1961) 106 CLR 95}:

  \begin{itemize}
  \tightlist
  \item
    Witness hostility is an objective fact.
  \item
    The judge may consider demeanor and prior inconsistent statements.
  \item
    The discretion of the trial judge should not be overturned on appeal
    unless principles were misconceived.
  \end{itemize}
\end{enumerate}

\subsubsection{Cross-Examining Adverse
Witnesses}\label{cross-examining-adverse-witnesses}

\begin{itemize}
\tightlist
\item
  If a witness is declared hostile, the jury is recalled, and legal
  provisions (ss 17-19 EA 1977 (Qld)) govern the cross-examination
  process.
\end{itemize}

\paragraph{Provisions Under EA (Qld)}\label{provisions-under-ea-qld}

\begin{longtable}[]{@{}
  >{\raggedright\arraybackslash}p{(\linewidth - 2\tabcolsep) * \real{0.0782}}
  >{\raggedright\arraybackslash}p{(\linewidth - 2\tabcolsep) * \real{0.9218}}@{}}
\toprule\noalign{}
\begin{minipage}[b]{\linewidth}\raggedright
Section
\end{minipage} & \begin{minipage}[b]{\linewidth}\raggedright
Description
\end{minipage} \\
\midrule\noalign{}
\endhead
\bottomrule\noalign{}
\endlastfoot
\textbf{s 17} & Discredits own witness by presenting contradictory
evidence if deemed adverse. \\
\textbf{s 18} & Allows proof of previous inconsistent statements of the
witness. \\
\textbf{s 19} & Permits cross-examination regarding written statements
without showing them to the witness. \\
\end{longtable}

\subsubsection{Refreshing Memory}\label{refreshing-memory}

\begin{itemize}
\tightlist
\item
  Witnesses may refer to documents to refresh their memory under certain
  conditions:

  \begin{enumerate}
  \def\labelenumi{\arabic{enumi}.}
  \tightlist
  \item
    Need to refresh memory.
  \item
    Document must be created or verified by the witness when facts were
    fresh.
  \item
    The witness must confirm the document as an accurate record.
  \end{enumerate}
\end{itemize}

\paragraph{Legal Cases on Memory
Refreshing}\label{legal-cases-on-memory-refreshing}

\begin{itemize}
\tightlist
\item
  \textbf{R v Beelen (1972)}: Document must be contemporaneous with
  described events.
\item
  \textbf{R v Vella (2006)}: Cross-examination can occur on parts of the
  document used for refreshing memory without needing the entire
  document in evidence.
\end{itemize}

\subsubsection{General Rules on Witness
Testimony}\label{general-rules-on-witness-testimony}

\begin{itemize}
\tightlist
\item
  Witness evidence should ideally come from their own memory without
  external aids.
\item
  The document used to refresh memory does not become evidence itself
  unless specific conditions are met.
\end{itemize}

\subsubsection{Rights and Privileges of
Witnesses}\label{rights-and-privileges-of-witnesses}

\begin{quote}
``All witnesses, regardless of their role, have the right to aid their
memory through contemporaneous notes or documents.''
\end{quote}

\paragraph{Preconditions for Document
Consultation}\label{preconditions-for-document-consultation}

\begin{itemize}
\tightlist
\item
  Document must be reasonably contemporaneous with the event.
\item
  Witness must attest to its accuracy and relevance.
\end{itemize}

\subsection{  Refreshing Memory in
Court}\label{refreshing-memory-in-court}

\begin{quote}
``Refreshing of memory from past records made by the witness is allowed
under certain circumstances if memory has been exhausted or is lost.''
\end{quote}

\subsubsection{Situations for Refreshing
Memory}\label{situations-for-refreshing-memory}

In \textbf{King v Bryant (No 2) {[}1956{]} QSR 570}, four situations
were identified where a witness may use a document to refresh their
memory:

\begin{enumerate}
\def\labelenumi{\arabic{enumi}.}
\tightlist
\item
  \textbf{Only remembers the document}: If the witness looks at the
  document and only recalls it without any separate memory of the event,
  they must verify and adopt the document for it to be admissible as
  evidence.
\item
  \textbf{Memory rekindled by the document}: If the witness has no
  recollection but their memory is revived by looking at the document,
  their testimony is admissible without needing to produce the document.
\item
  \textbf{Complete memory of events}: If the witness already remembers
  all events in the document, their testimony is admissible without
  needing to produce the document.
\item
  \textbf{Memory of unrelated events}: If the witness remembers events
  not included in the document, their testimony is admissible, and the
  document is irrelevant.
\end{enumerate}

\subsection{�� No Prior Consistent
Statements}\label{no-prior-consistent-statements}

\begin{quote}
``A witness cannot usually be asked about their former statements with a
view to the statement becoming evidence in the case or in order to
demonstrate consistency.''
\end{quote}

\subsubsection{General Ban on Previous
Statements}\label{general-ban-on-previous-statements}

\begin{itemize}
\tightlist
\item
  There is a common law general ban on asking witnesses about previous
  consistent statements they made.
\item
  A witness cannot enhance their credibility by referencing their own
  past statements.
\end{itemize}

\paragraph{Exceptions to the Rule}\label{exceptions-to-the-rule}

\begin{enumerate}
\def\labelenumi{\arabic{enumi}.}
\tightlist
\item
  \textbf{Fresh and Preliminary Complaints}:

  \begin{itemize}
  \tightlist
  \item
    In sexual assault cases, the complainant must have reported the
    incident at the earliest reasonable opportunity.
  \item
    This spontaneous account operates as an exception to the general
    rule regarding witness credibility.
  \end{itemize}
\item
  \textbf{Allegation of Recent Invention}:

  \begin{itemize}
  \tightlist
  \item
    If a witness's testimony is challenged as a recent invention, they
    may be asked if they made a previous consistent statement shortly
    after the event, which was unlikely to be fabricated.
  \end{itemize}
\end{enumerate}

\subsubsection{Legal References}\label{legal-references-1}

\begin{longtable}[]{@{}
  >{\raggedright\arraybackslash}p{(\linewidth - 2\tabcolsep) * \real{0.4348}}
  >{\raggedright\arraybackslash}p{(\linewidth - 2\tabcolsep) * \real{0.5652}}@{}}
\toprule\noalign{}
\begin{minipage}[b]{\linewidth}\raggedright
\textbf{Case}
\end{minipage} & \begin{minipage}[b]{\linewidth}\raggedright
\textbf{Details}
\end{minipage} \\
\midrule\noalign{}
\endhead
\bottomrule\noalign{}
\endlastfoot
R v Freeman (1980) & Fresh complaints must be spontaneous and timely. \\
Nominal Defendant v Clements (1960) & Previous consistent statements can
be used to challenge suggestions of fabrication. \\
\end{longtable}

\subsection{ Corroboration in Sexual Offence
Cases}\label{corroboration-in-sexual-offence-cases}

\begin{quote}
``Evidence of recent complaint in sexual cases is not
self-corroboration.''
\end{quote}

\subsubsection{Importance of Fresh
Complaint}\label{importance-of-fresh-complaint}

\begin{itemize}
\tightlist
\item
  Evidence that a complainant made a fresh complaint at the first
  reasonable opportunity is not to corroborate their testimony, but to
  show consistency and negate the possibility of an afterthought.
\end{itemize}

\subsection{  Cross-Examination}\label{cross-examination}

\begin{itemize}
\tightlist
\item
  Counsel may ask leading questions during cross-examination, and
  witnesses can be questioned about any prior inconsistent statements
  they have made.
\item
  Hearsay evidence remains prohibited in both examination-in-chief and
  cross-examination.
\item
  If the accused decides to testify, they waive their right to silence.
\end{itemize}

\subsubsection{Purpose of
Cross-Examination}\label{purpose-of-cross-examination}

\begin{quote}
``The purpose of cross-examination is to test evidence so that the jury
can decide whether or not to accept it.''
\end{quote}

\begin{longtable}[]{@{}
  >{\raggedright\arraybackslash}p{(\linewidth - 2\tabcolsep) * \real{0.6176}}
  >{\raggedright\arraybackslash}p{(\linewidth - 2\tabcolsep) * \real{0.3824}}@{}}
\toprule\noalign{}
\begin{minipage}[b]{\linewidth}\raggedright
\textbf{Legal Principle}
\end{minipage} & \begin{minipage}[b]{\linewidth}\raggedright
\textbf{Details}
\end{minipage} \\
\midrule\noalign{}
\endhead
\bottomrule\noalign{}
\endlastfoot
R v Martens (2007) & Cross-examination is essential for evaluating
witness credibility. \\
\end{longtable}

\subsection{  Cross-Examination and Its
Rules}\label{cross-examination-and-its-rules}

\subsubsection{Purpose of
Cross-Examination}\label{purpose-of-cross-examination-1}

Cross-examination serves two main purposes:

\begin{enumerate}
\def\labelenumi{\arabic{enumi}.}
\tightlist
\item
  \textbf{Credibility Challenge}: To reduce or weaken the credibility of
  the evidence given by the witness in chief (known as cross-examination
  as to credit).
\item
  \textbf{Establishing Facts}: To establish valuable facts in favor of
  the party cross-examining, which supports that party's theory of the
  case (known as cross-examination as to the issues).
\end{enumerate}

\subsubsection{Scope of Questions}\label{scope-of-questions}

A witness can be asked questions not only about the facts in issue but
also about questions that may be irrelevant but could potentially
impeach their credibility.

\subsubsection{Rule in Browne v Dunn}\label{rule-in-browne-v-dunn}

The rule established in \textbf{Browne v Dunn} (1893) outlines important
guidelines for cross-examination:

\begin{enumerate}
\def\labelenumi{\arabic{enumi}.}
\tightlist
\item
  \textbf{Contradiction Opportunity}: Any matter that is proposed to
  contradict the evidence-in-chief must be put to the witness so they
  can explain the contradiction.
\item
  \textbf{Implied Acceptance}: Failure to do so may imply acceptance of
  the evidence in chief.
\item
  \textbf{Fairness Enforcement}: This rule is strictly enforced as a
  matter of fairness.
\item
  \textbf{Manifest Obviousness}: The rule does not apply if it is
  manifestly obvious due to the pleadings and the trial that the witness
  is being impeached.
\item
  \textbf{Accused Consideration}: The rule is less strict concerning the
  accused, as they have been through the committal and trial processes.
\end{enumerate}

\subsubsection{Application of the Rule}\label{application-of-the-rule}

Examples of the application of the rule include:

\begin{itemize}
\tightlist
\item
  \textbf{Nicholls v The Queen} {[}2005{]} HCA 1
\item
  \textbf{Allied Pastoral Holdings v The Federal Commissioner of
  Taxation} {[}1983{]} 1 NSWLR 1: In this case, the Federal Commissioner
  failed to examine six witnesses regarding a specific provision,
  denying them the chance to explain.
\end{itemize}

\subsubsection{Reasons for the Rule}\label{reasons-for-the-rule}

Hurt J identified the reasons for the rule in \textbf{Browne v Dunn}:

\begin{enumerate}
\def\labelenumi{\arabic{enumi}.}
\tightlist
\item
  Denying a witness the chance to explain contradictions.
\item
  Providing the party calling the witness an opportunity to present
  corroborative evidence.
\item
  Allowing the witness to explain or qualify their evidence in light of
  contradictions.
\end{enumerate}

\subsubsection{Consequences of Breach}\label{consequences-of-breach}

If a counsel:

\begin{itemize}
\tightlist
\item
  Fails to cross-examine,
\item
  Fails to present contradictory parts of their case,
\item
  Calls other evidence that contradicts the witness without prior
  cross-examination,
\end{itemize}

they breach the rule, which applies to both civil and criminal
proceedings. A breach may allow the opponent to recall a witness even
after the case is closed.

\subsection{  Finality Rule}\label{finality-rule}

\subsubsection{Definition}\label{definition}

The \textbf{finality rule} states that leading evidence solely related
to the credit of a witness is not permitted. It is also referred to as
the rule of collateral finality, which applies to the cross-examination
of all witnesses in court.

\subsubsection{Key Points}\label{key-points-1}

\begin{itemize}
\tightlist
\item
  Answers given by witnesses during cross-examination on collateral
  issues are considered final.
\item
  \textbf{Collateral issues} may pertain to the witness's credibility or
  factual liability of a party.
\end{itemize}

\subsubsection{Exceptions to the Finality
Rule}\label{exceptions-to-the-finality-rule}

\begin{enumerate}
\def\labelenumi{\arabic{enumi}.}
\tightlist
\item
  \textbf{Previous Convictions}: A witness's conviction of a crime can
  be introduced to challenge their credibility (e.g., \textbf{R v
  Livingstone} {[}1987{]} 1 Qd R 38).
\item
  \textbf{Bias}: Evidence that a witness is biased toward the party
  calling them (e.g., \textbf{R v Umanski} {[}1961{]} VR 242).
\item
  \textbf{General Reputation}: The witness's reputation for
  untruthfulness.
\item
  \textbf{Inconsistent Statements}: Evidence that a witness previously
  made statements inconsistent with their current testimony.
\item
  \textbf{Disability}: Any disability affecting the reliability of the
  witness.
\end{enumerate}

\subsubsection{Cross-Examination
Limitations}\label{cross-examination-limitations}

While any witness can be cross-examined to discredit them, if the
questions solely affect the credit of the witness and are irrelevant to
the matters at issue, the witness's answers cannot be contradicted by
additional evidence except in exceptional cases.

\subsection{ Re-Examination}\label{re-examination}

Re-examination occurs after cross-examination and serves to repair any
damage caused by the cross-examination. Key points include:

\begin{itemize}
\tightlist
\item
  \textbf{Limited Scope}: Re-examination must be confined to matters
  raised during cross-examination; new matters may only be introduced
  with the judge's permission.
\item
  \textbf{No Leading Questions}: The re-examining party cannot ask
  leading questions.
\end{itemize}

\subsubsection{Example: Dispelling a Wrong
Impression}\label{example-dispelling-a-wrong-impression}

In a homicide case, if an identification witness states that the person
in the dock could be the defendant, the prosecutor can clarify the
witness's actual meaning during re-examination.

\subsection{ Rebuttal}\label{rebuttal}

Rebuttal is an exception allowing a party to present additional evidence
after their case is closed. It is meant to contradict or nullify
evidence presented by an opposing party and is confined solely to the
subject matter of the evidence rebutted.

\subsubsection{Distinction from Reopening a
Case}\label{distinction-from-reopening-a-case}

Rebuttal is different from reopening a case and is subject to practice
principles rather than strict rules of law.

\subsection{  Rebuttal Evidence}\label{rebuttal-evidence}

\begin{quote}
``Evidence in rebuttal will only be allowed if it relates to a matter
that the prosecution was unable to foresee, and this is stringently
applied.''
\end{quote}

\subsubsection{Case Example: Killick v R
(1981)}\label{case-example-killick-v-r-1981}

\begin{itemize}
\tightlist
\item
  \textbf{Facts:} The defence of alibi was raised, but the Crown did not
  call evidence to rebut it initially.
\item
  \textbf{Outcome:} After the defence case closed, evidence was
  introduced to show the accused's alibi was false.
\item
  \textbf{High Court Decision:}

  \begin{itemize}
  \tightlist
  \item
    The Crown should have foreseen reliance on the alibi.
  \item
    Evidence in rebuttal was not admissible as evidence in chief.
  \end{itemize}
\end{itemize}

\subsection{�� Corroboration}\label{corroboration}

\subsubsection{Definition}\label{definition-1}

\begin{quote}
``Corroboration is additional evidence that makes a particular witness's
testimony more probably true.''
\end{quote}

\subsubsection{Key Points}\label{key-points-2}

\begin{itemize}
\tightlist
\item
  \textbf{Requirements for Corroboration:}

  \begin{enumerate}
  \def\labelenumi{\arabic{enumi}.}
  \tightlist
  \item
    \textbf{Probative Independently:} Evidence must prove or disprove
    something in issue without relying on the testimony needing
    corroboration.
  \item
    \textbf{Probative Through Implication:} Evidence must tend to show
    the accused's involvement in the crime.
  \end{enumerate}
\end{itemize}

\subsubsection{Case Example: R v Baskerville
{[}1916{]}}\label{case-example-r-v-baskerville-1916}

\begin{itemize}
\tightlist
\item
  \textbf{Significant Feature:} Corroboration is crucial for acting upon
  a witness's testimony safely.
\item
  \textbf{Role of Corroboration:} Implicates the accused and connects
  them to the crime.
\end{itemize}

\subsubsection{Case Example: Doney v R
(1990)}\label{case-example-doney-v-r-1990}

\begin{itemize}
\tightlist
\item
  \textbf{Facts:} Doney was convicted based on an accomplice's
  testimony.
\item
  \textbf{Outcome:} A handwritten note capable of corroborating the
  accomplice's testimony was found.
\item
  \textbf{Conclusion:} Corroboration can be circumstantial and does not
  need to match other evidence point for point.
\end{itemize}

\subsection{  Historical Position on
Corroboration}\label{historical-position-on-corroboration}

\begin{itemize}
\tightlist
\item
  Certain classes of witnesses (complainants in sexual offences,
  children, accomplices) were historically viewed as unreliable without
  corroboration.
\item
  \textbf{Example:} Kelleher v R (1970) emphasized the dangers of
  convicting based on a single complainant's evidence.
\end{itemize}

\subsubsection{Current Position (QLD)}\label{current-position-qld}

\begin{itemize}
\tightlist
\item
  \textbf{QCC, s 632:}

  \begin{enumerate}
  \def\labelenumi{\arabic{enumi}.}
  \tightlist
  \item
    A person may be convicted on the uncorroborated testimony of one
    witness.
  \item
    Judges are not required to warn juries about the dangers of relying
    on uncorroborated testimony.
  \end{enumerate}
\end{itemize}

\subsubsection{Important Notes}\label{important-notes}

\begin{itemize}
\tightlist
\item
  Judges can give corroboration warnings but cannot imply any class of
  witness is unreliable.
\item
  Corroboration is primarily required in perjury cases now.
\end{itemize}

\subsection{ Judicial Warnings}\label{judicial-warnings}

\subsubsection{Robinson Direction}\label{robinson-direction}

\begin{itemize}
\tightlist
\item
  Given when it's necessary to warn the jury to scrutinize a witness's
  evidence carefully due to special circumstances.
\item
  \textbf{Key Concepts:}

  \begin{itemize}
  \tightlist
  \item
    The factual matrix that creates risk must be clear to the trial
    judge but not necessarily to the jury.
  \end{itemize}
\end{itemize}

\subsubsection{Case Examples}\label{case-examples}

\begin{itemize}
\tightlist
\item
  \textbf{Robinson v R (1999):} Clarified that warnings must avoid
  implying that a witness's evidence is unreliable based solely on their
  class (e.g., children, sexual offence complainants).
\end{itemize}

\subsubsection{Legislative Context}\label{legislative-context}

\begin{itemize}
\tightlist
\item
  \textbf{Criminal Law (Sexual Offences) Act 1978 (Qld), s 4A:}

  \begin{itemize}
  \tightlist
  \item
    Addresses the reliability of complainants' evidence and prohibits
    classist warnings.
  \end{itemize}
\end{itemize}

\subsubsection{Summary of Judicial
Directions}\label{summary-of-judicial-directions}

\begin{longtable}[]{@{}
  >{\raggedright\arraybackslash}p{(\linewidth - 2\tabcolsep) * \real{0.2561}}
  >{\raggedright\arraybackslash}p{(\linewidth - 2\tabcolsep) * \real{0.7439}}@{}}
\toprule\noalign{}
\begin{minipage}[b]{\linewidth}\raggedright
Direction
\end{minipage} & \begin{minipage}[b]{\linewidth}\raggedright
Circumstance of Use
\end{minipage} \\
\midrule\noalign{}
\endhead
\bottomrule\noalign{}
\endlastfoot
Robinson Direction & When special scrutiny of a witness's evidence is
warranted \\
McKinney Direction & Allegations of false confessions in police
custody \\
Domican Direction & Eye-witness identification by a stranger \\
Pollitt v R & Jail-yard confessions potentially serving ulterior
motives \\
\end{longtable}

\subsection{  Conclusion}\label{conclusion}

\begin{itemize}
\tightlist
\item
  The importance of understanding corroboration and rebuttal evidence
  lies in safeguarding fair trial practices and ensuring that justice is
  served without relying solely on potentially unreliable testimony.
\end{itemize}

\subsection{  Forensic Disadvantages and Jury
Warnings}\label{forensic-disadvantages-and-jury-warnings}

\subsubsection{Key Concepts}\label{key-concepts}

\begin{itemize}
\tightlist
\item
  \textbf{Forensic Disadvantage}: Refers to situations where the
  circumstances of a case may create an unfair trial for the accused,
  particularly when evidence cannot be adequately tested due to time
  delays.
\end{itemize}

\begin{quote}
``The question is whether all of the circumstances gave rise to some
forensic disadvantage to the appellant, palpable or obvious to a judge,
which may not have been apparent to the jury\ldots''
\end{quote}

\subsubsection{Case Law and Warnings}\label{case-law-and-warnings}

\paragraph{Longman Warning}\label{longman-warning}

\begin{itemize}
\item
  \textbf{Longman v R (1989) 168 CLR 79}: Establishes the need for
  warnings regarding uncorroborated evidence, particularly in sexual
  offence cases.

  \begin{itemize}
  \item
    \textbf{Facts}: The complainant (C) delayed reporting incidents of
    abuse for over 25 years, leading to a lack of corroboration.
  \item
    \textbf{Judgment}: The trial judge should have directed the jury to
    consider the dangers in relying solely on the complainant's
    uncorroborated testimony.
  \end{itemize}
\end{itemize}

\paragraph{Requirements of a Longman
Direction}\label{requirements-of-a-longman-direction}

\begin{enumerate}
\def\labelenumi{\arabic{enumi}.}
\tightlist
\item
  A warning is necessary to avoid a \textbf{miscarriage of justice}.
\item
  The warning must not suggest that complainants are inherently
  unreliable.
\item
  The court must acknowledge that the passage of time may impair the
  ability to adequately test evidence.
\end{enumerate}

\subsubsection{Important Principles}\label{important-principles}

\begin{longtable}[]{@{}
  >{\raggedright\arraybackslash}p{(\linewidth - 2\tabcolsep) * \real{0.4583}}
  >{\raggedright\arraybackslash}p{(\linewidth - 2\tabcolsep) * \real{0.5417}}@{}}
\toprule\noalign{}
\begin{minipage}[b]{\linewidth}\raggedright
Principle
\end{minipage} & \begin{minipage}[b]{\linewidth}\raggedright
Explanation
\end{minipage} \\
\midrule\noalign{}
\endhead
\bottomrule\noalign{}
\endlastfoot
\textbf{Credibility Comment} & Evidence from alleged victims is subject
to credibility comments like other criminal cases. \\
\textbf{Need for Warnings} & A warning is required when the jury's
ability to assess evidence is compromised due to delays. \\
\end{longtable}

\subsubsection{Judicial Insights}\label{judicial-insights}

\begin{itemize}
\item
  \textbf{R v MBX {[}2013{]} QCA 214}: Highlights the need for a warning
  to ensure jurors understand the risks associated with delayed
  testimonies.
\item
  \textbf{Jago v. District Court NSW}: Reinforces that the fairness of a
  trial can be impaired by significant delays.
\end{itemize}

\subsubsection{Examples of Judicial
Decisions}\label{examples-of-judicial-decisions}

\begin{itemize}
\item
  \textbf{R v Aristidis {[}1999{]} 2 Qd R 629}: The court upheld that
  delay impairs the defendant's right to a fair trial, reinforcing the
  need for warnings.
\item
  \textbf{Davies v DPP {[}1954{]} 1 All ER 507}: Emphasizes the
  necessity of a corroboration warning when dealing with accomplice
  evidence.
\item
  \textbf{Pollitt v R (1992) 174 CLR 558}: A warning is warranted for
  potentially unreliable evidence from prison informers.
\item
  \textbf{DPP v Faure {[}1993{]} 2 VR 497}: Highlights the need for
  warnings when a spouse testifies against the accused, due to potential
  bias.
\item
  \textbf{Bromley v R (1986) 161 CLR 315}: Required a warning regarding
  the reliability of testimony from mentally ill witnesses.
\end{itemize}

\subsubsection{Summary of Warning
Requirements}\label{summary-of-warning-requirements}

\begin{itemize}
\tightlist
\item
  The Longman Direction must inform jurors about:

  \begin{itemize}
  \tightlist
  \item
    The potential unreliability of delayed testimonies.
  \item
    The necessity for thorough scrutiny of such evidence.
  \item
    The dangers of convicting based solely on uncorroborated claims
    unless convinced beyond reasonable doubt.
  \end{itemize}
\end{itemize}

\subsubsection{Conclusion}\label{conclusion-1}

The combination of legal precedents and judicial guidelines emphasizes
the importance of ensuring that jurors are warned about the potential
pitfalls of relying on testimonies that cannot be adequately
substantiated due to delays, ensuring fairness in the judicial process.

\subsection{  Corroboration in
Confessions}\label{corroboration-in-confessions}

\subsubsection{Key Legal Principle}\label{key-legal-principle}

\begin{quote}
``Whenever police evidence of a confessional statement allegedly made by
an accused involuntarily held in police custody without access to a
lawyer or independent person was disputed, and its making not reliably
corroborated, a warning should be given to the jury.''
\end{quote}

\subsection{Case References}\label{case-references}

\begin{longtable}[]{@{}
  >{\raggedright\arraybackslash}p{(\linewidth - 4\tabcolsep) * \real{0.1383}}
  >{\raggedright\arraybackslash}p{(\linewidth - 4\tabcolsep) * \real{0.4734}}
  >{\raggedright\arraybackslash}p{(\linewidth - 4\tabcolsep) * \real{0.3883}}@{}}
\toprule\noalign{}
\begin{minipage}[b]{\linewidth}\raggedright
Case
\end{minipage} & \begin{minipage}[b]{\linewidth}\raggedright
Facts
\end{minipage} & \begin{minipage}[b]{\linewidth}\raggedright
Holding
\end{minipage} \\
\midrule\noalign{}
\endhead
\bottomrule\noalign{}
\endlastfoot
Carr v R (1988) & Confessions made under dispute without corroboration &
Established the need for corroboration warnings in such circumstances \\
Duke v R (1989) & Overruled previous cases on the reliability of police
evidence & Reinforced corroboration requirement \\
Nicholls and Coates v R & Admission made during a police interview
without recording, relied on to prove guilt & Appeal allowed; admission
was not recorded as per legal requirements \\
\end{longtable}

\subsection{Types of Corroborative
Evidence}\label{types-of-corroborative-evidence}

Corroborative evidence can come in various forms:

\begin{longtable}[]{@{}
  >{\raggedright\arraybackslash}p{(\linewidth - 2\tabcolsep) * \real{0.2524}}
  >{\raggedright\arraybackslash}p{(\linewidth - 2\tabcolsep) * \real{0.7476}}@{}}
\toprule\noalign{}
\begin{minipage}[b]{\linewidth}\raggedright
Type
\end{minipage} & \begin{minipage}[b]{\linewidth}\raggedright
Description
\end{minipage} \\
\midrule\noalign{}
\endhead
\bottomrule\noalign{}
\endlastfoot
Direct Evidence & Eyewitness accounts or testimonies from individuals
who witnessed the event \\
Circumstantial Evidence & Indirect evidence that implies a fact but does
not prove it directly \\
Admissions & Statements made by the accused that may imply guilt \\
Physical Injuries & Evidence of injuries that corroborate the account of
events \\
Flight & Evidence showing the accused fled the scene, indicating
consciousness of guilt \\
Similar Fact Evidence & Prior similar actions that indicate a pattern of
behavior \\
\end{longtable}

\subsubsection{Case Example: Eade v R
(1924)}\label{case-example-eade-v-r-1924}

\begin{quote}
``No witnesses to corroborate a child complainant's account of events,
which the accused denied.''
\end{quote}

\begin{itemize}
\tightlist
\item
  \textbf{Held}: Corroboration was found through:

  \begin{itemize}
  \tightlist
  \item
    Evidence of pie purchase by the child
  \item
    Pies found in the accused's house
  \item
    Child's presence at the accused's house
  \item
    Description of the accused's house matching the child's account
  \end{itemize}
\end{itemize}

\subsection{Edwards Direction}\label{edwards-direction}

\subsubsection{Case Context}\label{case-context}

\begin{quote}
``Lie about prison van assault was said to show a consciousness of guilt
which could corroborate the evidence\ldots''
\end{quote}

\begin{longtable}[]{@{}
  >{\raggedright\arraybackslash}p{(\linewidth - 2\tabcolsep) * \real{0.2667}}
  >{\raggedright\arraybackslash}p{(\linewidth - 2\tabcolsep) * \real{0.7333}}@{}}
\toprule\noalign{}
\begin{minipage}[b]{\linewidth}\raggedright
Factor
\end{minipage} & \begin{minipage}[b]{\linewidth}\raggedright
Requirement
\end{minipage} \\
\midrule\noalign{}
\endhead
\bottomrule\noalign{}
\endlastfoot
Prosecution Contention & Must argue that a lie indicates guilt \\
Identification of Lies & Prosecution must identify specific lies and
their implications \\
Jury Understanding & Direction may be required if there is a risk of
jury misunderstanding \\
\end{longtable}

\subsection{Zoneff Direction}\label{zoneff-direction}

\subsubsection{Case Context}\label{case-context-1}

\begin{quote}
``Zoneff defrauded many people through a fake real estate scheme.''
\end{quote}

\begin{itemize}
\tightlist
\item
  Lies regarding fraudulent loans were scrutinized.
\end{itemize}

\begin{longtable}[]{@{}
  >{\raggedright\arraybackslash}p{(\linewidth - 2\tabcolsep) * \real{0.2642}}
  >{\raggedright\arraybackslash}p{(\linewidth - 2\tabcolsep) * \real{0.7358}}@{}}
\toprule\noalign{}
\begin{minipage}[b]{\linewidth}\raggedright
Principle
\end{minipage} & \begin{minipage}[b]{\linewidth}\raggedright
Explanation
\end{minipage} \\
\midrule\noalign{}
\endhead
\bottomrule\noalign{}
\endlastfoot
Consciousness of Guilt & Lies may demonstrate the accused's awareness of
guilt \\
Jury's Role & Jury must determine the significance of the lies relative
to the charges \\
\end{longtable}

\subsection{What is NOT Corroborative
Evidence?}\label{what-is-not-corroborative-evidence}

\begin{itemize}
\item
  \textbf{Evidence of Recent Complaint}: \textgreater{} ``A person
  cannot corroborate his or her own evidence by demonstrating that the
  story was told previously.''
\item
  \textbf{Witnesses who Concoct Stories}:

  \begin{itemize}
  \tightlist
  \item
    Accomplices cannot corroborate each other's stories.
  \end{itemize}
\end{itemize}

\subsection{Identification Evidence}\label{identification-evidence}

\begin{itemize}
\tightlist
\item
  Identification evidence can be highly persuasive but is also notably
  uncertain.
\end{itemize}

\begin{quote}
``It is the single greatest cause of wrongful convictions.''
\end{quote}

\subsubsection{Key Considerations}\label{key-considerations}

\begin{itemize}
\tightlist
\item
  Juries often have confidence in identification evidence, despite its
  unreliability.
\item
  Eyewitness testimony should be evaluated critically to assess its
  validity.
\end{itemize}

\subsection{  Identification Evidence}\label{identification-evidence-1}

\begin{quote}
``Identification evidence can often reflect unconscious projection by
the witness of what he or she wants or expects to see, hear or otherwise
perceive.''
\end{quote}

\subsubsection{Reliability of Identification
Evidence}\label{reliability-of-identification-evidence}

\begin{itemize}
\tightlist
\item
  The reliability of identification evidence is \textbf{notoriously
  uncertain} and is influenced by numerous variables.
\item
  Problems arise from four main sources:

  \begin{enumerate}
  \def\labelenumi{\arabic{enumi}.}
  \tightlist
  \item
    \textbf{Variable quality} of evidence, much of which is fragile.
  \item
    Police methods of identification, while effective for
    investigations, may not yield high probative value in trials.
  \item
    The need to balance the accused's right to a fair trial with the
    state's interest in efficient crime investigation.
  \item
    Challenges in accommodating certain types of identification
    testimony within established legal principles.
  \end{enumerate}
\end{itemize}

\subsubsection{Legal Framework for
Identification}\label{legal-framework-for-identification}

\textbf{PPRA, s 617} outlines lawful procedures for police to gather
identification evidence: 1. Identification parade 2. Photo board with at
least 12 similar-looking individuals 3. Videotape 4. Computer-generated
images

\begin{itemize}
\tightlist
\item
  Police must follow identification procedure codes.
\item
  Participation in an identification parade is voluntary.
\end{itemize}

\subsubsection{Types of Identification
Evidence}\label{types-of-identification-evidence}

\begin{longtable}[]{@{}
  >{\raggedright\arraybackslash}p{(\linewidth - 2\tabcolsep) * \real{0.2222}}
  >{\raggedright\arraybackslash}p{(\linewidth - 2\tabcolsep) * \real{0.7778}}@{}}
\toprule\noalign{}
\begin{minipage}[b]{\linewidth}\raggedright
Type of Evidence
\end{minipage} & \begin{minipage}[b]{\linewidth}\raggedright
Description
\end{minipage} \\
\midrule\noalign{}
\endhead
\bottomrule\noalign{}
\endlastfoot
\textbf{Dock Identification} & Conducted in court; often biased as
witnesses may feel pressured to identify the accused. \\
\textbf{Informal Eye-Witness Identifications} & Witness identifies the
accused while the latter is seated in a police vehicle, increasing
liability to mistake. \\
\textbf{Voice IDs} & Identification based on voice; can be unreliable,
especially if the witness remembers only an accent. \\
\textbf{Photoboards} & Used to identify suspects; admissible in court if
proper procedures are followed. \\
\textbf{Identification Parade} & A group lineup; must be conducted
fairly to avoid any prompting or suggestion. \\
\end{longtable}

\subsubsection{Judicial Caution on Identification
Evidence}\label{judicial-caution-on-identification-evidence}

\textbf{R v Turnbull} emphasizes the need for caution when relying on
identification evidence: - Judges must warn juries about the potential
for mistaken identifications. - Factors for juries to consider include:
- Duration and conditions of observation - Previous familiarity with the
accused - Time elapsed between the event and identification - Any
discrepancies in descriptions given to police

\subsubsection{Domican Warning}\label{domican-warning}

In \textbf{Domican v R (1992)}, the judge provided the jury with
specific factors regarding identification evidence: 1. Previous
acquaintance with the identified person 2. The opportunity to observe
the person clearly 3. The time gap between the event and identification
4. Nature of the initial identification

\begin{itemize}
\tightlist
\item
  Warnings must be \textbf{detailed} and relevant to the specific case
  circumstances, ensuring juries understand the weaknesses in
  identification evidence.
\end{itemize}

\subsubsection{Judicial Summing Up}\label{judicial-summing-up}

Judges have the responsibility to instruct juries on applicable law and
evidence: - After evidence presentation, judges summarize the law and
offer observations. - Proper judicial instructions are crucial for
juries to reach informed verdicts.

\subsection{  Role of Trial Judge}\label{role-of-trial-judge}

\begin{quote}
``The fundamental task of a trial judge is, of course, to ensure a fair
trial of the accused.''
\end{quote}

\subsubsection{Responsibilities of a Trial
Judge}\label{responsibilities-of-a-trial-judge}

\begin{itemize}
\tightlist
\item
  \textbf{Ensure Fair Trial:} The judge must guarantee that the trial
  process is fair for the accused.
\item
  \textbf{Instruct the Jury:} The judge is responsible for providing the
  jury with necessary legal instructions, including:

  \begin{itemize}
  \tightlist
  \item
    Elements of the offence
  \item
    Burden and standard of proof
  \item
    Functions of judge and jury
  \end{itemize}
\item
  \textbf{Identify Issues:} The judge must identify the issues in the
  case and relate the law to those issues.
\item
  \textbf{Present Accused's Case:} The judge is required to fairly
  present the case that the accused makes.
\item
  \textbf{Guide Jury on Reasoning:} In certain cases, the judge must
  warn the jury against improper reasoning or caution them on the
  acceptance of specific types of evidence.
\end{itemize}

\subsection{  Opinion Evidence}\label{opinion-evidence}

\subsubsection{General Rule}\label{general-rule-1}

\begin{quote}
``In evidence law, the general rule is that a witness must testify about
the facts only.''
\end{quote}

\begin{itemize}
\tightlist
\item
  \textbf{Facts:} Refers to what the witness did or perceived through
  their senses.
\item
  \textbf{Irrelevance of Opinions:} A witness's opinion or speculations
  about an event are considered irrelevant.
\end{itemize}

\subsubsection{Distinguishing between Fact and
Opinion}\label{distinguishing-between-fact-and-opinion}

\begin{itemize}
\tightlist
\item
  \textbf{Exclusionary Rule:} Opinions are generally not admissible as
  evidence.
\end{itemize}

\paragraph{Exceptions to the Exclusionary
Rule}\label{exceptions-to-the-exclusionary-rule}

\begin{enumerate}
\def\labelenumi{\arabic{enumi}.}
\tightlist
\item
  \textbf{Lay Opinions:} Non-experts may provide opinion evidence under
  certain conditions.
\item
  \textbf{Expert Opinions:} Opinions from recognized experts in a
  specialized field may be admissible.
\end{enumerate}

\subsubsection{Lay Opinions}\label{lay-opinions}

\begin{itemize}
\tightlist
\item
  \textbf{Admissibility:} Opinion evidence from laypersons may be
  admissible under specific circumstances, as established in
  \textbf{Sherrard v Jacob (1965)}.
\end{itemize}

\paragraph{Established Categories of Lay Opinion
Evidence}\label{established-categories-of-lay-opinion-evidence}

\begin{longtable}[]{@{}
  >{\raggedright\arraybackslash}p{(\linewidth - 2\tabcolsep) * \real{0.5000}}
  >{\raggedright\arraybackslash}p{(\linewidth - 2\tabcolsep) * \real{0.5000}}@{}}
\toprule\noalign{}
\begin{minipage}[b]{\linewidth}\raggedright
Category
\end{minipage} & \begin{minipage}[b]{\linewidth}\raggedright
Description
\end{minipage} \\
\midrule\noalign{}
\endhead
\bottomrule\noalign{}
\endlastfoot
a) Identification & Familiar persons, things, handwriting \\
b) Apparent Age & Estimation of a person's age \\
c) Bodily Condition & Observations on health, illness, or
intoxication \\
d) Emotional State & Insights into a person's emotions \\
e) Condition of Things & Assessment of physical state or condition \\
f) Questions of Value & Opinions on value of items \\
g) Speed and Distance & Estimates related to speed and distance \\
\end{longtable}

\begin{itemize}
\tightlist
\item
  \textbf{Non-Established Categories:} If a lay opinion does not fit
  established categories, consider:

  \begin{itemize}
  \tightlist
  \item
    Is the opinion relevant and necessary?
  \item
    Is the witness's experience sufficient to form an opinion?
  \item
    Is the opinion based on facts that cannot be detailed precisely?
  \end{itemize}
\end{itemize}

\subsubsection{Alternate Approach}\label{alternate-approach}

\begin{itemize}
\tightlist
\item
  Evidence might be presented as a description based on experience
  rather than an opinion.
\end{itemize}

\paragraph{Example: Yildiz Case (1983)}\label{example-yildiz-case-1983}

\begin{itemize}
\tightlist
\item
  \textbf{Facts:} Yildiz claimed provocation for killing his
  brother-in-law due to insults regarding their alleged homosexual
  relationship.
\item
  \textbf{Ruling:} The witness's lay opinion regarding cultural
  perceptions was admissible despite not fitting established categories.
\end{itemize}

\subsection{��‍⚕️ Expert Opinions}\label{expert-opinions}

\subsubsection{Admissibility Criteria}\label{admissibility-criteria}

To admit expert evidence, the following must be satisfied:

\begin{enumerate}
\def\labelenumi{\arabic{enumi}.}
\tightlist
\item
  Is an expert's opinion necessary?
\item
  Is there a recognized field of specialized expertise?
\item
  Is the witness qualified as an expert in that field?
\item
  Is the question within the expert's expertise?
\item
  Is the evidence based on admissible evidence?
\item
  Does the evidence address the ultimate issue?
\end{enumerate}

\subsubsection{Necessity of Expert
Opinion}\label{necessity-of-expert-opinion}

\begin{itemize}
\tightlist
\item
  Expert opinions are admissible only when specialized knowledge is
  necessary to aid the jury's understanding of matters beyond their
  knowledge.
\end{itemize}

\subsubsection{Key Cases}\label{key-cases}

\begin{itemize}
\tightlist
\item
  \textbf{Farrell v R (1998)}: Expert evidence is admissible if it aids
  understanding of issues outside common knowledge.
\item
  \textbf{Murphy v R (1988)}: Court clarified that not all expert
  evidence is necessary, focusing on relevance to jurors' understanding.
\end{itemize}

\subsubsection{Notable Case: Honeysett v R
(2014)}\label{notable-case-honeysett-v-r-2014}

\begin{itemize}
\tightlist
\item
  Demonstrated the importance of expert evidence in assisting juries
  with specialized knowledge.
\end{itemize}

\subsubsection{Ordinary Human Nature}\label{ordinary-human-nature}

\begin{itemize}
\tightlist
\item
  The courts often remind us that expert evidence on ordinary human
  nature is generally inadmissible unless the circumstances are
  sufficiently unusual.
\end{itemize}

\paragraph{Example: Runjanjic \& Kontinnen Case
(1991)}\label{example-runjanjic-kontinnen-case-1991}

\begin{itemize}
\tightlist
\item
  In this case, the court allowed expert evidence on battered women's
  syndrome due to its relevance in understanding specific behaviors
  outside of jurors' general knowledge.
\end{itemize}

\subsubsection{Expert Evidence on Child Abuse Accommodation Syndrome
(CAAS)}\label{expert-evidence-on-child-abuse-accommodation-syndrome-caas}

\begin{itemize}
\tightlist
\item
  \textbf{C v R (1993)}: The court ruled that expert evidence on CAAS
  was not admissible, as it was deemed not sufficiently special to be
  beyond the jurors' understanding.
\end{itemize}

\begin{longtable}[]{@{}
  >{\raggedright\arraybackslash}p{(\linewidth - 2\tabcolsep) * \real{0.4109}}
  >{\raggedright\arraybackslash}p{(\linewidth - 2\tabcolsep) * \real{0.5891}}@{}}
\toprule\noalign{}
\begin{minipage}[b]{\linewidth}\raggedright
Case
\end{minipage} & \begin{minipage}[b]{\linewidth}\raggedright
Outcome
\end{minipage} \\
\midrule\noalign{}
\endhead
\bottomrule\noalign{}
\endlastfoot
Runjanjic \& Kontinnen & Expert evidence on battered women's syndrome
admissible. \\
C v R & Expert evidence on CAAS not admissible. \\
\end{longtable}

\subsection{  Expert Evidence in Court}\label{expert-evidence-in-court}

\subsubsection{Case Study: Honeysett v The Queen {[}2014{]} HCA
29}\label{case-study-honeysett-v-the-queen-2014-hca-29}

\paragraph{Background}\label{background}

\begin{itemize}
\tightlist
\item
  \textbf{Facts}: Honeysett was convicted of armed robbery based on CCTV
  footage where one robber's identity was obscured.
\item
  \textbf{Expert Testimony}: The prosecution presented evidence from
  Professor Henneberg, an anatomist, who identified anatomical
  similarities between Honeysett and the obscured robber (Offender One).
\end{itemize}

\paragraph{Key Points}\label{key-points-3}

\begin{itemize}
\item
  \textbf{Admissibility of Expert Opinion}: \textgreater{} ``Expert
  opinion is only admissible if the opinion derives from an accepted
  body of knowledge or established field of expertise.''
\item
  \textbf{Specialised Knowledge}: The court ruled that Henneberg's
  opinion was not based wholly or substantially on his knowledge of
  anatomy but rather on subjective impressions.
\end{itemize}

\subsubsection{Legal Framework}\label{legal-framework}

\begin{longtable}[]{@{}
  >{\raggedright\arraybackslash}p{(\linewidth - 2\tabcolsep) * \real{0.2000}}
  >{\raggedright\arraybackslash}p{(\linewidth - 2\tabcolsep) * \real{0.8000}}@{}}
\toprule\noalign{}
\begin{minipage}[b]{\linewidth}\raggedright
\textbf{Concept}
\end{minipage} & \begin{minipage}[b]{\linewidth}\raggedright
\textbf{Definition}
\end{minipage} \\
\midrule\noalign{}
\endhead
\bottomrule\noalign{}
\endlastfoot
\textbf{Specialised Knowledge} & Knowledge that is based on a body of
known facts or ideas inferred from such facts, requiring study or
experience. \\
\textbf{Field of Expertise} & An area recognized within the legal system
where expert testimony may be deemed admissible. \\
\end{longtable}

\subsubsection{Assessment of Expert
Witnesses}\label{assessment-of-expert-witnesses}

\paragraph{Recognized Fields of
Expertise}\label{recognized-fields-of-expertise}

\begin{itemize}
\tightlist
\item
  The courts evaluate whether the expert's knowledge comes from:

  \begin{itemize}
  \tightlist
  \item
    Special courses or lengthy experience.
  \item
    Peer-reviewed academic literature.
  \end{itemize}
\end{itemize}

\paragraph{Evaluating Qualifications}\label{evaluating-qualifications}

\begin{itemize}
\tightlist
\item
  Experts must:

  \begin{itemize}
  \tightlist
  \item
    Present their qualifications and relevant experience.
  \item
    Stay within their realm of expertise while testifying.
  \end{itemize}
\end{itemize}

\subsubsection{Important Cases}\label{important-cases-2}

\begin{longtable}[]{@{}
  >{\raggedright\arraybackslash}p{(\linewidth - 2\tabcolsep) * \real{0.2869}}
  >{\raggedright\arraybackslash}p{(\linewidth - 2\tabcolsep) * \real{0.7131}}@{}}
\toprule\noalign{}
\begin{minipage}[b]{\linewidth}\raggedright
\textbf{Case}
\end{minipage} & \begin{minipage}[b]{\linewidth}\raggedright
\textbf{Key Takeaway}
\end{minipage} \\
\midrule\noalign{}
\endhead
\bottomrule\noalign{}
\endlastfoot
\textbf{Clark v Ryan (1960)} & Established that expert opinions must
derive from an accepted body of knowledge. \\
\textbf{R v Gilmore (1977)} & Recognized voiceprint analysis as
admissible evidence---an evolving field of study. \\
\textbf{Honeysett v R (2014)} & Discussed the requirement of specialised
knowledge under Uniform Evidence Law. \\
\end{longtable}

\subsubsection{Limitations on Expert
Testimony}\label{limitations-on-expert-testimony}

\begin{itemize}
\tightlist
\item
  \textbf{Staying Within Expertise}: Experts must not stray beyond their
  expertise; questions may be disallowed if they do.
\end{itemize}

\subsubsection{Hearsay Rule}\label{hearsay-rule}

\begin{itemize}
\tightlist
\item
  \textbf{Admissibility of Facts}: Expert opinions based on hearsay or
  unverifiable facts are inadmissible.

  \begin{itemize}
  \tightlist
  \item
    \textbf{Example}: In Ramsay v Watson (1961), hearsay evidence was
    ruled inadmissible for forming an expert opinion.
  \end{itemize}
\end{itemize}

\subsubsection{Conclusion of the Honeysett
Case}\label{conclusion-of-the-honeysett-case}

\begin{itemize}
\tightlist
\item
  The High Court quashed Honeysett's conviction, emphasizing the need
  for expert opinions to be based on established knowledge rather than
  subjective interpretation.
\end{itemize}

\subsubsection{Conclusion}\label{conclusion-2}

\begin{itemize}
\tightlist
\item
  Understanding the framework for expert evidence is crucial for
  assessing its admissibility in legal proceedings.
\end{itemize}

\subsection{�� Expert Evidence and
Opinions}\label{expert-evidence-and-opinions}

\subsubsection{Requirements for Expert
Opinions}\label{requirements-for-expert-opinions}

In the context of expert evidence, the following must be established:

\begin{itemize}
\tightlist
\item
  \textbf{Observed Facts}: The expert must identify and prove facts
  based on their observations.
\item
  \textbf{Assumed Facts}: Facts that are assumed or accepted must also
  be identified and proven by another means.
\end{itemize}

\begin{quote}
``It must be established that the facts on which the opinion is based
form a proper foundation for it.''
\end{quote}

\begin{itemize}
\tightlist
\item
  \textbf{Demonstration of Knowledge}: The expert's evidence should
  explain how their specialized knowledge applies to the facts observed
  or assumed, leading to their opinion.
\end{itemize}

\subsubsection{Ultimate Issue Rule}\label{ultimate-issue-rule}

\begin{itemize}
\item
  \textbf{Definition}: The ultimate issue rule prohibits experts from
  giving evidence directly on the ultimate issue that the jury needs to
  decide.
\item
  \textbf{Historical Context}: Traditionally, this rule meant that an
  expert could not state whether the accused committed a crime or if a
  defendant was negligent.
\item
  \textbf{Modern Application}: In cases like \emph{Clark v Ryan}, this
  rule is now seldom enforced, allowing experts to provide opinions that
  may touch on the ultimate issue.
\end{itemize}

\paragraph{Examples of Expert
Testimony}\label{examples-of-expert-testimony}

\begin{itemize}
\tightlist
\item
  Experts may discuss the likelihood of DNA being from someone other
  than the accused rather than stating the accused did commit the crime.
\end{itemize}

\subsubsection{The Role of the Jury}\label{the-role-of-the-jury}

\begin{itemize}
\item
  Courts emphasize that it is the jury's role to determine facts, not
  the experts'.
\item
  If an expert's evidence is uncontradicted, the jury must act in
  accordance with it, as seen in \emph{R v De Voss}.
\end{itemize}

\subsubsection{\texorpdfstring{Case Study: \emph{Farrell v
R}}{Case Study: Farrell v R}}\label{case-study-farrell-v-r}

\begin{itemize}
\tightlist
\item
  An expert discussed a victim's personality disorder affecting their
  credibility but could not directly address whether the victim was
  lying about the assault.
\end{itemize}

\subsection{��️ Expert Evidence
Procedures}\label{expert-evidence-procedures}

\subsubsection{Civil Cases}\label{civil-cases}

\begin{itemize}
\tightlist
\item
  Governed by \textbf{Part 5 of the UCPR}:

  \begin{itemize}
  \tightlist
  \item
    Primary duty of experts is to the court, not to their client.
  \item
    Experts typically provide evidence through written reports rather
    than oral testimony, but must be available for cross-examination.
  \end{itemize}
\end{itemize}

\begin{longtable}[]{@{}
  >{\raggedright\arraybackslash}p{(\linewidth - 2\tabcolsep) * \real{0.3158}}
  >{\raggedright\arraybackslash}p{(\linewidth - 2\tabcolsep) * \real{0.6842}}@{}}
\toprule\noalign{}
\begin{minipage}[b]{\linewidth}\raggedright
Rule
\end{minipage} & \begin{minipage}[b]{\linewidth}\raggedright
Description
\end{minipage} \\
\midrule\noalign{}
\endhead
\bottomrule\noalign{}
\endlastfoot
r 426 & Experts must prioritize duty to the court. \\
r 427 & Evidence provided by written report; cross-examination
allowed. \\
r 429B & Experts with conflicting views may meet to isolate agreements
and disagreements. \\
\end{longtable}

\subsubsection{Criminal Cases}\label{criminal-cases-1}

\begin{itemize}
\tightlist
\item
  Prosecution must disclose any expert reports related to tests or
  forensic procedures.
\item
  Accused must notify prosecution about the intended use of expert
  evidence promptly.
\end{itemize}

\subsection{ Process for Expert
Testimony}\label{process-for-expert-testimony}

\begin{enumerate}
\def\labelenumi{\arabic{enumi}.}
\tightlist
\item
  \textbf{Calling the Expert}: The advocate introduces the expert, e.g.,
  ``The prosecution calls Dr.~Thuy Nguyen.''
\item
  \textbf{Identifying the Witness}: Name, occupation, and relevant
  background are established.
\item
  \textbf{Qualifying the Witness}: Questions regarding degrees,
  experience, and affiliations are asked to establish credibility.
\end{enumerate}

\subsubsection{Tendering Expert Reports}\label{tendering-expert-reports}

\begin{itemize}
\tightlist
\item
  The lawyer must:

  \begin{itemize}
  \tightlist
  \item
    Ask the judge if the witness can see a document.
  \item
    Have the witness identify and confirm the document's contents.
  \end{itemize}
\end{itemize}

\begin{quote}
``This is essential, as it means the witness adopts the content of the
document.''
\end{quote}

\begin{enumerate}
\def\labelenumi{\arabic{enumi}.}
\setcounter{enumi}{3}
\tightlist
\item
  \textbf{Cross-Examination}:

  \begin{itemize}
  \tightlist
  \item
    Limited grounds for challenging independent expert evidence.
  \item
    Questions may focus on:

    \begin{itemize}
    \tightlist
    \item
      Basis of expertise.
    \item
      Validity of the opinion based on additional information.
    \item
      Doubts regarding the accused's guilt.
    \end{itemize}
  \end{itemize}
\end{enumerate}

\subsection{ DNA Evidence and
Statistics}\label{dna-evidence-and-statistics}

\subsubsection{Key Concepts}\label{key-concepts-1}

\begin{itemize}
\tightlist
\item
  \textbf{DNA Evidence Introduction}: First used in Australia in 1989 in
  \emph{Applebee (ACT criminal case)}.
\end{itemize}

\begin{longtable}[]{@{}
  >{\raggedright\arraybackslash}p{(\linewidth - 2\tabcolsep) * \real{0.5714}}
  >{\raggedright\arraybackslash}p{(\linewidth - 2\tabcolsep) * \real{0.4286}}@{}}
\toprule\noalign{}
\begin{minipage}[b]{\linewidth}\raggedright
Statistic Type
\end{minipage} & \begin{minipage}[b]{\linewidth}\raggedright
Definition
\end{minipage} \\
\midrule\noalign{}
\endhead
\bottomrule\noalign{}
\endlastfoot
Base Rate Statistics & Probability of a characteristic in the overall
population (e.g., chance of guilt). \\
Incidence Rate Statistics & Use of base rate statistics to infer
circumstantial evidence regarding guilt. \\
\end{longtable}

\subsubsection{Confidence Intervals}\label{confidence-intervals}

\begin{itemize}
\tightlist
\item
  Expressed as the likelihood of a match based on large databases.
\item
  Example: ``It is 95\% certain that DNA evidence is between 2.67
  million and 2.85 million times more likely to come from the accused
  than from a randomly selected individual.''
\end{itemize}

\subsection{ Evidence and the Prosecutor's
Fallacy}\label{evidence-and-the-prosecutors-fallacy}

\subsubsection{Blood Evidence Analysis}\label{blood-evidence-analysis}

\begin{itemize}
\tightlist
\item
  \textbf{Crime Scene Context:}

  \begin{itemize}
  \tightlist
  \item
    In Brisbane, blood was found on a smashed window where the
    perpetrator exited after committing an offense.
  \item
    The blood type found matches that of the defendant, which is shared
    by \textbf{10\%} of the population.
  \end{itemize}
\end{itemize}

\begin{quote}
``This evidence indicates that the blood match is \textbf{not
inconsistent} with the defendant being the offender, making it more
likely he is the culprit compared to another random individual. However,
it does not establish \textbf{certainty}.''
\end{quote}

\subsubsection{Impact of Population
Size}\label{impact-of-population-size}

\begin{itemize}
\tightlist
\item
  \textbf{Population Consideration:}

  \begin{itemize}
  \tightlist
  \item
    In Greater Brisbane, with approximately \textbf{2.27 million}
    people, around \textbf{227,000} individuals would share the same
    blood type as the defendant.
  \end{itemize}
\item
  \textbf{Implication:}

  \begin{itemize}
  \tightlist
  \item
    This leads to the potential for misinterpretation of statistical
    evidence.
  \end{itemize}
\end{itemize}

\subsubsection{Misuse of Statistical
Evidence}\label{misuse-of-statistical-evidence}

\begin{longtable}[]{@{}
  >{\raggedright\arraybackslash}p{(\linewidth - 4\tabcolsep) * \real{0.1549}}
  >{\raggedright\arraybackslash}p{(\linewidth - 4\tabcolsep) * \real{0.7466}}
  >{\raggedright\arraybackslash}p{(\linewidth - 4\tabcolsep) * \real{0.0985}}@{}}
\toprule\noalign{}
\begin{minipage}[b]{\linewidth}\raggedright
Fallacy Type
\end{minipage} & \begin{minipage}[b]{\linewidth}\raggedright
Description
\end{minipage} & \begin{minipage}[b]{\linewidth}\raggedright
Consequences
\end{minipage} \\
\midrule\noalign{}
\endhead
\bottomrule\noalign{}
\endlastfoot
\textbf{Prosecutor's Fallacy} & Misinterprets the probability associated
with shared characteristics. Assumes a \(1/x\) chance implies a guilt
probability of \(90\%\). & Ignores other evidence and overstates
guilt. \\
\textbf{Defence Fallacy} & Claims that matching characteristics are
irrelevant, focusing on the large group they belong to (e.g., \(1\) in
\(227,000\) chance). & Disregards other evidence and overstates
innocence. \\
\end{longtable}

\subsubsection{Case Studies}\label{case-studies}

\paragraph{R v Keir {[}2002{]} NSWCCA 30}\label{r-v-keir-2002-nswcca-30}

\begin{itemize}
\tightlist
\item
  \textbf{Background:}

  \begin{itemize}
  \tightlist
  \item
    Mrs.~Keir went missing, and her husband was charged with her murder.
  \item
    Evidence suggested a \textbf{660,000:1} statistical probability that
    the bones found were hers.
  \end{itemize}
\end{itemize}

\begin{quote}
``The jury's decision was heavily influenced by the statistical claim,
leading to the case being overturned due to the prosecutor's fallacy.''
\end{quote}

\paragraph{Aytugrul v R {[}2012{]} HCA
15}\label{aytugrul-v-r-2012-hca-15}

\begin{itemize}
\tightlist
\item
  \textbf{Context:}

  \begin{itemize}
  \tightlist
  \item
    Expert evidence indicated \textbf{1 in 1,600} chance of sharing a
    DNA profile.
  \item
    \textbf{Exclusion percentage:} 99.9\% of people would not match this
    DNA profile.
  \end{itemize}
\end{itemize}

\begin{quote}
``The court ruled that the evidence was not unfairly prejudicial, as it
provided relevant context to the exclusion percentage.''
\end{quote}

\subsection{  Admissibility of
Evidence}\label{admissibility-of-evidence-1}

\subsubsection{Four Stages in Assessing
Admissibility}\label{four-stages-in-assessing-admissibility}

\begin{longtable}[]{@{}
  >{\raggedright\arraybackslash}p{(\linewidth - 2\tabcolsep) * \real{0.1680}}
  >{\raggedright\arraybackslash}p{(\linewidth - 2\tabcolsep) * \real{0.8320}}@{}}
\toprule\noalign{}
\begin{minipage}[b]{\linewidth}\raggedright
Stage
\end{minipage} & \begin{minipage}[b]{\linewidth}\raggedright
Description
\end{minipage} \\
\midrule\noalign{}
\endhead
\bottomrule\noalign{}
\endlastfoot
\textbf{1. Relevance} & Evidence must rationally affect the assessment
of a fact in issue (Wilson v R). \\
\textbf{2. Exclusionary Rules} & Certain rules prevent admission of
relevant evidence, though exceptions may apply. \\
\textbf{3. Discretions} & Judges may refuse to admit otherwise
admissible evidence; however, there is no discretion for inadmissible
evidence. \\
\textbf{4. Privilege} & Various privileges may prevent the court from
receiving certain evidence. \\
\end{longtable}

\subsubsection{Relevance Definition}\label{relevance-definition}

\begin{quote}
``At common law, relevance lacks a definitive definition, regarded as a
matter of logic and experience. Under the Evidence Act 1995 (Cth),
relevance is determined by whether the evidence could rationally affect
the assessment of a fact in issue.''
\end{quote}

\subsection{  The CSI Effect}\label{the-csi-effect}

\begin{itemize}
\tightlist
\item
  \textbf{Phenomenon Overview:}

  \begin{itemize}
  \tightlist
  \item
    Concerns about jurors being influenced by forensic evidence seen in
    television shows.
  \end{itemize}
\end{itemize}

\subsubsection{Juror Behavior Insights}\label{juror-behavior-insights}

\begin{itemize}
\tightlist
\item
  \textbf{Research Findings:}

  \begin{itemize}
  \tightlist
  \item
    Jurors are careful not to accept forensic evidence blindly.
  \item
    They consider all evidence, forensic or otherwise, and assess its
    weight based on the entirety of the case.
  \end{itemize}
\end{itemize}

\begin{quote}
``Jurors are not easily swayed by forensic science; they actively seek
to understand and evaluate all evidence presented.''
\end{quote}

\subsection{ Relevance of Evidence}\label{relevance-of-evidence}

\subsubsection{Definitions and Concepts}\label{definitions-and-concepts}

\begin{quote}
``Relevance refers to the logical connection between evidence and the
fact in issue, determining whether the evidence can make a fact more or
less probable.''
\end{quote}

\paragraph{Probative Value}\label{probative-value}

\begin{itemize}
\tightlist
\item
  \textbf{Probative value} is defined as:
\end{itemize}

\[\text{Probative Value} = \text{the extent to which evidence could rationally affect the assessment of the probability of the existence of a fact in issue}\]

\begin{itemize}
\tightlist
\item
  It assesses the \textbf{tendency of evidence} to prove or disprove a
  fact.
\end{itemize}

\subsubsection{Legal Framework}\label{legal-framework-1}

\begin{itemize}
\tightlist
\item
  \textbf{Evidence Act 1995 (Cth)} - Section 3(1) provides the
  definition of probative value.
\item
  \textbf{Section 137}: The court must refuse to admit evidence if its
  probative value is outweighed by the danger of unfair prejudice to the
  defendant.
\end{itemize}

\subsubsection{Example Case}\label{example-case}

\begin{itemize}
\tightlist
\item
  \textbf{Smith v R (2001) 206 CLR 650 HCA}:

  \begin{itemize}
  \tightlist
  \item
    Issue: Evidence from police that merely expressed their opinion
    without substantial grounding was deemed inadmissible.
  \item
    Rationale: The police had no better insight than the jury to
    identify the accused from a security photo.
  \end{itemize}
\end{itemize}

\subsubsection{Threshold for Relevance}\label{threshold-for-relevance}

\begin{itemize}
\tightlist
\item
  \textbf{Section 56(2) EA 1995 (Cth)}: Relevant evidence is admissible,
  while irrelevant evidence is not.
\item
  \textbf{Section 55(2) EAC}: Evidence is not considered irrelevant
  solely because it relates to:

  \begin{itemize}
  \tightlist
  \item
    Credibility of a witness
  \item
    Admissibility of other evidence
  \item
    A failure to adduce evidence
  \end{itemize}
\end{itemize}

\subsubsection{Case Analysis}\label{case-analysis}

\begin{itemize}
\tightlist
\item
  \textbf{Smith v The Queen (2001)}:

  \begin{itemize}
  \tightlist
  \item
    Key Issue: Identification of the accused in a series of photographs
    from a robbery.
  \item
    Ruling: Evidence of police officers identifying the accused was
    inadmissible as it did not provide a rational basis for identity
    assessment.
  \end{itemize}
\end{itemize}

\subsubsection{Degrees of Relevance}\label{degrees-of-relevance}

\begin{itemize}
\tightlist
\item
  \textbf{Minimum Relevance}:

  \begin{itemize}
  \tightlist
  \item
    The bare connection required to relate evidence to the case.
  \end{itemize}
\item
  \textbf{Legal Relevance}:

  \begin{itemize}
  \tightlist
  \item
    Evidence must be sufficiently relevant to warrant the court's
    attention.
  \end{itemize}
\end{itemize}

\begin{longtable}[]{@{}
  >{\raggedright\arraybackslash}p{(\linewidth - 2\tabcolsep) * \real{0.5000}}
  >{\raggedright\arraybackslash}p{(\linewidth - 2\tabcolsep) * \real{0.5000}}@{}}
\toprule\noalign{}
\begin{minipage}[b]{\linewidth}\raggedright
\textbf{Type of Relevance}
\end{minipage} & \begin{minipage}[b]{\linewidth}\raggedright
\textbf{Description}
\end{minipage} \\
\midrule\noalign{}
\endhead
\bottomrule\noalign{}
\endlastfoot
Minimum Relevance & Requires a minimal logical connection \\
Legal Relevance & Must justify consideration without wasting court
time \\
\end{longtable}

\subsubsection{Case Comparisons}\label{case-comparisons}

\begin{itemize}
\tightlist
\item
  \textbf{R v Buchanan {[}1966{]} VR 9}:

  \begin{itemize}
  \tightlist
  \item
    Relevant as it related to the accused's state of mind at the time of
    a driving incident.
  \end{itemize}
\item
  \textbf{R v Horvath {[}1972{]} VR 533}:

  \begin{itemize}
  \tightlist
  \item
    Evidence deemed insufficiently relevant due to lack of connection to
    the critical issue of driving at the time of the accident.
  \end{itemize}
\end{itemize}

\subsubsection{Judicial Discretion}\label{judicial-discretion}

\begin{itemize}
\tightlist
\item
  \textbf{Section 135 EA 1995 (Cth)}:

  \begin{itemize}
  \tightlist
  \item
    Allows a judge to refuse evidence which, although relevant, would
    waste time.
  \end{itemize}
\item
  \textbf{R v Stephenson}:

  \begin{itemize}
  \tightlist
  \item
    Established that not all logically relevant evidence is legally
    relevant; the evidence must have sufficient relevance to be
    admissible.
  \end{itemize}
\end{itemize}

\subsubsection{Conclusion of Relevance}\label{conclusion-of-relevance}

\begin{itemize}
\tightlist
\item
  Evidence must provide a meaningful contribution to the assessment of
  the principal issue.
\item
  \textbf{Legal Relevance}: The probative value must not be so minimal
  that it fails to add or detract from the probability of the principal
  issue being established.
\end{itemize}

\subsubsection{Summary of Legal
Principles}\label{summary-of-legal-principles}

\begin{itemize}
\tightlist
\item
  If evidence is \textbf{not relevant}, it cannot be admitted.
\item
  Only \textbf{relevant evidence} prompts further questions about its
  admissibility.
\item
  Courts distinguish between \textbf{logical relevance} (bare minimum)
  and \textbf{legal relevance} (sufficient to justify consideration).
\end{itemize}

\subsection{  Motor Vehicle Offences}\label{motor-vehicle-offences}

\subsubsection{Dangerous Operation of a Vehicle Causing
Death}\label{dangerous-operation-of-a-vehicle-causing-death}

\begin{itemize}
\tightlist
\item
  The offence of \textbf{dangerous operation of a vehicle causing death}
  can be illustrated through a case where the driver of a Fiat was
  involved in a collision.
\item
  Although the driver was not identified, the court considered the
  relevance of evidence concerning the blood-alcohol readings of
  potential drivers.
\end{itemize}

\subsubsection{Key Court Ruling}\label{key-court-ruling}

\begin{quote}
``Evidence of the condition of the driver could only be relevant if it
could properly be relied upon to raise reasonable doubt as to: 1.
Whether the applicant was negligent; 2. Whether his negligence was of
the quality required by the respective charge; 3. Whether his negligence
caused the deaths.''
\end{quote}

\begin{itemize}
\tightlist
\item
  The court ruled that without identifying the driver, evidence
  regarding their condition did not address the key questions of
  negligence and causation.
\end{itemize}

\subsubsection{Admissibility of
Evidence}\label{admissibility-of-evidence-2}

\begin{itemize}
\tightlist
\item
  The \textbf{Full Supreme Court} highlighted that evidence must
  establish a logical connection to the issue at hand. If evidence about
  the driver's intoxication cannot directly influence the question of
  negligence or causation, it may be deemed inadmissible.
\end{itemize}

\paragraph{Criteria for Relevance:}\label{criteria-for-relevance}

\begin{longtable}[]{@{}
  >{\raggedright\arraybackslash}p{(\linewidth - 2\tabcolsep) * \real{0.3418}}
  >{\raggedright\arraybackslash}p{(\linewidth - 2\tabcolsep) * \real{0.6582}}@{}}
\toprule\noalign{}
\begin{minipage}[b]{\linewidth}\raggedright
Criteria
\end{minipage} & \begin{minipage}[b]{\linewidth}\raggedright
Description
\end{minipage} \\
\midrule\noalign{}
\endhead
\bottomrule\noalign{}
\endlastfoot
Negligence & Evidence must address whether the applicant was
negligent. \\
Quality of Negligence & Evidence must pertain to whether the negligence
meets the required legal standard. \\
Causation & Evidence must demonstrate a causal link between negligence
and the deaths. \\
\end{longtable}

\subsubsection{Facts in Issue}\label{facts-in-issue}

\begin{itemize}
\tightlist
\item
  A \textbf{fact in issue} is a fact that a party must prove to succeed
  in their case.
\item
  The facts in issue are defined by substantive law and pleadings,
  reflecting material facts that constitute the claimant's cause of
  action.
\end{itemize}

\paragraph{Examples of Facts in
Issue:}\label{examples-of-facts-in-issue}

\begin{longtable}[]{@{}
  >{\raggedright\arraybackslash}p{(\linewidth - 2\tabcolsep) * \real{0.2222}}
  >{\raggedright\arraybackslash}p{(\linewidth - 2\tabcolsep) * \real{0.7778}}@{}}
\toprule\noalign{}
\begin{minipage}[b]{\linewidth}\raggedright
Context
\end{minipage} & \begin{minipage}[b]{\linewidth}\raggedright
Description
\end{minipage} \\
\midrule\noalign{}
\endhead
\bottomrule\noalign{}
\endlastfoot
Criminal & Facts supporting or refuting the elements of the offence and
any defences raised. \\
Civil & Determined by the nature of the action, e.g., in negligence
cases, only the quantum of damages may be in issue. \\
\end{longtable}

\subsubsection{Exclusionary Rules}\label{exclusionary-rules}

\begin{itemize}
\tightlist
\item
  Evidence deemed inadmissible for one purpose may still be admissible
  for another, provided a suitable warning is given to the jury
  regarding its limited purpose.
\end{itemize}

\paragraph{Example Case:}\label{example-case-1}

\begin{itemize}
\tightlist
\item
  In \textbf{Subramaniam v Public Prosecutor}, evidence of threats made
  by terrorists was inadmissible to prove their true intentions but was
  admissible to show its effect on the accused's mental state.
\end{itemize}

\subsection{ Real and Documentary
Evidence}\label{real-and-documentary-evidence}

\subsubsection{Importance of Human
Testimony}\label{importance-of-human-testimony}

\begin{itemize}
\tightlist
\item
  \textbf{Real and documentary evidence} often requires human testimony
  to explain their relevance to the material facts of the case.
\end{itemize}

\subsubsection{Real Evidence}\label{real-evidence}

\begin{itemize}
\tightlist
\item
  \textbf{Real evidence} refers to physical items that are part of the
  evidence, which can include:

  \begin{itemize}
  \tightlist
  \item
    Stolen property
  \item
    Murder weapons
  \item
    DNA samples
  \end{itemize}
\end{itemize}

\paragraph{Inferences from Real
Evidence:}\label{inferences-from-real-evidence}

\begin{itemize}
\tightlist
\item
  Observations about the appearance, condition, or location of real
  evidence can lead to conclusions without oral testimony, although
  context is often necessary.
\end{itemize}

\subsubsection{Categories of Real
Evidence}\label{categories-of-real-evidence}

\begin{longtable}[]{@{}
  >{\raggedright\arraybackslash}p{(\linewidth - 2\tabcolsep) * \real{0.2234}}
  >{\raggedright\arraybackslash}p{(\linewidth - 2\tabcolsep) * \real{0.7766}}@{}}
\toprule\noalign{}
\begin{minipage}[b]{\linewidth}\raggedright
Category
\end{minipage} & \begin{minipage}[b]{\linewidth}\raggedright
Example
\end{minipage} \\
\midrule\noalign{}
\endhead
\bottomrule\noalign{}
\endlastfoot
Physical Objects & A dog brought into court for assessment of its
temperament. \\
Documentary Evidence & ATM slip found on a person charged with theft,
inferring guilt. \\
Witness Demeanor & The demeanor of a witness can influence the
credibility of their testimony. \\
\end{longtable}

\subsubsection{Use of Charts}\label{use-of-charts}

\begin{itemize}
\tightlist
\item
  Charts prepared for trial purposes can be admitted as real evidence,
  aiding the jury in understanding complex cases.
\end{itemize}

\subsubsection{Conclusion on Evidence
Types}\label{conclusion-on-evidence-types}

\begin{itemize}
\tightlist
\item
  Both \textbf{real evidence} and \textbf{documents} play a critical
  role in the trial process, supported by witness testimony to establish
  their significance and connection to the case.
\end{itemize}

\subsection{ Real Evidence in Court}\label{real-evidence-in-court}

\subsubsection{Photographs as Real
Evidence}\label{photographs-as-real-evidence}

\begin{itemize}
\tightlist
\item
  \textbf{R v Ames {[}1964-65{]} NSWLR 1489}

  \begin{itemize}
  \tightlist
  \item
    Photographs of wounds on a corpse were admitted.
  \item
    Demonstrated a blood splatter pattern indicating murder over
    suicide.
  \item
    Allowed the court to draw its own conclusions about the location of
    the body.
  \end{itemize}
\item
  \textbf{R v Ireland (1970) 126 CLR 321}

  \begin{itemize}
  \tightlist
  \item
    Photographs of the accused's hands were used to show evidence of a
    fight.
  \end{itemize}
\end{itemize}

\subsubsection{Views, Demonstrations, and
Reconstructions}\label{views-demonstrations-and-reconstructions}

\begin{itemize}
\tightlist
\item
  \textbf{Views}

  \begin{itemize}
  \tightlist
  \item
    A visit to a location by the trier of fact to better understand the
    evidence.
  \item
    Not a form of real evidence but serves as an aid.
  \end{itemize}
\item
  \textbf{Demonstrations}

  \begin{itemize}
  \tightlist
  \item
    Attempt to reconstruct an event relevant to the case.
  \item
    Considered a form of real evidence.
  \end{itemize}
\item
  \textbf{Scott v Numurkah Corporation (1994) 91 CLR 300}

  \begin{itemize}
  \tightlist
  \item
    A judge visited a hall and witnessed a demonstration.
  \item
    The High Court ruled that a view aids understanding while a
    demonstration is a form of real evidence.
  \end{itemize}
\item
  \textbf{R v Alexander {[}1979{]} VR 615, 631}

  \begin{itemize}
  \tightlist
  \item
    Emphasized the importance of distinguishing between a view and the
    taking of evidence.
  \end{itemize}
\end{itemize}

\subsubsection{Tape Recordings}\label{tape-recordings}

\begin{itemize}
\tightlist
\item
  Tape recordings can be hearsay if relying on the truth of the words
  said.
\item
  May be introduced as real evidence, allowing the trier of fact to draw
  conclusions about the recorded conversation.
\item
  \textbf{Butera v DPP (1987) 164 CLR 180}

  \begin{itemize}
  \tightlist
  \item
    Transcripts from recordings may be considered real evidence if
    expert testimony explains the contents.
  \end{itemize}
\end{itemize}

\subsubsection{Fingerprints and DNA
Evidence}\label{fingerprints-and-dna-evidence}

\begin{itemize}
\tightlist
\item
  Expert witnesses may identify fingerprints and use photographs to
  demonstrate similarities with the accused's prints.
\item
  DNA evidence may also be presented with statistical analyses to assist
  understanding.
\item
  \textbf{s 95A EA 1977 (QLD)}

  \begin{itemize}
  \tightlist
  \item
    Allows for the admission of DNA evidentiary certificates, but
    typically expert testimony is preferred.
  \end{itemize}
\end{itemize}

\subsection{ Documentary Evidence}\label{documentary-evidence}

\subsubsection{Admissibility of
Documents}\label{admissibility-of-documents}

\begin{itemize}
\tightlist
\item
  Documents must be original evidence or admissible under an exception
  to the hearsay rule.
\item
  \textbf{Types of Evidence:}

  \begin{itemize}
  \tightlist
  \item
    \textbf{Original Evidence:} Real evidence that can be directly
    observed.
  \item
    \textbf{Testimonial Evidence:} Relevant only because of the
    testimony contained within them.
  \end{itemize}
\end{itemize}

\subsubsection{Key Definitions}\label{key-definitions}

\begin{quote}
``Document includes a variety of forms such as writing, photographs,
sound recordings, and any other record of information.''
\end{quote}

\subsubsection{How a Document Becomes
Evidence}\label{how-a-document-becomes-evidence}

\begin{enumerate}
\def\labelenumi{\arabic{enumi}.}
\tightlist
\item
  \textbf{Real Evidence:} Document is significant due to its physical
  presence or condition.
\item
  \textbf{Evidence of Legal Transactions:} Examples include receipts or
  title certificates.
\item
  \textbf{Silent Testimony:} Documents that record terms of agreements,
  which must be supported by exceptions to hearsay rules.
\end{enumerate}

\subsubsection{Hearsay and Original
Documents}\label{hearsay-and-original-documents}

\begin{itemize}
\tightlist
\item
  Original documents are required unless an explanation is provided for
  their absence.
\item
  The rule applies primarily to traditional paper documents, not to
  modern digital records.
\end{itemize}

\subsubsection{Process for Document
Admissibility}\label{process-for-document-admissibility}

\begin{itemize}
\tightlist
\item
  Must comply with sections 95(1) and 95(2)-(7) of the Evidence Act to
  establish the admissibility of documents generated by devices or
  processes.
\end{itemize}

This structure ensures that each section is well-defined and that
students can clearly understand the distinctions between types of
evidence discussed in the lecture.

\subsection{ Parol Evidence Rule and Documentary
Evidence}\label{parol-evidence-rule-and-documentary-evidence}

\subsubsection{Parol Evidence Rule}\label{parol-evidence-rule}

\begin{quote}
``A party cannot seek to amend, vary or contradict what appear to be
clear unequivocal terms of a contract or other binding document by means
of oral evidence.''
\end{quote}

\begin{itemize}
\tightlist
\item
  This rule has numerous exceptions and qualifications in modern
  practice, diminishing its effectiveness.
\item
  Reference: EA 1995 (Cth) s 51.
\end{itemize}

\subsubsection{Provisions for Document
Contents}\label{provisions-for-document-contents}

\begin{itemize}
\tightlist
\item
  Section 48 outlines various methods for proving the contents of a
  document, including the production of a copy under \(s48(1)(b)\).
\end{itemize}

\subsection{  Purposes for Introducing
Documents}\label{purposes-for-introducing-documents}

Documents may serve several purposes in legal contexts, subject to
statutory provisions and common law limitations:

\begin{longtable}[]{@{}
  >{\raggedright\arraybackslash}p{(\linewidth - 4\tabcolsep) * \real{0.1024}}
  >{\raggedright\arraybackslash}p{(\linewidth - 4\tabcolsep) * \real{0.1805}}
  >{\raggedright\arraybackslash}p{(\linewidth - 4\tabcolsep) * \real{0.7171}}@{}}
\toprule\noalign{}
\begin{minipage}[b]{\linewidth}\raggedright
Purpose
\end{minipage} & \begin{minipage}[b]{\linewidth}\raggedright
Type of Evidence
\end{minipage} & \begin{minipage}[b]{\linewidth}\raggedright
Description
\end{minipage} \\
\midrule\noalign{}
\endhead
\bottomrule\noalign{}
\endlastfoot
Non-Testimonial & NOT HEARSAY & Documents are relevant for their
existence or condition rather than the truth of their contents. For
example, a will evidencing a legal transaction. \\
Testimonial & HEARSAY & Documents are used to prove the truth of their
contents or relevant statements, requiring adherence to hearsay
exceptions. \\
\end{longtable}

\subsection{�� Hearsay and its
Exceptions}\label{hearsay-and-its-exceptions}

\subsubsection{Hearsay Evidence}\label{hearsay-evidence}

\begin{quote}
``Hearsay evidence consists of the passing on of information given to
the maker of the statement by someone else who experienced what is
described.''
\end{quote}

\begin{itemize}
\tightlist
\item
  Documents can be considered hearsay if introduced to prove the truth
  of their contents, such as a letter offered to confirm its assertions.
\end{itemize}

\subsubsection{Exceptions to Hearsay for Statements in
Documents}\label{exceptions-to-hearsay-for-statements-in-documents}

Hearsay statements contained in documents may be admissible under
certain conditions defined in EA 1977 (Qld):

\begin{enumerate}
\def\labelenumi{\arabic{enumi}.}
\tightlist
\item
  \textbf{For Civil Cases (Section 92)}:

  \begin{itemize}
  \tightlist
  \item
    Direct oral evidence of a fact would be admissible.
  \item
    The maker of the document must have personal knowledge or the
    document must be part of an undertaking.
  \item
    If the maker is unavailable (e.g., deceased, unfit, or out of
    state), the document can be used without the witness.
  \end{itemize}
\end{enumerate}

\subsection{  Documentary Evidence and the Best Evidence
Rule}\label{documentary-evidence-and-the-best-evidence-rule}

\subsubsection{Best Evidence Rule}\label{best-evidence-rule}

\begin{itemize}
\tightlist
\item
  Common law principle that requires courts to consider the best
  available evidence.
\item
  The original document is preferred unless a plausible explanation for
  its absence is provided.
\end{itemize}

\subsubsection{Statutory Provisions}\label{statutory-provisions}

\begin{itemize}
\tightlist
\item
  \textbf{Section 116 EAQ}: Machine copies are admissible without proof
  of comparison with the original document.
\end{itemize}

\begin{longtable}[]{@{}
  >{\raggedright\arraybackslash}p{(\linewidth - 2\tabcolsep) * \real{0.2048}}
  >{\raggedright\arraybackslash}p{(\linewidth - 2\tabcolsep) * \real{0.7952}}@{}}
\toprule\noalign{}
\begin{minipage}[b]{\linewidth}\raggedright
Provision
\end{minipage} & \begin{minipage}[b]{\linewidth}\raggedright
Description
\end{minipage} \\
\midrule\noalign{}
\endhead
\bottomrule\noalign{}
\endlastfoot
\(s116\) Copies to be evidence & Copies made by machine are admissible
if proven to be taken from the original without the need to produce the
original document. \\
\end{longtable}

\subsubsection{Copies and
Authentication}\label{copies-and-authentication}

\begin{itemize}
\tightlist
\item
  \textbf{Section 97 EAQ}: Allows for the production of a copy of a
  document as evidence if authenticated by the court.
\end{itemize}

\begin{longtable}[]{@{}
  >{\raggedright\arraybackslash}p{(\linewidth - 2\tabcolsep) * \real{0.2048}}
  >{\raggedright\arraybackslash}p{(\linewidth - 2\tabcolsep) * \real{0.7952}}@{}}
\toprule\noalign{}
\begin{minipage}[b]{\linewidth}\raggedright
Provision
\end{minipage} & \begin{minipage}[b]{\linewidth}\raggedright
Description
\end{minipage} \\
\midrule\noalign{}
\endhead
\bottomrule\noalign{}
\endlastfoot
\(s97\) Authentication & A statement in a document can be proved by
producing a copy authenticated as approved by the court. \\
\end{longtable}

\subsection{  Documentary Evidence and Books of
Account}\label{documentary-evidence-and-books-of-account}

\subsubsection{Relevant Case: Investments Commission v Rich
(2005)}\label{relevant-case-investments-commission-v-rich-2005}

\begin{itemize}
\tightlist
\item
  Any document recording transactions is admissible under certain
  sections.
\end{itemize}

\begin{longtable}[]{@{}
  >{\raggedright\arraybackslash}p{(\linewidth - 2\tabcolsep) * \real{0.2114}}
  >{\raggedright\arraybackslash}p{(\linewidth - 2\tabcolsep) * \real{0.7886}}@{}}
\toprule\noalign{}
\begin{minipage}[b]{\linewidth}\raggedright
Section
\end{minipage} & \begin{minipage}[b]{\linewidth}\raggedright
Description
\end{minipage} \\
\midrule\noalign{}
\endhead
\bottomrule\noalign{}
\endlastfoot
\(s83\) Definitions & Defines ``book of account'' as any document used
in the ordinary course of business to record transactions. \\
\(s84\) Entries in Book & Entries in a book of account are evidence of
recorded transactions. \\
\end{longtable}

\subsection{ Verification of Copies}\label{verification-of-copies}

\subsubsection{Verification
Requirements}\label{verification-requirements}

\begin{itemize}
\tightlist
\item
  A copy of an entry in a book of account must be proved correct and
  examined against the original entry to be admissible.
\end{itemize}

\begin{longtable}[]{@{}
  >{\raggedright\arraybackslash}p{(\linewidth - 2\tabcolsep) * \real{0.2114}}
  >{\raggedright\arraybackslash}p{(\linewidth - 2\tabcolsep) * \real{0.7886}}@{}}
\toprule\noalign{}
\begin{minipage}[b]{\linewidth}\raggedright
Section
\end{minipage} & \begin{minipage}[b]{\linewidth}\raggedright
Description
\end{minipage} \\
\midrule\noalign{}
\endhead
\bottomrule\noalign{}
\endlastfoot
\(s85\) Proof of Book & Proves that the book is an ordinary book of
account and that entries were made in the usual course. \\
\(s86\) Verification & A copy of an entry is inadmissible unless
verified as correct by a person who examined it with the original. \\
\end{longtable}

\subsection{ Admissibility of Evidence in Legal
Proceedings}\label{admissibility-of-evidence-in-legal-proceedings}

\subsubsection{General Rules for
Admissibility}\label{general-rules-for-admissibility}

\begin{itemize}
\tightlist
\item
  \textbf{Section 92}: \textbf{Admissibility of Documentary Evidence in
  Civil Cases}

  \begin{itemize}
  \tightlist
  \item
    Direct oral evidence of a fact is required for admissibility.
  \item
    A statement in a document is admissible if:

    \begin{itemize}
    \tightlist
    \item
      The maker had personal knowledge and is called as a witness.
    \item
      The document is part of a record relating to an undertaking.
    \end{itemize}
  \item
    If the maker is:

    \begin{itemize}
    \tightlist
    \item
      Dead or unfit to attend.
    \item
      Out of the state or cannot be found.
    \item
      Lacking recollection of the information.
    \item
      No party requires cross-examination of the maker.
    \end{itemize}
  \end{itemize}
\end{itemize}

\subsubsection{Special Provisions}\label{special-provisions}

\begin{itemize}
\tightlist
\item
  \textbf{Section 93}: \textbf{Admissibility of Documentary Evidence in
  Criminal Cases}

  \begin{itemize}
  \tightlist
  \item
    Similar to Section 92 but applies to criminal proceedings.
  \item
    A document is admissible if:

    \begin{itemize}
    \tightlist
    \item
      It forms part of a record relating to a trade or business.
    \item
      The supplier of the information is dead or unfit to attend.
    \end{itemize}
  \end{itemize}
\item
  \textbf{Section 93A}: \textbf{Statements by Children or Mentally
  Impaired Persons}

  \begin{itemize}
  \tightlist
  \item
    Statements made by a child or mentally impaired person are
    admissible if:

    \begin{itemize}
    \tightlist
    \item
      The maker had personal knowledge.
    \item
      The maker is available to give evidence.
    \end{itemize}
  \item
    Related statements are also admissible if the maker of the related
    statement is available.
  \end{itemize}
\item
  \textbf{Section 93B}: \textbf{Admissibility of Representation in
  Prescribed Criminal Proceedings}

  \begin{itemize}
  \tightlist
  \item
    Applies when a person with personal knowledge is unavailable.
  \item
    Hearsay is admissible if:

    \begin{itemize}
    \tightlist
    \item
      Made shortly after the event and unlikely to be fabricated.
    \item
      Highly probable that representation is reliable.
    \item
      Adverse to the interest of the maker.
    \end{itemize}
  \end{itemize}
\end{itemize}

\subsubsection{Key Definitions}\label{key-definitions-1}

\begin{longtable}[]{@{}
  >{\raggedright\arraybackslash}p{(\linewidth - 2\tabcolsep) * \real{0.1683}}
  >{\raggedright\arraybackslash}p{(\linewidth - 2\tabcolsep) * \real{0.8317}}@{}}
\toprule\noalign{}
\begin{minipage}[b]{\linewidth}\raggedright
Term
\end{minipage} & \begin{minipage}[b]{\linewidth}\raggedright
Definition
\end{minipage} \\
\midrule\noalign{}
\endhead
\bottomrule\noalign{}
\endlastfoot
Undertaking & Includes any business or profession, whether for profit or
not, by the Crown or statutory body, in Queensland or elsewhere. \\
Personal Knowledge & Refers to the firsthand knowledge the maker has
about the matters dealt with in their statement. \\
Unavailability & Occurs when the maker cannot give evidence due to being
dead, unfit, out of the state, or lacking recollection. \\
\end{longtable}

\subsubsection{Case Example}\label{case-example}

\begin{itemize}
\tightlist
\item
  \textbf{Sio v The Queen}:

  \begin{itemize}
  \tightlist
  \item
    The court ruled that refusing to give evidence due to the right to
    silence or self-incrimination does not prevent the admissibility of
    evidence from the dead or incapable speaker.
  \end{itemize}
\end{itemize}

\subsubsection{Summary of Conditions for
Admissibility}\label{summary-of-conditions-for-admissibility}

\begin{itemize}
\tightlist
\item
  \textbf{For Civil Cases} (Section 92):

  \begin{itemize}
  \tightlist
  \item
    Maker must have personal knowledge.
  \item
    Admissibility conditions vary if the maker is unavailable.
  \end{itemize}
\item
  \textbf{For Criminal Cases} (Section 93):

  \begin{itemize}
  \tightlist
  \item
    Similar conditions apply as in civil cases, with specific emphasis
    on business records.
  \end{itemize}
\item
  \textbf{For Statements by Children/Mentally Impaired} (Section 93A):

  \begin{itemize}
  \tightlist
  \item
    Must show personal knowledge and availability of the maker.
  \end{itemize}
\item
  \textbf{For Prescribed Criminal Proceedings} (Section 93B):

  \begin{itemize}
  \tightlist
  \item
    Hearsay is permissible under specific circumstances regarding the
    maker's availability.
  \end{itemize}
\end{itemize}

\subsection{  Criminal Proceedings}\label{criminal-proceedings}

\subsubsection{Definition of Criminal
Proceedings}\label{definition-of-criminal-proceedings}

\begin{quote}
``A criminal proceeding means a criminal proceeding against a person for
an offence defined in the Criminal Code, chapters 28 to 32.''
\end{quote}

\subsubsection{Representation}\label{representation}

The term \textbf{representation} includes various forms:

\begin{longtable}[]{@{}
  >{\raggedright\arraybackslash}p{(\linewidth - 2\tabcolsep) * \real{0.3088}}
  >{\raggedright\arraybackslash}p{(\linewidth - 2\tabcolsep) * \real{0.6912}}@{}}
\toprule\noalign{}
\begin{minipage}[b]{\linewidth}\raggedright
Type of Representation
\end{minipage} & \begin{minipage}[b]{\linewidth}\raggedright
Description
\end{minipage} \\
\midrule\noalign{}
\endhead
\bottomrule\noalign{}
\endlastfoot
\textbf{Express or Implied} & Can be oral or written. \\
\textbf{Conduct Inference} & A representation can be inferred from
conduct. \\
\textbf{Unintended Communication} & A representation not intended to be
communicated to or seen by another person. \\
\textbf{Non-communicated Representation} & A representation that is not
communicated for any reason. \\
\end{longtable}

\subsection{  Legal References}\label{legal-references-2}

\subsubsection{R v SCJ; Ex parte A-G (Qld) {[}2015{]} QCA
123}\label{r-v-scj-ex-parte-a-g-qld-2015-qca-123}

On January 23, 2015, the Attorney-General referred significant points of
law regarding the competency of child witnesses in court:

\begin{enumerate}
\def\labelenumi{\arabic{enumi}.}
\tightlist
\item
  \textbf{Question (a)}: Does a finding that a child witness is not
  competent to give evidence preclude the admission of an earlier out of
  court statement under \(s93A\) EAQ?

  \begin{itemize}
  \tightlist
  \item
    \textbf{Answer}: Yes.
  \end{itemize}
\item
  \textbf{Question (b)}: Does a finding that a child witness is not
  competent preclude the admission of earlier out of court
  representations under \(s93B\) EAQ?

  \begin{itemize}
  \tightlist
  \item
    \textbf{Answer}: No.
  \end{itemize}
\end{enumerate}

\subsubsection{Section 93A and 93B
Distinction}\label{section-93a-and-93b-distinction}

\begin{itemize}
\item
  \textbf{Section 93A} allows for the admission of earlier out of court
  statements if the maker is available to give evidence.
\item
  \textbf{Section 93B} requires the maker to be unavailable due to
  reasons such as death or incapacity but allows for earlier
  representations to be admitted, making it applicable in prescribed
  criminal proceedings.
\end{itemize}

\begin{longtable}[]{@{}
  >{\raggedright\arraybackslash}p{(\linewidth - 4\tabcolsep) * \real{0.1657}}
  >{\raggedright\arraybackslash}p{(\linewidth - 4\tabcolsep) * \real{0.5455}}
  >{\raggedright\arraybackslash}p{(\linewidth - 4\tabcolsep) * \real{0.2888}}@{}}
\toprule\noalign{}
\begin{minipage}[b]{\linewidth}\raggedright
Section
\end{minipage} & \begin{minipage}[b]{\linewidth}\raggedright
Requirements
\end{minipage} & \begin{minipage}[b]{\linewidth}\raggedright
Application
\end{minipage} \\
\midrule\noalign{}
\endhead
\bottomrule\noalign{}
\endlastfoot
\(s93A\) & Maker must be available to give evidence. & General
application in any proceeding. \\
\(s93B\) & Maker must be unavailable (e.g., deceased, incapacitated). &
Applies only to prescribed criminal proceedings. \\
\end{longtable}

\subsection{   Case Law Examples}\label{case-law-examples}

\subsubsection{R v Crump {[}2004{]} QCA
176}\label{r-v-crump-2004-qca-176}

\begin{itemize}
\tightlist
\item
  \textbf{Shortly after event}: Defined as broader than common law res
  gestae.
\item
  \textbf{Statement Reliability}: If a statement is made shortly after
  an event, it is considered less likely to be fabricated.
\end{itemize}

\subsubsection{R v Lester {[}2004{]} QCA
34}\label{r-v-lester-2004-qca-34}

\begin{itemize}
\tightlist
\item
  \textbf{Relationship Evidence}:

  \begin{itemize}
  \tightlist
  \item
    Three witnesses provided statements regarding the victim's fear of
    her husband.
  \item
    Victim's state of mind relevant under \(s93B\).
  \end{itemize}
\end{itemize}

\begin{longtable}[]{@{}
  >{\raggedright\arraybackslash}p{(\linewidth - 4\tabcolsep) * \real{0.2946}}
  >{\raggedright\arraybackslash}p{(\linewidth - 4\tabcolsep) * \real{0.3334}}
  >{\raggedright\arraybackslash}p{(\linewidth - 4\tabcolsep) * \real{0.3720}}@{}}
\toprule\noalign{}
\begin{minipage}[b]{\linewidth}\raggedright
Statement
\end{minipage} & \begin{minipage}[b]{\linewidth}\raggedright
Admissibility
\end{minipage} & \begin{minipage}[b]{\linewidth}\raggedright
Remarks
\end{minipage} \\
\midrule\noalign{}
\endhead
\bottomrule\noalign{}
\endlastfoot
``I'm scared of my husband'' & Admitted under \(s93B\) & Relevant to
relationship and admissible. \\
``He hired a hitman'' & Not admissible under \(s93B\) & Lacked personal
knowledge of the fact. \\
\end{longtable}

\subsubsection{R v Robertson and Ors {[}2015{]} QCA
11}\label{r-v-robertson-and-ors-2015-qca-11}

\begin{itemize}
\tightlist
\item
  Distinguished from \textbf{Lester} due to the relevance of the
  deceased's relationship with the accused, which negated suicide and
  examined the acts complained of.
\end{itemize}

\subsection{  Procedural Issues Related to Section
93B}\label{procedural-issues-related-to-section-93b}

\subsubsection{Jury Warnings and
Discretions}\label{jury-warnings-and-discretions}

\begin{itemize}
\tightlist
\item
  \(s93C\): If evidence is admitted under \(s93B\), the jury must be
  warned about potential unreliability.
\end{itemize}

\begin{longtable}[]{@{}
  >{\raggedright\arraybackslash}p{(\linewidth - 2\tabcolsep) * \real{0.2769}}
  >{\raggedright\arraybackslash}p{(\linewidth - 2\tabcolsep) * \real{0.7231}}@{}}
\toprule\noalign{}
\begin{minipage}[b]{\linewidth}\raggedright
Jury Responsibilities
\end{minipage} & \begin{minipage}[b]{\linewidth}\raggedright
Description
\end{minipage} \\
\midrule\noalign{}
\endhead
\bottomrule\noalign{}
\endlastfoot
\textbf{Warn about unreliability} & Jury must be informed that hearsay
evidence may be unreliable. \\
\textbf{Caution needed} & Jurors should exercise caution when accepting
hearsay evidence and consider its weight. \\
\end{longtable}

\subsubsection{Weight of Evidence}\label{weight-of-evidence}

\begin{itemize}
\tightlist
\item
  \(s102\): The jury estimates the weight of hearsay evidence based on
  the circumstances of the statement's making.
\end{itemize}

\subsubsection{Rejection of Evidence}\label{rejection-of-evidence}

\begin{itemize}
\tightlist
\item
  \(s98\): The court may reject admissible statements if it is deemed
  inexpedient in the interests of justice.
\end{itemize}

\subsubsection{Admissibility of Statements in
Documents}\label{admissibility-of-statements-in-documents}

\begin{itemize}
\tightlist
\item
  \(s95\): Statements in documents produced by processes or devices are
  admissible in proceedings where direct oral evidence would be.
\end{itemize}

\subsection{ Admissibility of
Evidence}\label{admissibility-of-evidence-3}

\subsubsection{Evidence Produced by Processes or
Devices}\label{evidence-produced-by-processes-or-devices}

\begin{itemize}
\tightlist
\item
  \textbf{Subsection (2)(a)}: Evidence that a document or thing was
  produced wholly or partly by a specific process or device.
\item
  \textbf{Subsection (2)(b)}: Evidence that the document or thing was
  produced in a particular way by the process or device.
\item
  \textbf{Subsection (2)(c)}: If properly used, the process or device
  produces documents or things of a particular kind.
\item
  \textbf{Subsection (2)(d)}: Any particulars relevant to matters
  mentioned in paragraphs (a), (b), or (c).
\end{itemize}

\subsubsection{Computer Records
Definition}\label{computer-records-definition}

\begin{quote}
``The effect of s 95 EA 1977 (Qld) extends the definition to include
records generated by a computer.''
\end{quote}

\begin{itemize}
\tightlist
\item
  This definition applies not only to computer records but also to their
  admissibility if oral evidence would be admissible.
\end{itemize}

\subsubsection{Certificate and Offence}\label{certificate-and-offence}

\begin{itemize}
\tightlist
\item
  A person who signs a certificate under subsection (3) commits an
  offence if:

  \begin{itemize}
  \tightlist
  \item
    \begin{enumerate}
    \def\labelenumi{(\alph{enumi})}
    \tightlist
    \item
      The certificate contains a matter that is known or ought
      reasonably to be known as false.
    \end{enumerate}
  \item
    \begin{enumerate}
    \def\labelenumi{(\alph{enumi})}
    \setcounter{enumi}{1}
    \tightlist
    \item
      The statement is material in the proceeding.
    \end{enumerate}
  \end{itemize}
\end{itemize}

\begin{quote}
\textbf{Maximum penalty}: 20 penalty units or 1 year's imprisonment.
\end{quote}

\subsubsection{Notification
Requirements}\label{notification-requirements}

\begin{itemize}
\tightlist
\item
  If a party intends to rely on the certificate, they must provide a
  copy to each other party:

  \begin{itemize}
  \tightlist
  \item
    At least \textbf{10 business days} before the hearing day, or
  \item
    A later date allowed by the court.
  \end{itemize}
\item
  If a party intends to challenge the certificate:

  \begin{itemize}
  \tightlist
  \item
    They must give written notice to the relying party at least
    \textbf{3 business days} before the hearing day, or
  \item
    A later date allowed by the court.
  \end{itemize}
\end{itemize}

\subsubsection{Definitions}\label{definitions-1}

\begin{itemize}
\tightlist
\item
  \textbf{Responsible Person}: A person responsible for the operation or
  management of the process or device that produces a document.
\end{itemize}

\subsection{  Hearsay}\label{hearsay}

\subsubsection{Definition of Hearsay}\label{definition-of-hearsay}

\begin{quote}
Hearsay is relying on the truth of an out-of-court statement to prove
the facts mentioned in that statement.
\end{quote}

\begin{itemize}
\tightlist
\item
  An assertion made outside of court by a non-witness is generally
  inadmissible as evidence.
\end{itemize}

\subsubsection{Key Cases}\label{key-cases-1}

\begin{itemize}
\tightlist
\item
  \textbf{Subramaniam v Public Prosecutor {[}1956{]}}:

  \begin{itemize}
  \tightlist
  \item
    Evidence of a statement made to a witness by a person not called as
    a witness may or may not be hearsay.
  \item
    It is hearsay if the purpose is to establish the truth of the
    statement.
  \end{itemize}
\end{itemize}

\subsubsection{Elements of Hearsay}\label{elements-of-hearsay}

\begin{enumerate}
\def\labelenumi{\arabic{enumi}.}
\tightlist
\item
  The statement must be originally communicated outside of court.
\item
  The purpose of seeking its admission is to prove the truth of what was
  communicated.
\end{enumerate}

\subsubsection{Types of Hearsay
Statements}\label{types-of-hearsay-statements}

\begin{longtable}[]{@{}
  >{\raggedright\arraybackslash}p{(\linewidth - 2\tabcolsep) * \real{0.2892}}
  >{\raggedright\arraybackslash}p{(\linewidth - 2\tabcolsep) * \real{0.7108}}@{}}
\toprule\noalign{}
\begin{minipage}[b]{\linewidth}\raggedright
Type
\end{minipage} & \begin{minipage}[b]{\linewidth}\raggedright
Description
\end{minipage} \\
\midrule\noalign{}
\endhead
\bottomrule\noalign{}
\endlastfoot
Verbal Representations & Testimony about verbal statements of others. \\
Written Statements & Documents that contain statements made outside of
court. \\
\end{longtable}

\subsubsection{Key Cases on Hearsay}\label{key-cases-on-hearsay}

\begin{itemize}
\tightlist
\item
  \textbf{Myers v DPP {[}1965{]}}: Microfilm records were deemed hearsay
  because the employee could not testify to their correctness.
\item
  \textbf{Re Gardiner {[}1967{]}}: A used airline ticket was considered
  hearsay as it was being used to prove the truth of its contents.
\end{itemize}

\subsubsection{Implied Representations}\label{implied-representations}

\begin{itemize}
\tightlist
\item
  Under common law, the hearsay rule may apply to statements that are
  impliedly assertive:

  \begin{itemize}
  \tightlist
  \item
    \textbf{Walton v R {[}1989{]}}: Statements made can imply danger and
    need for police assistance.
  \item
    \textbf{Chandrasekera v R {[}1937{]}}: Conduct that implies
    agreement can fall under hearsay.
  \end{itemize}
\end{itemize}

\subsection{  Proving the Truth of Words
Said}\label{proving-the-truth-of-words-said}

\begin{itemize}
\tightlist
\item
  It is crucial to assess the relevance of the representation when
  considering hearsay:

  \begin{itemize}
  \tightlist
  \item
    What are the facts in issue?
  \item
    How does the out-of-court statement relate to these facts?
  \end{itemize}
\end{itemize}

\subsubsection{Important Cases}\label{important-cases-3}

\begin{itemize}
\tightlist
\item
  \textbf{R v Benz and Murray {[}1989{]}}: The court held that
  statements made in the context of a lie were considered hearsay.
\item
  Hearsay is only inadmissible if it's used to establish the truth of
  what was said. If the statement is relevant for another purpose, it
  can be admissible.
\end{itemize}

\subsubsection{Reasons Against Hearsay
Evidence}\label{reasons-against-hearsay-evidence}

\begin{enumerate}
\def\labelenumi{\arabic{enumi}.}
\tightlist
\item
  It is not the best evidence available.
\item
  It is not delivered on oath.
\item
  Truthfulness and accuracy cannot be tested by cross-examination.
\item
  The demeanor of the maker of the statement cannot be assessed.
\item
  Hearsay is easily fabricated and difficult to disprove.
\end{enumerate}

\subsubsection{Consideration for
Hearsay}\label{consideration-for-hearsay}

\begin{quote}
``Hearsay is not the best evidence available; it is better to get a
witness to testify directly in court.''
\end{quote}

\subsection{  Hearsay vs.~Original
Evidence}\label{hearsay-vs.-original-evidence}

\subsubsection{Definition of Original
Evidence}\label{definition-of-original-evidence}

\begin{quote}
``Original evidence refers to an out-of-court statement admitted for a
purpose other than to prove the truth of the contents of the
statement.''
\end{quote}

\subsubsection{Key Concepts}\label{key-concepts-2}

\begin{itemize}
\tightlist
\item
  \textbf{Hearsay}: A statement made outside of court that is offered to
  prove the truth of the matter asserted.
\item
  \textbf{Original Evidence}: Evidence that is not considered hearsay
  because it is not relied upon to establish the truth of the statement.
\end{itemize}

\subsubsection{Legal Precedents}\label{legal-precedents-1}

\begin{itemize}
\tightlist
\item
  \textbf{Pollitt v R (1991-92)}: A case illustrating the distinction
  between hearsay and original evidence.
\item
  \textbf{Ratten v R (1972)}: Lord Wilberforce stated that the speaking
  of words is a relevant fact and may be admissible unless relied upon
  for testimonial purposes.
\end{itemize}

\begin{longtable}[]{@{}
  >{\raggedright\arraybackslash}p{(\linewidth - 2\tabcolsep) * \real{0.2791}}
  >{\raggedright\arraybackslash}p{(\linewidth - 2\tabcolsep) * \real{0.7209}}@{}}
\toprule\noalign{}
\begin{minipage}[b]{\linewidth}\raggedright
Case
\end{minipage} & \begin{minipage}[b]{\linewidth}\raggedright
Ruling
\end{minipage} \\
\midrule\noalign{}
\endhead
\bottomrule\noalign{}
\endlastfoot
Pollitt v R & Established distinction between hearsay and original
evidence \\
Ratten v R & Speaking words is a fact and can be admitted as relevant \\
\end{longtable}

\subsubsection{Categories of Original
Evidence}\label{categories-of-original-evidence}

\begin{enumerate}
\def\labelenumi{\arabic{enumi}.}
\tightlist
\item
  \textbf{State of Mind}: Evidence that establishes someone's state of
  mind or intention.
\item
  \textbf{Relationship Evidence}: Evidence demonstrating the nature of a
  relationship.
\end{enumerate}

\subsubsection{Relevance of State of
Mind}\label{relevance-of-state-of-mind}

\begin{itemize}
\tightlist
\item
  \textbf{State of Mind}: Must be relevant to the facts in issue.
\item
  \textbf{Impacts}:

  \begin{itemize}
  \tightlist
  \item
    Likely impact on the hearer's state of mind
  \item
    Someone's state of knowledge
  \item
    Maker's intention to act in the future
  \item
    Nature of relationships
  \end{itemize}
\end{itemize}

\subsubsection{Case Examples}\label{case-examples-1}

\paragraph{1. Subramaniam v Public Prosecutor
(1956)}\label{subramaniam-v-public-prosecutor-1956}

\begin{itemize}
\tightlist
\item
  \textbf{Context}: Evidence of terrorist threats to a hostage was
  deemed original evidence.
\item
  \textbf{Significance}: The court held that the state of mind was
  relevant because the subjective belief was crucial in the defense of
  duress.
\end{itemize}

\begin{longtable}[]{@{}
  >{\raggedright\arraybackslash}p{(\linewidth - 4\tabcolsep) * \real{0.1579}}
  >{\raggedright\arraybackslash}p{(\linewidth - 4\tabcolsep) * \real{0.4342}}
  >{\raggedright\arraybackslash}p{(\linewidth - 4\tabcolsep) * \real{0.4079}}@{}}
\toprule\noalign{}
\begin{minipage}[b]{\linewidth}\raggedright
Case
\end{minipage} & \begin{minipage}[b]{\linewidth}\raggedright
Facts
\end{minipage} & \begin{minipage}[b]{\linewidth}\raggedright
Ruling
\end{minipage} \\
\midrule\noalign{}
\endhead
\bottomrule\noalign{}
\endlastfoot
Subramaniam v Public Prosecutor & Hostage under duress due to terrorist
threats. & Evidence was admissible to prove state of mind, not
hearsay. \\
\end{longtable}

\paragraph{2. Kamleh v R (2005)}\label{kamleh-v-r-2005}

\begin{itemize}
\tightlist
\item
  \textbf{Context}: Statements used to show knowledge of significant
  past events or future intentions.
\item
  \textbf{Example}: A statement that the accused was aware of a
  significant event.
\end{itemize}

\begin{longtable}[]{@{}
  >{\raggedright\arraybackslash}p{(\linewidth - 2\tabcolsep) * \real{0.2727}}
  >{\raggedright\arraybackslash}p{(\linewidth - 2\tabcolsep) * \real{0.7273}}@{}}
\toprule\noalign{}
\begin{minipage}[b]{\linewidth}\raggedright
Case
\end{minipage} & \begin{minipage}[b]{\linewidth}\raggedright
Relevance of Statements
\end{minipage} \\
\midrule\noalign{}
\endhead
\bottomrule\noalign{}
\endlastfoot
Kamleh v R & Statements can show knowledge of past events or future
intentions. \\
\end{longtable}

\paragraph{3. Hytch v R (2000)}\label{hytch-v-r-2000}

\begin{itemize}
\tightlist
\item
  \textbf{Context}: Deceased's statements of intention relevant for
  inferring future actions.
\item
  \textbf{Significance}: Statements about intentions can be
  circumstantial evidence.
\end{itemize}

\begin{longtable}[]{@{}
  >{\raggedright\arraybackslash}p{(\linewidth - 4\tabcolsep) * \real{0.1579}}
  >{\raggedright\arraybackslash}p{(\linewidth - 4\tabcolsep) * \real{0.4342}}
  >{\raggedright\arraybackslash}p{(\linewidth - 4\tabcolsep) * \real{0.4079}}@{}}
\toprule\noalign{}
\begin{minipage}[b]{\linewidth}\raggedright
Case
\end{minipage} & \begin{minipage}[b]{\linewidth}\raggedright
Facts
\end{minipage} & \begin{minipage}[b]{\linewidth}\raggedright
Ruling
\end{minipage} \\
\midrule\noalign{}
\endhead
\bottomrule\noalign{}
\endlastfoot
Hytch v R & Deceased planned to meet the accused, relevant to motive. &
Statements admitted as circumstantial evidence of conduct. \\
\end{longtable}

\subsubsection{Admissibility of Intention
Statements}\label{admissibility-of-intention-statements}

\begin{itemize}
\tightlist
\item
  Statements regarding intentions are admissible to infer conduct.
\item
  \textbf{Walton v R (1989)}: The deceased's intention to meet her
  estranged husband was admissible as evidence of her state of mind.
\end{itemize}

\begin{longtable}[]{@{}
  >{\raggedright\arraybackslash}p{(\linewidth - 2\tabcolsep) * \real{0.2758}}
  >{\raggedright\arraybackslash}p{(\linewidth - 2\tabcolsep) * \real{0.7242}}@{}}
\toprule\noalign{}
\begin{minipage}[b]{\linewidth}\raggedright
Case
\end{minipage} & \begin{minipage}[b]{\linewidth}\raggedright
Importance of Statements
\end{minipage} \\
\midrule\noalign{}
\endhead
\bottomrule\noalign{}
\endlastfoot
Walton v R & Statements of intention can infer actions and state of
mind. \\
\end{longtable}

\subsubsection{Key Takeaways}\label{key-takeaways}

\begin{itemize}
\tightlist
\item
  The distinction between hearsay and original evidence hinges on the
  purpose for which a statement is offered.
\item
  Original evidence can serve to establish state of mind, knowledge, or
  intentions without being classified as hearsay.
\item
  Relevant case law demonstrates the nuanced applications of these
  concepts in legal contexts.
\end{itemize}

\subsection{  Hearsay Rule and
Exceptions}\label{hearsay-rule-and-exceptions}

\subsubsection{Overview of Hearsay Rule}\label{overview-of-hearsay-rule}

\begin{quote}
``The hearsay rule prohibits the use of out-of-court statements to prove
the truth of the matter asserted.''
\end{quote}

\begin{itemize}
\tightlist
\item
  \textbf{Hearsay Evidence}: Generally inadmissible unless it falls
  within a recognized exception.
\item
  \textbf{Judicial Discretion}: Judges may exclude evidence rigidly
  based on hearsay, but a flexible approach may be warranted in some
  cases.
\end{itemize}

\subsubsection{Flexible Approach to
Hearsay}\label{flexible-approach-to-hearsay}

\begin{itemize}
\tightlist
\item
  \textbf{Deane J's Stance}: Advocated for a flexible application of the
  hearsay rule where strict application might result in injustice.
\item
  \textbf{Telephone Exception}: Proposed by Deane J to allow
  contemporaneous statements identifying the party in a phone call,
  which has not been widely accepted in later cases.
\end{itemize}

\subsubsection{Case Law Illustrations}\label{case-law-illustrations}

\begin{longtable}[]{@{}
  >{\raggedright\arraybackslash}p{(\linewidth - 2\tabcolsep) * \real{0.1965}}
  >{\raggedright\arraybackslash}p{(\linewidth - 2\tabcolsep) * \real{0.8035}}@{}}
\toprule\noalign{}
\begin{minipage}[b]{\linewidth}\raggedright
Case
\end{minipage} & \begin{minipage}[b]{\linewidth}\raggedright
Key Points
\end{minipage} \\
\midrule\noalign{}
\endhead
\bottomrule\noalign{}
\endlastfoot
R v Firman (1989) & Telephone calls to the accused's premises about drug
purchases were evidence of the fact calls were made, not proof of actual
wrongdoing. \\
R v Perry {[}2011{]} & Tapped phone calls showed intent, but were not
used to prove the truth of the content discussed. \\
Walton v R (1989) & Majority judgment distinguished hearsay based on
whether statements were assertions or relevant inferences. \\
\end{longtable}

\subsection{  Telephone Exception}\label{telephone-exception}

\subsubsection{Proposed by Deane J}\label{proposed-by-deane-j}

\begin{itemize}
\tightlist
\item
  \textbf{Nature of Exception}: Relevant for identifying the caller
  based on implied assertions during overheard conversations.
\item
  \textbf{Limitations}: Not widely adopted as a standard exception in
  case law.
\end{itemize}

\subsection{  Majority vs.~Minority
Judgments}\label{majority-vs.-minority-judgments}

\subsubsection{Key Judicial Opinions}\label{key-judicial-opinions}

\begin{longtable}[]{@{}
  >{\raggedright\arraybackslash}p{(\linewidth - 4\tabcolsep) * \real{0.1441}}
  >{\raggedright\arraybackslash}p{(\linewidth - 4\tabcolsep) * \real{0.4685}}
  >{\raggedright\arraybackslash}p{(\linewidth - 4\tabcolsep) * \real{0.3874}}@{}}
\toprule\noalign{}
\begin{minipage}[b]{\linewidth}\raggedright
Justice
\end{minipage} & \begin{minipage}[b]{\linewidth}\raggedright
Stance on Hearsay Rule
\end{minipage} & \begin{minipage}[b]{\linewidth}\raggedright
Remarks
\end{minipage} \\
\midrule\noalign{}
\endhead
\bottomrule\noalign{}
\endlastfoot
Mason CJ & Invent exception & Advocated for a telephone exception. \\
Brennan J & Inadmissible & Rejected the flexible approach. \\
Deane J & Invent exception & Supported flexibility in hearsay cases. \\
Dawson \& Gaudron JJ & Circumstantial fact & Questioned the need for a
telephone exception. \\
McHugh J & Invent exception & Did not support the Canadian approach. \\
\end{longtable}

\subsection{  Relationship Evidence}\label{relationship-evidence}

\subsubsection{Admissibility Criteria}\label{admissibility-criteria-1}

\begin{itemize}
\tightlist
\item
  \textbf{Nature of Relationship}: Relevant evidence from relationships
  can be admissible to establish context, not merely hearsay.
\item
  \textbf{Judicial Perspective}: Evidence of past conduct or statements
  can illuminate relationships and motives.
\end{itemize}

\subsubsection{Case Law Examples}\label{case-law-examples-1}

\begin{longtable}[]{@{}
  >{\raggedright\arraybackslash}p{(\linewidth - 2\tabcolsep) * \real{0.1869}}
  >{\raggedright\arraybackslash}p{(\linewidth - 2\tabcolsep) * \real{0.8131}}@{}}
\toprule\noalign{}
\begin{minipage}[b]{\linewidth}\raggedright
Case
\end{minipage} & \begin{minipage}[b]{\linewidth}\raggedright
Key Findings
\end{minipage} \\
\midrule\noalign{}
\endhead
\bottomrule\noalign{}
\endlastfoot
Wilson v R (1970) & Admission of statements about the accused's
intentions towards the deceased, relevant to relationship context. \\
R v Clark (2001) & Relationship evidence is relevant even if not made in
the accused's presence; it informs the jury about potential motives. \\
R v Frawley (1993) & Relationship evidence must be unpacked for
relevance; not all statements are automatically admissible. \\
\end{longtable}

\subsection{ Legislative Context}\label{legislative-context-1}

\subsubsection{Section 132B EAQ - Evidence of Domestic
Violence}\label{section-132b-eaq---evidence-of-domestic-violence}

\begin{enumerate}
\def\labelenumi{\arabic{enumi}.}
\tightlist
\item
  \textbf{Application}: Relevant evidence of domestic relationship
  history is admissible in criminal proceedings for specific offenses.
\item
  \textbf{Definition}: Domestic relationship includes intimate personal,
  family, or informal care relationships as per the Domestic and Family
  Violence Protection Act 2012.
\end{enumerate}

\begin{itemize}
\tightlist
\item
  \textbf{Key Inquiry}: What type of evidence is prescribed as
  admissible based on the relationship's nature?
\end{itemize}

This study guide encapsulates the intricate discussions surrounding the
hearsay rule, its exceptions, and the admissibility of relationship
evidence as discussed in the lecture.

\subsection{  Evidence in Domestic
Relationships}\label{evidence-in-domestic-relationships}

\subsubsection{Relationship Evidence vs.~Similar Fact
Evidence}\label{relationship-evidence-vs.-similar-fact-evidence}

\begin{itemize}
\tightlist
\item
  \textbf{Relationship Evidence}: Evidence that pertains specifically to
  the nature of a domestic relationship.
\item
  \textbf{Similar Fact Evidence (SFE)}: Evidence that relates to past
  misconduct, which can sometimes overlap with relationship evidence.
\end{itemize}

\begin{quote}
``Sometimes evidence might be able to be characterised both as
relationship evidence and as similar fact evidence.''
\end{quote}

\subsubsection{Roach v R {[}2011{]} HCA 12}\label{roach-v-r-2011-hca-12}

\begin{itemize}
\tightlist
\item
  \textbf{Significance}: This case addresses the admissibility of
  similar fact evidence in the context of domestic relationships.
\item
  \textbf{Facts}: The case involved Roach and his ex-partner, who had a
  history of abuse. Roach's defense argued that evidence of past conduct
  was prejudicial and should not be admitted.
\end{itemize}

\begin{longtable}[]{@{}
  >{\raggedright\arraybackslash}p{(\linewidth - 2\tabcolsep) * \real{0.3417}}
  >{\raggedright\arraybackslash}p{(\linewidth - 2\tabcolsep) * \real{0.6583}}@{}}
\toprule\noalign{}
\begin{minipage}[b]{\linewidth}\raggedright
Aspect
\end{minipage} & \begin{minipage}[b]{\linewidth}\raggedright
Detail
\end{minipage} \\
\midrule\noalign{}
\endhead
\bottomrule\noalign{}
\endlastfoot
Case Name & Roach v R {[}2011{]} HCA 12 \\
Key Issue & Admissibility of SFE and relationship evidence \\
Context & Relationship marked by past abuse \\
\end{longtable}

\subsection{  Res Gestae}\label{res-gestae}

\subsubsection{Definition}\label{definition-2}

\begin{itemize}
\tightlist
\item
  \textbf{Res Gestae}: Refers to the events or transactions surrounding
  the crime, essential for the jury to understand the context of the
  crime.
\end{itemize}

\begin{quote}
``An evidential fact will become relevant and admissible if it forms
part of the res gestae.''
\end{quote}

\subsubsection{Requirements for Res
Gestae}\label{requirements-for-res-gestae}

\begin{enumerate}
\def\labelenumi{\arabic{enumi}.}
\tightlist
\item
  \textbf{Contemporaneity}: The statement must occur at or close to the
  time of the event.
\item
  \textbf{Spontaneity}: The statement must be made spontaneously,
  without reflection or chance of concoction.
\end{enumerate}

\begin{longtable}[]{@{}
  >{\raggedright\arraybackslash}p{(\linewidth - 2\tabcolsep) * \real{0.2143}}
  >{\raggedright\arraybackslash}p{(\linewidth - 2\tabcolsep) * \real{0.7857}}@{}}
\toprule\noalign{}
\begin{minipage}[b]{\linewidth}\raggedright
Requirement
\end{minipage} & \begin{minipage}[b]{\linewidth}\raggedright
Explanation
\end{minipage} \\
\midrule\noalign{}
\endhead
\bottomrule\noalign{}
\endlastfoot
Contemporaneity & Allows evidence of events close in time to the
disputed event. \\
Spontaneity & Ensures reliability due to the circumstances under which
the statement is made. \\
\end{longtable}

\subsubsection{Leading Cases}\label{leading-cases}

\begin{itemize}
\tightlist
\item
  \textbf{O'Leary v R (1946) 73 CLR 566}: Established the need for the
  jury to understand the broader context of the event.
\item
  \textbf{Adelaide Chemical and Fertilizer Co Ltd v Carlyle (1940) 64
  CLR 514}: Discussed almost exact contemporaneity.
\item
  \textbf{Ratten v R {[}1972{]} AC 378}: Relaxed the technical
  requirements for admissibility, emphasizing the importance of
  excluding possible concoction.
\end{itemize}

\subsubsection{Ratten v R {[}1972{]} AC
378}\label{ratten-v-r-1972-ac-378}

\begin{itemize}
\tightlist
\item
  \textbf{Facts}: A phone call made 8 minutes before a shooting was
  deemed admissible because it was spontaneous and contemporaneous
  enough.
\item
  \textbf{Judgment}: Strict contemporaneity is not required; approximate
  timing is acceptable as long as concoction can be excluded.
\end{itemize}

\subsection{  Declarations by Deceased
Persons}\label{declarations-by-deceased-persons}

\subsubsection{Justification for
Admissibility}\label{justification-for-admissibility}

\begin{itemize}
\tightlist
\item
  Statements made by deceased individuals can be admitted as evidence
  because they are often seen as the best form of evidence available.
\end{itemize}

\subsubsection{Categories of Statements}\label{categories-of-statements}

\begin{enumerate}
\def\labelenumi{\arabic{enumi}.}
\tightlist
\item
  \textbf{Declarations Against Pecuniary or Proprietary Interest}:
  Statements made by the deceased that could harm their interests.
\item
  \textbf{Declarations Made in the Course of Duty}: Official statements
  made during the course of their employment.
\end{enumerate}

\begin{quote}
``The evidence can only be received if it would have been admissible as
testimony, had the declarant been alive to testify.''
\end{quote}

\subsubsection{Limitations}\label{limitations}

\begin{itemize}
\tightlist
\item
  \textbf{Pecuniary Interests}: The deceased must have had first-hand
  knowledge of the subject matter.
\item
  \textbf{Exclusions}: Statements made against penal interest
  (confessions and admissions) are not admissible, as seen in
  \emph{Sussex Peerage Case (1844)}.
\end{itemize}

\begin{longtable}[]{@{}
  >{\raggedright\arraybackslash}p{(\linewidth - 2\tabcolsep) * \real{0.3483}}
  >{\raggedright\arraybackslash}p{(\linewidth - 2\tabcolsep) * \real{0.6517}}@{}}
\toprule\noalign{}
\begin{minipage}[b]{\linewidth}\raggedright
Category
\end{minipage} & \begin{minipage}[b]{\linewidth}\raggedright
Details
\end{minipage} \\
\midrule\noalign{}
\endhead
\bottomrule\noalign{}
\endlastfoot
Declarations Against Interest & Must be based on firsthand knowledge. \\
Declarations in Course of Duty & Must arise from official duties related
to employment. \\
\end{longtable}

\subsection{  Dying Declarations}\label{dying-declarations}

\begin{quote}
``Dying declarations are admissible only in cases involving criminal
charges of unlawful homicide of the declarant.''
\end{quote}

\subsubsection{Key Aspects of Dying
Declarations}\label{key-aspects-of-dying-declarations}

\begin{itemize}
\tightlist
\item
  \textbf{Settled Expectation of Death}: The statement must be made when
  the victim has a settled, hopeless expectation of death (R v Peel
  (1860) 175 ER 941).
\item
  \textbf{Purpose of Admission}: The contents of the statement are
  admitted to explain the circumstances leading to death and identify
  the responsible person.
\item
  \textbf{Timing of Belief}: The declarant does not need to die
  immediately; they must believe they will die soon (R v Bernadotti
  (1869)).
\end{itemize}

\subsection{ Declarations Regarding Physical or Mental
Health}\label{declarations-regarding-physical-or-mental-health}

\subsubsection{Admissibility of
Statements}\label{admissibility-of-statements}

\begin{itemize}
\tightlist
\item
  This exception allows statements regarding the maker's
  \textbf{contemporaneous health} or feelings.
\item
  Statements about past health are \textbf{not} admissible under this
  exception (Ramsay v Watson (1961) 108 CLR 642).
\end{itemize}

\subsubsection{Example Case}\label{example-case-2}

\begin{itemize}
\tightlist
\item
  In R v Perry (No 2) (1981) SASR, the Crown could introduce evidence of
  symptoms described by a victim to his doctor in a trial for attempted
  murder by arsenic poisoning.
\end{itemize}

\subsection{ Statements Showing the Nature of a
Business}\label{statements-showing-the-nature-of-a-business}

\subsubsection{Criteria for Admission}\label{criteria-for-admission}

\begin{itemize}
\tightlist
\item
  Statements concerning the nature of a business can be admitted.
\item
  This has been particularly useful in prosecuting SP bookmakers and
  drug traffickers.
\end{itemize}

\begin{longtable}[]{@{}
  >{\raggedright\arraybackslash}p{(\linewidth - 2\tabcolsep) * \real{0.5000}}
  >{\raggedright\arraybackslash}p{(\linewidth - 2\tabcolsep) * \real{0.5000}}@{}}
\toprule\noalign{}
\begin{minipage}[b]{\linewidth}\raggedright
Case
\end{minipage} & \begin{minipage}[b]{\linewidth}\raggedright
Context
\end{minipage} \\
\midrule\noalign{}
\endhead
\bottomrule\noalign{}
\endlastfoot
McGregor v Stokes {[}1952{]} & SP bookmakers \\
Police v Machirus {[}1977{]} & SP bookmakers \\
R v Firman (1989) & Drug traffickers \\
Abrahamson (1994) & Drug traffickers \\
Al Khair BC9402671 & Drug traffickers \\
\end{longtable}

\subsection{  Tripodi Principle}\label{tripodi-principle}

\subsubsection{Overview}\label{overview}

\begin{itemize}
\tightlist
\item
  This principle applies to conspiracies, allowing statements of
  conspirators to be used against each other, even if relying on the
  truth of an out-of-court statement.
\item
  It is often incorrectly referred to as the \textbf{Conspirators Rule}.
\end{itemize}

\subsubsection{Key Cases}\label{key-cases-2}

\begin{longtable}[]{@{}
  >{\raggedright\arraybackslash}p{(\linewidth - 2\tabcolsep) * \real{0.1477}}
  >{\raggedright\arraybackslash}p{(\linewidth - 2\tabcolsep) * \real{0.8523}}@{}}
\toprule\noalign{}
\begin{minipage}[b]{\linewidth}\raggedright
Case
\end{minipage} & \begin{minipage}[b]{\linewidth}\raggedright
Details
\end{minipage} \\
\midrule\noalign{}
\endhead
\bottomrule\noalign{}
\endlastfoot
Tripodi (1961) & Establishes the principle regarding conspirators'
statements. \\
Ahern (1988) & Clarifies that the principle applies to factual
conspiracies regardless of formal charges. \\
\end{longtable}

\subsubsection{Example Scenario}\label{example-scenario-1}

\begin{itemize}
\tightlist
\item
  If A and B conspire to rob a bank and B tells a potential getaway
  driver, X, about the plan, X can testify to B's statement against both
  A and B.
\end{itemize}

\subsection{  Statements Made in Previous
Proceedings}\label{statements-made-in-previous-proceedings}

\subsubsection{Admissibility Criteria}\label{admissibility-criteria-2}

\begin{itemize}
\tightlist
\item
  Sworn evidence from previous proceedings can be admissible if a
  witness cannot attend for valid reasons (e.g., death).
\item
  Governed by statutes such as the Justices Act 1886 (Qld) s111 and EAQ,
  ss 79 \& 80.
\end{itemize}

\subsection{  Informal Admissions}\label{informal-admissions}

\subsubsection{Definition and Rationale}\label{definition-and-rationale}

\begin{itemize}
\tightlist
\item
  Informal admissions are acknowledgments of liability by a party.
\item
  They can be inferred from circumstances, such as failing to stop at a
  car crash (Holloway v McFeeters (1956) CLR).
\end{itemize}

\subsubsection{Key Points}\label{key-points-4}

\begin{itemize}
\tightlist
\item
  Only admissible against the party who made them.
\item
  Can be used as implied admissions in criminal cases where the accused
  has lied (Edwards v R (1993) CLR).
\end{itemize}

\subsection{ Admissions and
Confessions}\label{admissions-and-confessions}

\subsubsection{Key Distinctions}\label{key-distinctions}

\begin{itemize}
\tightlist
\item
  \textbf{Hearsay Rule}: Out-of-court statements are generally
  inadmissible to prove truth.
\item
  \textbf{Exception}: Confessions and admissions are exceptions to this
  rule due to their inherent reliability.
\end{itemize}

\subsubsection{Definitions}\label{definitions-2}

\begin{itemize}
\tightlist
\item
  \textbf{Confession}: A direct admission of guilt, found only in
  criminal cases.
\item
  \textbf{Admission}: Can be partial or inferred from conduct,
  applicable in both civil and criminal cases.
\end{itemize}

\subsubsection{Examples}\label{examples}

\begin{longtable}[]{@{}
  >{\raggedright\arraybackslash}p{(\linewidth - 2\tabcolsep) * \real{0.1215}}
  >{\raggedright\arraybackslash}p{(\linewidth - 2\tabcolsep) * \real{0.8785}}@{}}
\toprule\noalign{}
\begin{minipage}[b]{\linewidth}\raggedright
Type
\end{minipage} & \begin{minipage}[b]{\linewidth}\raggedright
Definition
\end{minipage} \\
\midrule\noalign{}
\endhead
\bottomrule\noalign{}
\endlastfoot
Confession & Direct admission of guilt (e.g., ``I committed the
crime''). \\
Admission & A partial acknowledgment (e.g., ``I was at the crime
scene''). \\
\end{longtable}

\subsubsection{Considerations for
Admissibility}\label{considerations-for-admissibility}

\begin{itemize}
\tightlist
\item
  Admissions must relate to specific offences charged.
\item
  Context and extrinsic material may be considered in determining if a
  statement is an admission.
\end{itemize}

\subsection{  Equivocal Statements}\label{equivocal-statements}

\subsubsection{Definition}\label{definition-3}

\begin{itemize}
\tightlist
\item
  \textbf{Equivocal}: Statements with multiple interpretations, often
  intentionally ambiguous.
\end{itemize}

\subsubsection{Admissibility}\label{admissibility}

\begin{itemize}
\tightlist
\item
  If an admission is equivocal, it may be excluded as irrelevant or
  prejudicial (R v Doolan {[}1962{]} Qd R 449).
\end{itemize}

\subsection{  Adoption of Someone Else's
Statement}\label{adoption-of-someone-elses-statement}

\subsubsection{Criteria for Admission}\label{criteria-for-admission-1}

\begin{itemize}
\tightlist
\item
  An accused can adopt a statement made in their presence, acknowledging
  its truth.
\end{itemize}

\subsubsection{Example}\label{example}

\begin{itemize}
\tightlist
\item
  If someone asserts the accused's guilt and the accused admits it, this
  can be considered an admission.
\end{itemize}

\subsection{  Admissions and Silence}\label{admissions-and-silence}

\subsubsection{General Rule and
Exceptions}\label{general-rule-and-exceptions}

\begin{quote}
``General rule: statements of others cannot constitute an admission by
the accused. Exception: when the accused, either by words or conduct,
adopts another's statement.''
\end{quote}

\paragraph{Case: R v MMJ {[}2006{]} VSCA
226}\label{case-r-v-mmj-2006-vsca-226}

\begin{itemize}
\tightlist
\item
  \textbf{Key Point}: Adoption by silence is allowed when the speaker
  and accused are on equal terms and denial would clearly be expected.
\item
  \textbf{Facts}: A wife asked her husband if he was sexually
  interfering with her daughter. His silence while looking at the TV led
  the Victorian Court of Appeal to infer guilt.
\item
  \textbf{Significance}: An innocent person would have denied such a
  serious accusation, making silence indicative of consciousness of
  guilt.
\end{itemize}

\begin{center}\rule{0.5\linewidth}{0.5pt}\end{center}

\subsection{  Lies and Prior Inconsistent
Statements}\label{lies-and-prior-inconsistent-statements}

\subsubsection{Relevance of Lies}\label{relevance-of-lies}

\begin{itemize}
\tightlist
\item
  \textbf{Ordinarily}: Evidence of a lie is relevant for discrediting
  the honesty of the accused or a witness.
\item
  \textbf{However}: When an accused lies, it can also indicate
  consciousness of guilt.
\end{itemize}

\paragraph{Case: Edwards v R (1993) 178 CLR
193}\label{case-edwards-v-r-1993-178-clr-193}

\begin{itemize}
\tightlist
\item
  \textbf{Key Observation}:

  \begin{itemize}
  \tightlist
  \item
    Not every lie indicates guilt; only lies that are:

    \begin{itemize}
    \tightlist
    \item
      \textbf{Deliberate}
    \item
      \textbf{Inconsistent with innocence}
    \item
      \textbf{Related to a material issue}
    \end{itemize}
  \end{itemize}
\item
  \textbf{Conclusion}: The lie must stem from a consciousness that the
  truth would implicate the accused.
\end{itemize}

\begin{center}\rule{0.5\linewidth}{0.5pt}\end{center}

\subsection{�� Incriminating Conduct}\label{incriminating-conduct}

\subsubsection{Demeanor Not Consistent with
Innocence}\label{demeanor-not-consistent-with-innocence}

\begin{itemize}
\tightlist
\item
  \textbf{Types of Conduct}:

  \begin{itemize}
  \tightlist
  \item
    Fleeing the scene of a crime
  \item
    Attempting to evade police
  \item
    Destroying/hiding evidence
  \item
    Warning witnesses not to cooperate with police
  \end{itemize}
\end{itemize}

\paragraph{Case: R v Adam (1999) 106 A Crim R
510}\label{case-r-v-adam-1999-106-a-crim-r-510}

\begin{itemize}
\tightlist
\item
  \textbf{Key Point}: Conduct that reveals consciousness of guilt can be
  considered an implied admission.
\end{itemize}

\paragraph{Case: Woon v R (1964) 109 CLR
529}\label{case-woon-v-r-1964-109-clr-529}

\begin{itemize}
\tightlist
\item
  \textbf{Observation}: Selectivity in answering police questions alone
  is not indicative of guilt. However, in the overall context, it could
  imply a consciousness of guilt.
\end{itemize}

\begin{center}\rule{0.5\linewidth}{0.5pt}\end{center}

\subsection{ Right to Silence}\label{right-to-silence}

\subsubsection{Distinct Notions of Right to
Silence}\label{distinct-notions-of-right-to-silence}

\begin{enumerate}
\def\labelenumi{\arabic{enumi}.}
\tightlist
\item
  \textbf{Pre-trial Right to Silence}: The right not to answer police
  questioning.
\item
  \textbf{Right at Trial}: The right to remain silent and not testify.
\end{enumerate}

\paragraph{Key Legal Framework}\label{key-legal-framework}

\begin{itemize}
\tightlist
\item
  \textbf{EAQ, s 8(1)}: An accused cannot be compelled to testify
  against themselves.
\end{itemize}

\subsubsection{Case: R v Baden-Clay}\label{case-r-v-baden-clay}

\begin{itemize}
\tightlist
\item
  \textbf{Discussion}: Is silence an admission by conduct? Generally,
  silence is not an admission of guilt.
\end{itemize}

\paragraph{Case: R v Ireland (1970) 126 CLR
321}\label{case-r-v-ireland-1970-126-clr-321}

\begin{itemize}
\tightlist
\item
  \textbf{Facts}: Accused refused to answer questions during
  interrogation. The High Court ruled the video evidence was irrelevant
  and prejudicial.
\end{itemize}

\begin{center}\rule{0.5\linewidth}{0.5pt}\end{center}

\subsection{  Drawing Inferences from
Silence}\label{drawing-inferences-from-silence}

\subsubsection{Exception to General
Rule}\label{exception-to-general-rule}

\begin{itemize}
\tightlist
\item
  \textbf{Key Point}: Inferences can be drawn if the accused uniquely
  possesses knowledge but chooses to remain silent.
\end{itemize}

\paragraph{Case: Weissensteiner v R (1973) 178 CLR
217}\label{case-weissensteiner-v-r-1973-178-clr-217}

\begin{itemize}
\tightlist
\item
  \textbf{Facts}: The accused was charged with murder after being the
  only survivor of a boating trip. His silence was considered because he
  had unique knowledge of the events.
\end{itemize}

\subsubsection{Adverse Inferences}\label{adverse-inferences}

\begin{itemize}
\tightlist
\item
  \textbf{Legal Principle}: No adverse inference can be drawn from an
  accused's failure to provide a defense prior to trial.
\end{itemize}

\paragraph{Case: Petty and Maiden v The Queen (1991) 173 CLR
95}\label{case-petty-and-maiden-v-the-queen-1991-173-clr-95}

\begin{itemize}
\tightlist
\item
  \textbf{Conclusion}: Adverse inferences from silence would undermine
  the right to silence and should not be permitted.
\end{itemize}

\begin{center}\rule{0.5\linewidth}{0.5pt}\end{center}

\subsection{  Consciousness of Guilt}\label{consciousness-of-guilt}

\subsubsection{Woon v R (1964) 109 CLR
529}\label{woon-v-r-1964-109-clr-529}

\begin{itemize}
\tightlist
\item
  \textbf{Facts:} Woon was charged with breaking, entering, and stealing
  from a bank. He answered questions selectively, and his demeanor
  suggested a knowledge that indicated involvement in the crime.
\item
  \textbf{Key Finding:} The High Court recognized that while a statement
  made to an accused is not evidence, the accused's reaction may
  constitute an admission by conduct. The court emphasized the need for
  caution when admitting such statements as they may be equivocal.
\end{itemize}

\begin{quote}
``The test is whether the behavior is consistent with what an innocent
person would do or whether it indicates a consciousness of guilt.''
\end{quote}

\subsubsection{Weissensteiner v R (1973) 178 CLR
217}\label{weissensteiner-v-r-1973-178-clr-217}

\begin{itemize}
\tightlist
\item
  \textbf{Circumstantial Evidence:} This case allows for guilt to be
  inferred from a collection of circumstances, especially when the
  accused does not provide evidence that could easily be within their
  knowledge.
\item
  \textbf{Key Points:}

  \begin{itemize}
  \tightlist
  \item
    The appellant could have explained his possession of a boat that was
    central to the prosecution's case.
  \item
    The trial judge summarized that the crown must prove guilt beyond a
    reasonable doubt, and the accused has no obligation to prove
    anything.
  \item
    The absence of evidence from the accused can allow the jury to
    accept the prosecution's case as the only rational inference.
  \end{itemize}
\end{itemize}

\begin{longtable}[]{@{}
  >{\raggedright\arraybackslash}p{(\linewidth - 2\tabcolsep) * \real{0.3409}}
  >{\raggedright\arraybackslash}p{(\linewidth - 2\tabcolsep) * \real{0.6591}}@{}}
\toprule\noalign{}
\begin{minipage}[b]{\linewidth}\raggedright
Aspect
\end{minipage} & \begin{minipage}[b]{\linewidth}\raggedright
Details
\end{minipage} \\
\midrule\noalign{}
\endhead
\bottomrule\noalign{}
\endlastfoot
Crown's Responsibility & Must prove guilt beyond a reasonable doubt \\
Accused's Position & Under no obligation to present evidence \\
Implication of Silence & Silence cannot be taken as an admission of
guilt \\
\end{longtable}

\subsubsection{Implications of Silence}\label{implications-of-silence}

\begin{itemize}
\tightlist
\item
  The High Court ruled that the failure of the accused to provide
  evidence is not inherently evidence of guilt. It is an exercise of the
  right to silence.
\item
  Silence can only affect the probative value of the evidence already
  presented. If the accused's silence is relevant to the evidence's
  probative value, it may be considered by the jury.
\end{itemize}

\subsubsection{Right to Silence vs.~Inferences from
Silence}\label{right-to-silence-vs.-inferences-from-silence}

\begin{itemize}
\tightlist
\item
  The right to silence means that silence cannot act as a confession or
  indication of guilt.
\item
  The prosecution must still provide evidence to prove each element of
  the case beyond a reasonable doubt.
\item
  The jury can weigh the prosecution's evidence against the lack of
  evidence from the accused, leading them to reject unsupported
  hypotheses.
\end{itemize}

\subsection{  R v Baden-Clay {[}2016{]} HCA
35}\label{r-v-baden-clay-2016-hca-35}

\begin{itemize}
\tightlist
\item
  \textbf{Context:} The court briefly referenced Weissensteiner, stating
  that the defense does not have to prove an alternative hypothesis.
\item
  \textbf{Key Finding:} The jury can reject a hypothesis without
  evidence supporting the accused if the prosecution's case is
  established beyond a reasonable doubt.
\end{itemize}

\subsubsection{Important Concepts from
Baden-Clay}\label{important-concepts-from-baden-clay}

\begin{enumerate}
\def\labelenumi{\arabic{enumi}.}
\tightlist
\item
  \textbf{Disbelieved Evidence:} If the jury disbelieves the accused's
  evidence, it does not mean that the evidence has no bearing on the
  hypotheses of innocence.
\item
  \textbf{Hypothesis Evaluation:} The jury can reasonably reject any
  hypothesis for which there is no supporting evidence or that is
  clearly contradicted by the evidence presented.
\end{enumerate}

\begin{longtable}[]{@{}
  >{\raggedright\arraybackslash}p{(\linewidth - 2\tabcolsep) * \real{0.2500}}
  >{\raggedright\arraybackslash}p{(\linewidth - 2\tabcolsep) * \real{0.7500}}@{}}
\toprule\noalign{}
\begin{minipage}[b]{\linewidth}\raggedright
Case
\end{minipage} & \begin{minipage}[b]{\linewidth}\raggedright
Finding
\end{minipage} \\
\midrule\noalign{}
\endhead
\bottomrule\noalign{}
\endlastfoot
Weissensteiner & Silence can strengthen prosecution's case if the
accused does not explain their knowledge. \\
Baden-Clay & Disbelieved evidence still holds weight in evaluating
hypotheses of innocence. \\
\end{longtable}

\subsection{  The Right to Silence and
Confessions}\label{the-right-to-silence-and-confessions}

\subsubsection{Right to Silence}\label{right-to-silence-1}

\begin{itemize}
\tightlist
\item
  The \textbf{right to silence} can lead to implications in legal
  contexts, particularly regarding the admissibility of an accused's
  silence when faced with accusations.
\end{itemize}

\begin{quote}
``There have been cases where evidence of an accused's silence in the
face of an accusation has been admitted as an implicit adoption of that
accusation.''
\end{quote}

\begin{itemize}
\tightlist
\item
  It is essential that the accusation is \textbf{sufficiently specific}
  and made in circumstances where an \textbf{innocent person} would
  typically deny the allegation. Notable cases:

  \begin{itemize}
  \tightlist
  \item
    R v MMJ {[}2006{]} VSCA 226
  \item
    Parkes v R {[}1976{]} 1 WLR 1251
  \end{itemize}
\end{itemize}

\subsubsection{Admissibility of Silence}\label{admissibility-of-silence}

\begin{itemize}
\tightlist
\item
  Silence can constitute an admission under specific conditions.
  However, if an accused exercises their right to silence during police
  questioning, evidence of that silence is generally \textbf{not
  admissible}.
\end{itemize}

\subsubsection{Confessions}\label{confessions}

\paragraph{Definition of Confessions}\label{definition-of-confessions}

\begin{quote}
``Confessions = a special form of admission that involves a direct and
express acknowledgment of facts by an accused person suggesting guilt.''
\end{quote}

\begin{itemize}
\tightlist
\item
  \textbf{Confessions} differ from informal admissions, which may be
  indirect or implied from conduct.
\end{itemize}

\paragraph{Legal Treatment of
Confessions}\label{legal-treatment-of-confessions}

\begin{itemize}
\tightlist
\item
  Confessions are admitted as an exception to the hearsay rule due to
  their inherent reliability:

  \begin{itemize}
  \tightlist
  \item
    People are presumed unlikely to make statements adverse to their
    interests unless those statements are true.
  \end{itemize}
\end{itemize}

\begin{quote}
``Confessions are received in evidence as narrative statements made
trustworthy by the improbability of a party's falsely stating what tends
to expose him to penal or civil liability.''
\end{quote}

\paragraph{Importance of Confessions in
Trials}\label{importance-of-confessions-in-trials}

\begin{itemize}
\tightlist
\item
  The admissibility of a confession is often critical to the
  prosecution's case. Without it, the case may collapse or rely on
  unreliable evidence. A confession can be sufficient for conviction.
\end{itemize}

\subsubsection{Exclusion of Confessions}\label{exclusion-of-confessions}

\paragraph{Bases for Exclusion}\label{bases-for-exclusion}

There are four main bases for excluding a confession due to unfairness:

\begin{longtable}[]{@{}
  >{\raggedright\arraybackslash}p{(\linewidth - 2\tabcolsep) * \real{0.3500}}
  >{\raggedright\arraybackslash}p{(\linewidth - 2\tabcolsep) * \real{0.6500}}@{}}
\toprule\noalign{}
\begin{minipage}[b]{\linewidth}\raggedright
Basis
\end{minipage} & \begin{minipage}[b]{\linewidth}\raggedright
Description
\end{minipage} \\
\midrule\noalign{}
\endhead
\bottomrule\noalign{}
\endlastfoot
(1) Voluntariness & Confessions must be voluntary; an involuntary
confession is inadmissible. \\
(2) Unfair to Accused & If admitting the confession is unfair to the
accused. \\
(3) Against Public Policy & If the confession goes against public
interests. \\
(4) Probative Value vs.~Prejudicial Impact & If the probative value of
the confession is less than its prejudicial impact. \\
\end{longtable}

\subsubsection{Assessing Voluntariness}\label{assessing-voluntariness}

\begin{enumerate}
\def\labelenumi{\arabic{enumi}.}
\tightlist
\item
  \textbf{Voluntariness Requirement}:

  \begin{itemize}
  \tightlist
  \item
    A confession must be voluntary; if it is found to be involuntary, it
    is deemed inadmissible.
  \end{itemize}
\end{enumerate}

\begin{quote}
``The Crown has the burden of satisfying the trial judge in every case
as to the voluntary character of a statement before it becomes
admissible.''
\end{quote}

\begin{enumerate}
\def\labelenumi{\arabic{enumi}.}
\setcounter{enumi}{1}
\tightlist
\item
  \textbf{Judicial Standards for Voluntariness}:

  \begin{itemize}
  \tightlist
  \item
    A confessional statement will be inadmissible if it is shown to have
    been made involuntarily due to duress, intimidation, or persistent
    pressure.
  \item
    A statement is involuntary if it is preceded by an inducement
    (threat or promise) by someone in authority.
  \end{itemize}
\end{enumerate}

\subsubsection{Relevant Case Law}\label{relevant-case-law}

\begin{itemize}
\item
  \textbf{Tofilau v R (2007)}: This case illustrates that confessions
  obtained by undercover operations do not automatically render
  confessions involuntary, provided the accused does not perceive the
  undercover officers as persons in authority.
\item
  \textbf{Jack Thomas Case}: Highlighted issues of voluntariness when
  threats were made against an accused without legal counsel.
\end{itemize}

\subsubsection{Summary of Key Legal
Principles}\label{summary-of-key-legal-principles}

\begin{itemize}
\tightlist
\item
  \textbf{Voluntariness}: Essential for the admissibility of
  confessions.
\item
  \textbf{Judicial Discretions}: If a confession is voluntary, three
  judicial discretions may still apply.
\item
  \textbf{Person in Authority}: The definition impacts how confessions
  are evaluated, particularly in relation to threats or promises made.
\end{itemize}

This study guide encapsulates the critical elements discussed in the
lecture regarding the implications of the right to silence, the nature
of confessions, and the legal principles surrounding their admissibility
and potential exclusion in judicial proceedings.

\subsection{  Confessions and
Voluntariness}\label{confessions-and-voluntariness}

\subsubsection{Voluntariness in
Confessions}\label{voluntariness-in-confessions}

\begin{quote}
``Voluntariness is crucial in determining whether a confession can be
admitted as evidence in court.''
\end{quote}

\begin{itemize}
\tightlist
\item
  \textbf{Key Aspects}:

  \begin{itemize}
  \tightlist
  \item
    The \textbf{Australian Federal Police (AFP)} were aware that the
    voluntariness of a confession could be challenged.
  \item
    \textbf{Thomas} was given a choice to either speak or remain silent,
    indicating the importance of the suspect's subjective experience.
  \end{itemize}
\end{itemize}

\subsubsection{Subjective Experience and Police
Conduct}\label{subjective-experience-and-police-conduct}

\begin{itemize}
\tightlist
\item
  The legal focus is not on the \textbf{intentions} of the police but
  rather on how the \textbf{suspect experienced} the situation.
\item
  The question arises: \emph{Did the police conduct actually induce the
  suspect to confess?}
\end{itemize}

\subsubsection{Authority and Influence}\label{authority-and-influence}

\begin{itemize}
\tightlist
\item
  Both \textbf{Pakistani agents} and \textbf{AFP agents} were recognized
  as persons in authority.
\item
  The Pakistani agents applied pressure through \textbf{threats}, while
  the AFP's inaction (not taking Thomas back to Australia) tacitly
  endorsed this pressure.
\end{itemize}

\begin{longtable}[]{@{}
  >{\raggedright\arraybackslash}p{(\linewidth - 2\tabcolsep) * \real{0.5000}}
  >{\raggedright\arraybackslash}p{(\linewidth - 2\tabcolsep) * \real{0.5000}}@{}}
\toprule\noalign{}
\begin{minipage}[b]{\linewidth}\raggedright
Agent Type
\end{minipage} & \begin{minipage}[b]{\linewidth}\raggedright
Conduct
\end{minipage} \\
\midrule\noalign{}
\endhead
\bottomrule\noalign{}
\endlastfoot
Pakistani Agents & Used threats to pressure confession \\
AFP Agents & Failed to intervene, tacitly endorsing threats \\
\end{longtable}

\subsubsection{Judicial Discretion: Christie
Discretion}\label{judicial-discretion-christie-discretion}

\begin{quote}
``Christie Discretion allows a court to exclude evidence if it is deemed
more prejudicial than probative.''
\end{quote}

\begin{itemize}
\tightlist
\item
  \textbf{Definition}: The discretion is not a true discretion but a
  judgment based on whether evidence is more prejudicial than probative.
\end{itemize}

\subsubsection{Statutory Framework}\label{statutory-framework}

\begin{itemize}
\tightlist
\item
  Section \textbf{EAC, s 137}: In criminal proceedings, courts must
  refuse to admit evidence if its \textbf{probative value} is outweighed
  by the risk of unfair prejudice to the defendant.
\end{itemize}

\begin{longtable}[]{@{}
  >{\raggedright\arraybackslash}p{(\linewidth - 2\tabcolsep) * \real{0.5000}}
  >{\raggedright\arraybackslash}p{(\linewidth - 2\tabcolsep) * \real{0.5000}}@{}}
\toprule\noalign{}
\begin{minipage}[b]{\linewidth}\raggedright
Factor
\end{minipage} & \begin{minipage}[b]{\linewidth}\raggedright
Description
\end{minipage} \\
\midrule\noalign{}
\endhead
\bottomrule\noalign{}
\endlastfoot
Probative Value & Capacity of evidence to prove a fact in issue \\
Prejudicial Effect & Risk of evidence causing emotional or biased
judgment \\
\end{longtable}

\subsubsection{Two-Part Test for Evidence
Exclusion}\label{two-part-test-for-evidence-exclusion}

\begin{enumerate}
\def\labelenumi{\arabic{enumi}.}
\tightlist
\item
  \textbf{Consider Probative Value}:

  \begin{itemize}
  \tightlist
  \item
    Probative refers to the capacity to prove a fact.
  \end{itemize}
\item
  \textbf{Consider Risk of Prejudicial Use}:

  \begin{itemize}
  \tightlist
  \item
    Evidence that provokes sympathy or horror can lead to biased
    decision-making.
  \end{itemize}
\end{enumerate}

\subsubsection{Complexities of
Prejudice}\label{complexities-of-prejudice}

\begin{itemize}
\tightlist
\item
  The concept of \textbf{prejudice} involves evidence being prejudicial
  in a manner beyond its legitimate probative value.
\item
  The mere compelling nature of evidence does not make it prejudicial.
\end{itemize}

\subsubsection{Example Case: R v Duggan}\label{example-case-r-v-duggan}

\begin{itemize}
\tightlist
\item
  \textbf{Facts}:

  \begin{itemize}
  \tightlist
  \item
    Duggan confessed during police questioning while heavily
    intoxicated.
  \item
    Warnings were issued only after extensive questioning.
  \end{itemize}
\item
  \textbf{Judicial Findings}:

  \begin{itemize}
  \tightlist
  \item
    Duggan's intoxication undermined his voluntariness.
  \item
    The court determined that there was a denial of the right to
    silence.
  \end{itemize}
\end{itemize}

\begin{longtable}[]{@{}
  >{\raggedright\arraybackslash}p{(\linewidth - 2\tabcolsep) * \real{0.5000}}
  >{\raggedright\arraybackslash}p{(\linewidth - 2\tabcolsep) * \real{0.5000}}@{}}
\toprule\noalign{}
\begin{minipage}[b]{\linewidth}\raggedright
Factor
\end{minipage} & \begin{minipage}[b]{\linewidth}\raggedright
Description
\end{minipage} \\
\midrule\noalign{}
\endhead
\bottomrule\noalign{}
\endlastfoot
Intoxication & Affected the voluntariness of the confession \\
Timing of Warnings & Warnings issued after questioning initiated \\
\end{longtable}

\subsubsection{Public Policy
Considerations}\label{public-policy-considerations}

\begin{itemize}
\tightlist
\item
  The court balances the seriousness of charges (e.g., murder) against
  the potential unfairness of admitting evidence obtained improperly.
\item
  The exclusion of tainted evidence is justified if it does not
  significantly weaken the case against the accused.
\end{itemize}

\subsubsection{Conclusion on Discretion}\label{conclusion-on-discretion}

\begin{itemize}
\tightlist
\item
  Judicial discretion requires careful consideration of public interest
  and circumstances surrounding statements or admissions.
\item
  The exercise of discretion hinges on ensuring that justice is served
  without the risk of miscarriages due to improperly obtained
  confessions.
\end{itemize}

\subsection{  Unfairness Discretion in
Confessions}\label{unfairness-discretion-in-confessions}

\subsubsection{Definition of Unfairness
Discretion}\label{definition-of-unfairness-discretion}

\begin{quote}
``The unfairness discretion applies ONLY to confessions, and not to any
other type of evidence.''
\end{quote}

\begin{itemize}
\tightlist
\item
  The discretion to refuse to admit a confession on grounds of
  unfairness is unique to confessional evidence.
\end{itemize}

\subsubsection{Purpose of Unfairness
Discretion}\label{purpose-of-unfairness-discretion}

\begin{itemize}
\tightlist
\item
  Protects the \textbf{rights} and \textbf{privileges} of the accused,
  including the right to a \textbf{fair trial}.
\item
  Guards against \textbf{unfair disadvantages} arising from abuses of
  rights, such as the right to silence.
\end{itemize}

\subsubsection{Key Considerations}\label{key-considerations-1}

\begin{enumerate}
\def\labelenumi{\arabic{enumi}.}
\tightlist
\item
  \textbf{Voluntariness}: The confession must be voluntary.
\item
  \textbf{Reliability}: The confession must not be excluded based on
  considerations of reliability.
\item
  \textbf{Unfair Forensic Disadvantage}: Admission of the confession
  must not cause an unfair disadvantage to the accused.
\end{enumerate}

\subsubsection{Common Law and Statutory
Discretion}\label{common-law-and-statutory-discretion}

\begin{itemize}
\tightlist
\item
  Unfairness tests are established both in \textbf{Common Law} and
  \textbf{Statute}.
\item
  \textbf{Statutory Discretion} (EAQ, s 130): Courts can exclude
  evidence if it would be unfair to admit it.
\end{itemize}

\subsubsection{Common Law Approach}\label{common-law-approach}

In assessing the admissibility of confessions, the following factors are
considered:

\begin{longtable}[]{@{}
  >{\raggedright\arraybackslash}p{(\linewidth - 2\tabcolsep) * \real{0.2462}}
  >{\raggedright\arraybackslash}p{(\linewidth - 2\tabcolsep) * \real{0.7538}}@{}}
\toprule\noalign{}
\begin{minipage}[b]{\linewidth}\raggedright
Factor
\end{minipage} & \begin{minipage}[b]{\linewidth}\raggedright
Description
\end{minipage} \\
\midrule\noalign{}
\endhead
\bottomrule\noalign{}
\endlastfoot
Voluntariness & Is the confession voluntary? \\
Reliability & Can the confession be excluded based on compromised
reliability? \\
Unfair Forensic Disadvantage & Will admitting the confession cause
unfair disadvantage? \\
\end{longtable}

\subsubsection{Relevant Case Law}\label{relevant-case-law-1}

\begin{itemize}
\tightlist
\item
  \textbf{Foster v R (1993)}: Established that unfairness discretion can
  apply even to lawfully obtained confessions.
\item
  \textbf{R v Swaffield; Pavic v R (1998)}: Highlights the importance of
  considering the means of confession elicitation.
\end{itemize}

\subsubsection{Case Examples}\label{case-examples-2}

\begin{enumerate}
\def\labelenumi{\arabic{enumi}.}
\tightlist
\item
  \textbf{R v Swaffield}:

  \begin{itemize}
  \tightlist
  \item
    Facts: In a trial for arson, the judge declined to exclude secretly
    recorded admissions made by the accused. These were excluded on
    appeal.
  \end{itemize}
\item
  \textbf{R v Pavic}:

  \begin{itemize}
  \tightlist
  \item
    Facts: In a murder trial, a taped confession made to a friend was
    admitted into evidence. The Court upheld the judge's decision.
  \end{itemize}
\item
  \textbf{R v Belford \& Bound {[}2011{]}}:

  \begin{itemize}
  \tightlist
  \item
    Facts: Discussed the admissibility of confessions made in a cell
    with undercover police operatives. Majority opinion allowed the
    evidence due to the absence of unlawful police conduct.
  \end{itemize}
\end{enumerate}

\subsubsection{Public Policy Discretion}\label{public-policy-discretion}

The discretion to exclude evidence based on public policy arises when
evidence is tainted by illegality or impropriety from law enforcement.

\begin{longtable}[]{@{}
  >{\raggedright\arraybackslash}p{(\linewidth - 2\tabcolsep) * \real{0.2348}}
  >{\raggedright\arraybackslash}p{(\linewidth - 2\tabcolsep) * \real{0.7652}}@{}}
\toprule\noalign{}
\begin{minipage}[b]{\linewidth}\raggedright
\textbf{Public Policy Interests}
\end{minipage} & \begin{minipage}[b]{\linewidth}\raggedright
\textbf{Description}
\end{minipage} \\
\midrule\noalign{}
\endhead
\bottomrule\noalign{}
\endlastfoot
Convicting Criminals & Need for deterrence and safety. \\
Judicial Approval of Misconduct & Avoiding tacit approval of unlawful
conduct by law enforcement. \\
\end{longtable}

\subsubsection{Additional Case Law}\label{additional-case-law}

\begin{itemize}
\tightlist
\item
  \textbf{R v Ireland (1970)}: Balances the need to convict criminals
  against the undesirable effects of approving unlawful police conduct.
\end{itemize}

\subsubsection{Summary of Key Points}\label{summary-of-key-points-1}

\begin{itemize}
\tightlist
\item
  \textbf{Unfairness discretion} solely applies to confessions and
  involves assessing the \textbf{voluntariness}, \textbf{reliability},
  and potential \textbf{unfair disadvantages} caused by admitting a
  confession.
\item
  Legal precedents illustrate the application of this discretion and
  highlight the importance of context in assessing confessional
  evidence.
\end{itemize}

\subsection{  Admissibility of
Evidence}\label{admissibility-of-evidence-4}

\subsubsection{General Principles}\label{general-principles-2}

\begin{quote}
``The general rule is that silence is not an implied admission because
it is usually not relevant and, even if relevant, probably prejudicial
and would undermine the accused's right to silence.''
\end{quote}

\begin{itemize}
\tightlist
\item
  The case of \textbf{Bunning v Cross (1977) 141 CLR 54} establishes
  that unlawfully obtained evidence is admissible if it is relevant and
  not otherwise excluded.
\item
  However, if evidence is illegally obtained, public policy discretion
  must be considered, requiring a balance between:

  \begin{itemize}
  \tightlist
  \item
    The need to convict criminals
  \item
    The undesirable effect of tacit judicial approval of unlawful
    conduct by law enforcement.
  \end{itemize}
\end{itemize}

\subsubsection{Factors in Discretionary
Exclusion}\label{factors-in-discretionary-exclusion}

In assessing whether to exclude unlawfully obtained evidence, the judge
must consider:

\begin{longtable}[]{@{}
  >{\raggedright\arraybackslash}p{(\linewidth - 2\tabcolsep) * \real{0.3355}}
  >{\raggedright\arraybackslash}p{(\linewidth - 2\tabcolsep) * \real{0.6645}}@{}}
\toprule\noalign{}
\begin{minipage}[b]{\linewidth}\raggedright
Factor
\end{minipage} & \begin{minipage}[b]{\linewidth}\raggedright
Consideration
\end{minipage} \\
\midrule\noalign{}
\endhead
\bottomrule\noalign{}
\endlastfoot
Nature and seriousness of alleged offending & Importance of bringing
guilty offenders to justice. \\
Nature of wrongdoing by authorities & Whether it was deliberate,
reckless, or merely inadvertent. \\
Cogency of evidence & High levels of cogency favor admission;
inadvertent breaches may be treated differently. \\
Ease of compliance & Consider how easily the police could have complied
with the law. \\
Legislative intention & Whether there is a legislative intention to
constrain police conduct in relation to their powers. \\
\end{longtable}

\subsubsection{\texorpdfstring{Key Case: \textbf{Pollard v R (1992) 176
CLR
177}}{Key Case: Pollard v R (1992) 176 CLR 177}}\label{key-case-pollard-v-r-1992-176-clr-177}

\begin{quote}
``The principal considerations of `high public policy' which favour
exclusion of evidence procured by unlawful conduct transcend any
question of unfairness to the particular accused.''
\end{quote}

\begin{itemize}
\tightlist
\item
  Courts must ensure that unlawful police conduct is not encouraged by
  judicial acquiescence.
\item
  High public policy concerns take precedence over fairness to the
  accused.
\end{itemize}

\subsection{  Propensity and Similar Fact
Evidence}\label{propensity-and-similar-fact-evidence}

\subsubsection{Definitions}\label{definitions-3}

\begin{itemize}
\tightlist
\item
  \textbf{Propensity Evidence}: Evidence that indicates the accused has
  committed other criminal acts, suggesting they are the type to commit
  crimes of a certain nature.
\item
  \textbf{Similar Fact Evidence (SFE)}: Evidence that demonstrates the
  accused has committed similar criminal acts before, with a high level
  of cogency.
\end{itemize}

\subsubsection{Admissibility Rules}\label{admissibility-rules}

\begin{longtable}[]{@{}
  >{\raggedright\arraybackslash}p{(\linewidth - 4\tabcolsep) * \real{0.1379}}
  >{\raggedright\arraybackslash}p{(\linewidth - 4\tabcolsep) * \real{0.4224}}
  >{\raggedright\arraybackslash}p{(\linewidth - 4\tabcolsep) * \real{0.4397}}@{}}
\toprule\noalign{}
\begin{minipage}[b]{\linewidth}\raggedright
Evidence Type
\end{minipage} & \begin{minipage}[b]{\linewidth}\raggedright
General Rule
\end{minipage} & \begin{minipage}[b]{\linewidth}\raggedright
Exception
\end{minipage} \\
\midrule\noalign{}
\endhead
\bottomrule\noalign{}
\endlastfoot
\textbf{Propensity Evidence} & Generally inadmissible; prima facie more
prejudicial than probative. & N/A \\
\textbf{Similar Fact Evidence (SFE)} & Generally admissible if relevant
for non-propensity purposes. & May be excluded if its prejudicial effect
outweighs its probative value. \\
\end{longtable}

\subsubsection{Key Considerations for Admissibility of
SFE}\label{key-considerations-for-admissibility-of-sfe}

\begin{enumerate}
\def\labelenumi{\arabic{enumi}.}
\tightlist
\item
  \textbf{High Degree of Cogency}: SFE must be very compelling due to
  the potential for prejudice.
\item
  \textbf{Striking Similarity}: Evidence must often show a pattern that
  is objectively improbable to have occurred other than as alleged.
\item
  \textbf{Contextual Evaluation}: The whole prosecution case must be
  considered to determine if any rational view of the evidence is
  consistent with the accused's innocence.
\end{enumerate}

\subsubsection{Leading Cases}\label{leading-cases-1}

\begin{itemize}
\tightlist
\item
  \textbf{Makin v A-G (NSW) {[}1894{]} AC 57}: Established the two-limb
  rule regarding admissibility of SFE.
\item
  \textbf{Pfennig v R (1995) 182 CLR 461}: Outlined necessary conditions
  for SFE's admissibility:

  \begin{itemize}
  \tightlist
  \item
    SFE must have a high degree of cogency.
  \item
    The evidence must negate any defense of accident or natural cause.
  \end{itemize}
\end{itemize}

\subsubsection{\texorpdfstring{Example Case: \textbf{Pfennig v R (1994)
182 CLR
461}}{Example Case: Pfennig v R (1994) 182 CLR 461}}\label{example-case-pfennig-v-r-1994-182-clr-461}

\begin{itemize}
\tightlist
\item
  Accused charged with kidnapping and murder.
\item
  Circumstantial evidence presented included:

  \begin{itemize}
  \tightlist
  \item
    Possessions found at a specific location.
  \item
    Previous sightings of the accused near the scene.
  \end{itemize}
\item
  SFE admitted at trial as it was relevant to negate the defense of
  accident.
\end{itemize}

\subsubsection{Summary of SFE
Admissibility}\label{summary-of-sfe-admissibility}

\begin{itemize}
\tightlist
\item
  The admissibility of SFE hinges on whether its probative value exceeds
  its prejudicial effect, assessed within the context of the entire
  prosecution case.
\end{itemize}

\subsection{  Similar Fact Evidence
(SFE)}\label{similar-fact-evidence-sfe}

\subsubsection{Definition of Similar Fact Evidence
(SFE)}\label{definition-of-similar-fact-evidence-sfe}

\begin{quote}
``Similar Fact Evidence refers to evidence that is used in court to show
patterns of behavior or conduct by the accused that are relevant to the
charges being considered.''
\end{quote}

\subsubsection{Context of SFE in Legal
Proceedings}\label{context-of-sfe-in-legal-proceedings}

\begin{itemize}
\tightlist
\item
  SFE is particularly relevant in cases involving serious crimes such as
  \textbf{sexual offenses}, where the prosecution may seek to use
  evidence of prior similar acts to establish a pattern of behavior.
\item
  In the case of \textbf{Pfenning}, SFE was utilized to support the
  argument that he was responsible for the crimes against Harry, given
  his prior convictions for similar offenses.
\end{itemize}

\subsubsection{Admissibility of SFE}\label{admissibility-of-sfe}

The test for the admissibility of SFE involves two key criteria:

\begin{enumerate}
\def\labelenumi{\arabic{enumi}.}
\tightlist
\item
  \textbf{Probative Value}: The SFE must possess a particular probative
  value or cogency such that, if accepted, it supports an inference of
  the accused's guilt regarding the offense charged.
\item
  \textbf{Objective Improbability}: The evidence must be such that there
  is no reasonable explanation for it other than as supporting an
  inference of guilt.
\end{enumerate}

\subsubsection{Grounds of Appeal Related to
SFE}\label{grounds-of-appeal-related-to-sfe}

\begin{itemize}
\tightlist
\item
  \textbf{Trial Judge's Allowance}: The trial judge allowed SFE to be
  used for propensity, which became a point of contention.
\item
  \textbf{Misapplication of Legal Precedents}: The High Court held that
  the case of \textbf{Re Makin} had been misunderstood, emphasizing that
  the admissibility of SFE should not be overly narrowed.
\end{itemize}

\subsubsection{Relevant Cases Involving
SFE}\label{relevant-cases-involving-sfe}

\begin{longtable}[]{@{}
  >{\raggedright\arraybackslash}p{(\linewidth - 4\tabcolsep) * \real{0.0962}}
  >{\raggedright\arraybackslash}p{(\linewidth - 4\tabcolsep) * \real{0.4863}}
  >{\raggedright\arraybackslash}p{(\linewidth - 4\tabcolsep) * \real{0.4175}}@{}}
\toprule\noalign{}
\begin{minipage}[b]{\linewidth}\raggedright
Case
\end{minipage} & \begin{minipage}[b]{\linewidth}\raggedright
Facts
\end{minipage} & \begin{minipage}[b]{\linewidth}\raggedright
Outcome
\end{minipage} \\
\midrule\noalign{}
\endhead
\bottomrule\noalign{}
\endlastfoot
\textbf{Hoch v R (1988)} & Involved charges from three boys, where the
defense argued concoction of stories. & The High Court held that SFE
should be admissible unless there is a reasonable view consistent with
innocence. \\
\textbf{R v Phillips (2006)} & Multiple rape charges with claims of
consent and varying degrees of violence. & The court ruled that SFE must
be exceptionally probative and must clearly transcend prejudicial
effects. \\
\textbf{BBH v R (2012)} & Involved charges against a father by his
daughter, with testimony from the complainant's brother. & The High
Court upheld the admissibility of SFE, citing its relevance to
demonstrate sexual interest. \\
\end{longtable}

\subsubsection{Legislative Responses to SFE
Issues}\label{legislative-responses-to-sfe-issues}

\begin{itemize}
\tightlist
\item
  Following the \textbf{Hoch} ruling, legislation was enacted (s 132A
  QCC) that clarified the standards for the admissibility of SFE. It
  states that SFE must not be ruled inadmissible based solely on the
  grounds of possible collusion or concoction.
\end{itemize}

\subsubsection{Directions for Jury Regarding
SFE}\label{directions-for-jury-regarding-sfe}

\begin{itemize}
\tightlist
\item
  In cases where SFE is admitted, courts must provide clear directions
  to prevent misuse of such evidence. The jury should consider competing
  circumstantial inferences, including the possibility of collusion.
\end{itemize}

\subsubsection{Summary of Key
Principles}\label{summary-of-key-principles}

\begin{itemize}
\tightlist
\item
  SFE can be used to:

  \begin{itemize}
  \tightlist
  \item
    \textbf{Circumstantially prove identity}
  \item
    \textbf{Circumstantially disprove a defense}
  \item
    \textbf{Render it improbable that separate complainants would make
    similar false allegations}
  \end{itemize}
\item
  \textbf{Crucial Test for SFE}: There must be no reasonable view of the
  facts consistent with the accused's innocence for SFE to be
  admissible.
\end{itemize}

\subsubsection{Conclusion on SFE's Role in Legal
Proceedings}\label{conclusion-on-sfes-role-in-legal-proceedings}

\begin{itemize}
\tightlist
\item
  SFE plays a significant role in serious criminal cases, particularly
  concerning the credibility and reliability of evidence against the
  accused. Its admissibility hinges on careful consideration of
  probative value versus prejudicial effect.
\end{itemize}

\subsection{  Character Evidence in Legal
Context}\label{character-evidence-in-legal-context}

\subsubsection{Definitions and
Concepts}\label{definitions-and-concepts-1}

\begin{quote}
``Character evidence refers to the inherent moral qualities of a person
and how those qualities are perceived within the community.''
\end{quote}

\begin{longtable}[]{@{}
  >{\raggedright\arraybackslash}p{(\linewidth - 2\tabcolsep) * \real{0.2045}}
  >{\raggedright\arraybackslash}p{(\linewidth - 2\tabcolsep) * \real{0.7955}}@{}}
\toprule\noalign{}
\begin{minipage}[b]{\linewidth}\raggedright
\textbf{Term}
\end{minipage} & \begin{minipage}[b]{\linewidth}\raggedright
\textbf{Definition}
\end{minipage} \\
\midrule\noalign{}
\endhead
\bottomrule\noalign{}
\endlastfoot
\textbf{Good Character} & Signifies that a defendant in a criminal case
has no previous convictions. \\
\textbf{Similar Fact Evidence} & Common law test for admitting evidence
of previous bad character to suggest misconduct in the present. \\
\textbf{Credibility} & The likelihood that a witness is telling the
truth. \\
\textbf{Propensity} & The likelihood that a person who has behaved in a
certain manner in the past will do so again. \\
\end{longtable}

\subsubsection{Civil Cases and Evidence}\label{civil-cases-and-evidence}

In civil cases, such as \emph{SFE Mister Figgins P/L v Centrepoint
Freeholds P/L (1981)}, the admissibility of evidence is based on:

\begin{itemize}
\tightlist
\item
  \textbf{Logical Relevance:} Evidence must be logically probative.
\item
  \textbf{Lower Standard of Proof:} This is a lower benchmark compared
  to criminal cases, focusing on avoiding a miscarriage of justice.
\end{itemize}

\subsubsection{Distinction Between Character Evidence and Propensity
Evidence}\label{distinction-between-character-evidence-and-propensity-evidence}

\begin{itemize}
\tightlist
\item
  \textbf{Character Evidence} is about the overall reputation of a
  person, while \textbf{Propensity Evidence} is used to imply that past
  behavior indicates guilt in the current case.
\end{itemize}

\subsubsection{Character Evidence of the
Accused}\label{character-evidence-of-the-accused}

\begin{enumerate}
\def\labelenumi{\arabic{enumi}.}
\tightlist
\item
  \textbf{Leading Evidence of Good Character:}

  \begin{itemize}
  \tightlist
  \item
    The accused can call witnesses to testify about their good
    character.
  \item
    Evidence must be based on the general reputation within the
    community, not specific acts.
  \end{itemize}
\item
  \textbf{Rebuttal by Prosecution:}

  \begin{itemize}
  \tightlist
  \item
    The prosecution can only present evidence of the accused's bad
    character if the accused has first introduced evidence of good
    character.
  \end{itemize}
\end{enumerate}

\subsubsection{Legal Cases Illustrating Character
Evidence}\label{legal-cases-illustrating-character-evidence}

\begin{itemize}
\item
  \textbf{R v Rowton (1865):} Established that evidence for or against a
  person's good character must be limited to general reputation.
\item
  \textbf{R v Redgrave (1981):} Confirmed that good character in trials
  usually implies no previous convictions.
\item
  \textbf{R v Trimboli (1979):} Emphasized that evidence of good
  character must be considered when determining if the Crown has proven
  guilt beyond a reasonable doubt.
\end{itemize}

\subsubsection{Judicial Directions on Good Character
Evidence}\label{judicial-directions-on-good-character-evidence}

Three influential propositions for judicial directions regarding good
character evidence:

\begin{enumerate}
\def\labelenumi{\arabic{enumi}.}
\tightlist
\item
  A direction should be given on how the jury might use evidence of good
  character.
\item
  There's no specific wording required, but the jury should consider
  past good character while assessing guilt.
\item
  Judges may remind juries that individuals can commit crimes for the
  first time, thus good character cannot outweigh compelling evidence of
  guilt.
\end{enumerate}

\subsubsection{Procedural Issues in Good Character
Evidence}\label{procedural-issues-in-good-character-evidence}

\begin{itemize}
\tightlist
\item
  In cases like \emph{Melbourne v R (1999)}, it was noted that a trial
  judge doesn't have to direct the jury on the use of good character
  evidence, yet concerns exist that such evidence may distract from the
  core issues of the case.
\end{itemize}

\subsubsection{Key Principles Regarding Good Character
Evidence}\label{key-principles-regarding-good-character-evidence}

\begin{itemize}
\tightlist
\item
  \textbf{Presumption of Innocence:} Good character evidence is admitted
  condition-free, unlike stricter requirements for bad character
  evidence to avoid unfair conviction.
\item
  \textbf{Historical Indulgence:} The unconditional right to present
  good character evidence is maintained for historical and humane
  reasons, rather than purely logical grounds.
\end{itemize}

\subsubsection{Putting Character in
Issue}\label{putting-character-in-issue}

\begin{itemize}
\tightlist
\item
  The prosecution may introduce evidence of bad character only to rebut
  claims made by the accused regarding their good character. This is
  seen in cases such as \emph{R v Perrier (No.~1) (1991)} where the
  court retains discretion to disallow evidence that could be
  disproportionately prejudicial.
\end{itemize}

\begin{longtable}[]{@{}
  >{\raggedright\arraybackslash}p{(\linewidth - 2\tabcolsep) * \real{0.1811}}
  >{\raggedright\arraybackslash}p{(\linewidth - 2\tabcolsep) * \real{0.8189}}@{}}
\toprule\noalign{}
\begin{minipage}[b]{\linewidth}\raggedright
\textbf{Case}
\end{minipage} & \begin{minipage}[b]{\linewidth}\raggedright
\textbf{Key Facts}
\end{minipage} \\
\midrule\noalign{}
\endhead
\bottomrule\noalign{}
\endlastfoot
\emph{SFE Mister Figgins P/L v Centrepoint Freeholds P/L} &
Admissibility of evidence based on logical relevance and lower standard
of proof. \\
\emph{R v Rowton (1865)} & Evidence for or against good character must
be confined to general reputation. \\
\emph{R v Trimboli (1979)} & Evidence of good character must be taken
into account when proving guilt beyond a reasonable doubt. \\
\emph{Melbourne v R (1999)} & The judge does not need to explain how
good character evidence should be used by the jury. \\
\end{longtable}

\section{  Character Evidence in Criminal
Trials}\label{character-evidence-in-criminal-trials}

\subsection{Testimonial Evidence}\label{testimonial-evidence}

In the case of \textbf{Murray Perrier}, a letter written by a Crown
witness was presented during the trial. The letter stated:

\begin{quote}
``During the short time that I have known Murray Perrier I have always
found him to be scrupulously honest and reliable.''
\end{quote}

\subsubsection{Key Points from the
Case:}\label{key-points-from-the-case}

\begin{itemize}
\tightlist
\item
  The trial judge allowed the Crown to present prior convictions of
  Perrier after the character was put into issue by the defense.
\item
  Prior convictions included:

  \begin{itemize}
  \tightlist
  \item
    A \textbf{1967 conviction for trafficking drugs} in Switzerland.
  \item
    A recent conviction for \textbf{possession of heroin} in Singapore.
  \end{itemize}
\end{itemize}

\subsubsection{Court's Ruling:}\label{courts-ruling}

\begin{itemize}
\tightlist
\item
  The judge determined the letter was a \textbf{testimonial}, and the
  counsel intended for the jury to perceive Perrier as a person of good
  character.
\item
  On appeal, Perrier's arguments included:

  \begin{enumerate}
  \def\labelenumi{\arabic{enumi}.}
  \tightlist
  \item
    The reading of the letter did not put his character in issue.
  \item
    The judge should have excluded bad-character evidence as
    overwhelmingly prejudicial.
  \end{enumerate}
\end{itemize}

\subsubsection{Important Concepts:}\label{important-concepts}

\begin{itemize}
\tightlist
\item
  \textbf{Judicial Discretion}: The judge has the authority to disallow
  evidence if it is disproportionately prejudicial to the accused.
\end{itemize}

\subsection{Bad Character Evidence}\label{bad-character-evidence}

In \textbf{Phillips v R (1985)}, the court addressed the prosecution's
ability to present the accused's prior character evidence.

\subsubsection{Case Summary:}\label{case-summary}

\begin{itemize}
\tightlist
\item
  Phillips was charged with \textbf{rape} and \textbf{breaking and
  entering}.
\item
  The defense raised previous convictions to prove dishonesty.
\end{itemize}

\subsubsection{Statutory Provisions:}\label{statutory-provisions-1}

\begin{itemize}
\tightlist
\item
  \textbf{EAQ, s 15(2)}: Evidence of prior convictions or bad character
  is generally inadmissible in a criminal trial when the accused gives
  evidence.
\end{itemize}

\subsubsection{Exceptions:}\label{exceptions-1}

\begin{itemize}
\tightlist
\item
  \textbf{Similar fact evidence} (s 15(2)(a))
\item
  \textbf{Bad character evidence} (s 15(2)(c)):

  \begin{itemize}
  \tightlist
  \item
    The accused must give evidence.
  \item
    The defense must trigger the exception by:

    \begin{itemize}
    \tightlist
    \item
      Adducing evidence of good character.
    \item
      Implicating a prosecution witness's character.
    \end{itemize}
  \end{itemize}
\end{itemize}

\subsection{Co-Accused and Character
Evidence}\label{co-accused-and-character-evidence}

When a co-accused blames another, they may be cross-examined regarding
their character.

\subsubsection{\texorpdfstring{Case Example: \textbf{Lowery v R
(1974)}}{Case Example: Lowery v R (1974)}}\label{case-example-lowery-v-r-1974}

\begin{itemize}
\tightlist
\item
  Lowery and King were charged with murder. Lowery claimed he was not
  involved, blaming King.
\item
  King introduced psychologist evidence to contrast their personalities.
\end{itemize}

\subsubsection{Court's View:}\label{courts-view}

\begin{itemize}
\tightlist
\item
  The psychologist's evidence was relevant to show the likelihood of
  Lowery's culpability.
\item
  It was deemed unfair for King not to rebut Lowery's claims about
  character.
\end{itemize}

\subsection{Character of a Witness}\label{character-of-a-witness}

Witnesses may have their character impugned through:

\begin{itemize}
\tightlist
\item
  \textbf{Prior Inconsistent Statements}: EAQ, s 18 allows for proof of
  previous statements that contradict current testimony.
\end{itemize}

\subsubsection{Prior Convictions:}\label{prior-convictions}

\begin{itemize}
\tightlist
\item
  Under EAQ, s 16, witnesses can be questioned about prior convictions
  if:

  \begin{itemize}
  \tightlist
  \item
    The witness denies or refuses to answer.
  \end{itemize}
\end{itemize}

\subsubsection{General Limits:}\label{general-limits}

\begin{itemize}
\tightlist
\item
  \textbf{EAQ, s 20}: Courts may disallow questions regarding a
  witness's credibility if admitting the truth does not impair
  confidence significantly in their reliability.
\end{itemize}

\subsubsection{Summary of Statutory
Limits:}\label{summary-of-statutory-limits}

\begin{longtable}[]{@{}
  >{\raggedright\arraybackslash}p{(\linewidth - 2\tabcolsep) * \real{0.2143}}
  >{\raggedright\arraybackslash}p{(\linewidth - 2\tabcolsep) * \real{0.7857}}@{}}
\toprule\noalign{}
\begin{minipage}[b]{\linewidth}\raggedright
Statutory Provision
\end{minipage} & \begin{minipage}[b]{\linewidth}\raggedright
Description
\end{minipage} \\
\midrule\noalign{}
\endhead
\bottomrule\noalign{}
\endlastfoot
EAQ, s 15(2) & Prior convictions/bad character inadmissible unless
exceptions apply. \\
EAQ, s 18 & Allows proof of prior inconsistent statements. \\
EAQ, s 16 & Allows questioning of witnesses about prior convictions. \\
EAQ, s 20 & Courts may disallow questions affecting witness
credibility. \\
\end{longtable}

\subsection{  Witness Credibility and Character
Evidence}\label{witness-credibility-and-character-evidence}

\subsubsection{Disallowance of Improper
Questions}\label{disallowance-of-improper-questions}

\begin{itemize}
\tightlist
\item
  The court may disallow a question during cross-examination if deemed
  improper.
\end{itemize}

\subsubsection{Importance of Character
Evidence}\label{importance-of-character-evidence}

\begin{itemize}
\tightlist
\item
  Character evidence is crucial for assessing a witness's credibility.
\item
  A party's own witness is often introduced in a favorable light through
  initial questions.
\item
  Cross-examination aims to test the credibility of a witness, often
  involving character-related inquiries.
\end{itemize}

\subsubsection{Finality Principle}\label{finality-principle}

\begin{quote}
``Under the finality principle, a witness's answers to questions about
their credit cannot be contradicted by the cross-examiner through other
evidence, with some exceptions.''
\end{quote}

\begin{center}\rule{0.5\linewidth}{0.5pt}\end{center}

\subsection{  Character of the Victim: Case
Study}\label{character-of-the-victim-case-study}

\subsubsection{Case: Re Knowles {[}1984{]} VR
751}\label{case-re-knowles-1984-vr-751}

\begin{itemize}
\tightlist
\item
  \textbf{Facts}: Knowles was convicted of stabbing his wife and
  appealed, claiming his defense counsel failed to present character
  witnesses that would show the victim's mental instability.
\end{itemize}

\begin{center}\rule{0.5\linewidth}{0.5pt}\end{center}

\subsection{�� Rape and Sexual Assault
Cases}\label{rape-and-sexual-assault-cases}

\subsubsection{Common Law Rules}\label{common-law-rules}

\begin{itemize}
\tightlist
\item
  Early common law rules about questioning rape victims have faced
  substantial criticism.
\item
  Legislation has been introduced to protect victims and limit the
  admission of prior sexual experience in court.
\end{itemize}

\subsubsection{Relevant Legislation}\label{relevant-legislation}

\begin{itemize}
\tightlist
\item
  \textbf{Criminal Law (Sexual Offences) Act 1978 (Qld) (CLSOA)},
  Section 4 includes significant restrictions regarding evidence in rape
  cases.
\end{itemize}

\subsubsection{Prosecution Requirements in Rape
Cases}\label{prosecution-requirements-in-rape-cases}

\begin{itemize}
\tightlist
\item
  The prosecution must prove:

  \begin{itemize}
  \tightlist
  \item
    The victim's state of mind (i.e., lack of consent)
  \item
    The accused's state of mind regarding the victim's lack of consent
  \end{itemize}
\end{itemize}

\subsubsection{Questions Regarding
Cross-examination}\label{questions-regarding-cross-examination}

\begin{enumerate}
\def\labelenumi{\arabic{enumi}.}
\tightlist
\item
  Should a victim be cross-examined about sexual relations with others?
\item
  Should a victim be questioned about sexual relations with the accused
  before or after the alleged rape?
\end{enumerate}

\begin{center}\rule{0.5\linewidth}{0.5pt}\end{center}

\subsection{  CLSOA Section 4: Special Rules Limiting
Evidence}\label{clsoa-section-4-special-rules-limiting-evidence}

\begin{longtable}[]{@{}
  >{\raggedright\arraybackslash}p{(\linewidth - 2\tabcolsep) * \real{0.3158}}
  >{\raggedright\arraybackslash}p{(\linewidth - 2\tabcolsep) * \real{0.6842}}@{}}
\toprule\noalign{}
\begin{minipage}[b]{\linewidth}\raggedright
Rule
\end{minipage} & \begin{minipage}[b]{\linewidth}\raggedright
Description
\end{minipage} \\
\midrule\noalign{}
\endhead
\bottomrule\noalign{}
\endlastfoot
1 & Evidence of the complainant's general reputation regarding chastity
is inadmissible. \\
2 & Without court approval, questions regarding the complainant's sexual
activities are not permitted. \\
3 & Court approval for cross-examination on sexual activities requires
substantial relevance to the facts. \\
4 & Evidence of sexual activity is not relevant solely based on
inferences about the complainant's disposition. \\
5 & Evidence of past sexual activity cannot be used to undermine the
complainant's credibility unless it materially impairs confidence in
their evidence. \\
6 & Applications for leave under rule 2 are made in the absence of the
jury and the complainant. \\
\end{longtable}

\begin{center}\rule{0.5\linewidth}{0.5pt}\end{center}

\subsection{ Case Studies on Rape and
Consent}\label{case-studies-on-rape-and-consent}

\subsubsection{R v De Angelis (1979)}\label{r-v-de-angelis-1979}

\begin{itemize}
\tightlist
\item
  \textbf{Facts}: Four accused allegedly raped a girl after forcing her
  into a car.
\item
  \textbf{Decision}: The court allowed evidence of the victim's prior
  sexual behavior to establish consent under similar circumstances.
\end{itemize}

\subsubsection{Kilby v R (1973)}\label{kilby-v-r-1973}

\begin{itemize}
\tightlist
\item
  \textbf{Finding}: The court ruled that the absence of a complaint does
  not equate to proof of consent. A victim's delay in reporting does not
  imply consent.
\end{itemize}

\subsubsection{Bull v R (2000)}\label{bull-v-r-2000}

\begin{itemize}
\tightlist
\item
  \textbf{Overview}: Evidence of a conversation between the complainant
  and the accused prior to the alleged rape was admissible to infer
  consent and state of mind.
\end{itemize}

\begin{center}\rule{0.5\linewidth}{0.5pt}\end{center}

\subsection{ CLSOA Section 4A: Evidence of
Complaint}\label{clsoa-section-4a-evidence-of-complaint}

\subsubsection{Admissibility Rules}\label{admissibility-rules-1}

\begin{enumerate}
\def\labelenumi{\arabic{enumi}.}
\tightlist
\item
  Applies to trials concerning sexual offences.
\item
  Preliminary complaints can be admitted regardless of when made.
\item
  Courts may exclude evidence if it is deemed unfair to the defendant.
\item
  Judges must not suggest any bias towards the reliability of the
  complainant's evidence based on the timing of their complaint.
\end{enumerate}

\subsubsection{Example of Complaints}\label{example-of-complaints}

\begin{itemize}
\tightlist
\item
  \textbf{Preliminary Complaints}: Complaints made shortly after the
  alleged offense (e.g., to parents or teachers).
\item
  \textbf{Non-Preliminary Complaints}: Formal statements made to police
  after legal proceedings have begun.
\end{itemize}
